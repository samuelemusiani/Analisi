\section{Insiemi}
\subsection{Simbologia}
Per indicare gli insiemi si utilizzano lettere maiuscole ($A, B, X, Y ...$). 
Per indicare gli elementi appartenenti ad un insieme si usano le lettere 
minuscole ($a, b, x, y ...$). Di seguito una lista di simboli necessari per 
lavorare con gli insiemi e la logica in generale:

\begin{multicols}{2}

\begin{itemize}
    \item $\in$ \qquad Appartiene
    \item $\notin$ \qquad Non appartiene
    \item $\forall$ \qquad Per ogni
    \item $:$ \textit{or} $|$ \qquad Tale che
    \item $\exists$ \qquad Esiste (almeno)
    \item $\nexists$ \qquad Non esiste
    \item $\exists \oc$ \qquad Esiste ed è unico
    
    \item $\subseteq$ \qquad Inclusione insiemi
    \item $\subsetneqq$ \qquad Inclusione stretta (cioè $A \subseteq B$ e $A 
        \neq B$)
    \item $\nsubseteq$ \qquad Non incluso
    
    \item $\cup$ \qquad Unione
    \item $\cap$ \qquad Intersezione
    \item $\emptyset$ \qquad Insieme vuoto
    \item $\setminus$ \qquad Meno tra insiemi
    \item $\mathbb{U}$ \qquad Insieme universale
    \item $\complement \left(A\right)$ \qquad Complementare di $A$ in 
        $\mathbb{U}$, cioè tutti gli elementi di $\mathbb{U}$ che non sono in 
        $A$ ($\complement \left(A\right) = \mathbb{U} \setminus A$).
    
    \item $\lor$ \qquad OR logico
    \item $\land$ \qquad AND logico
    \item $\implies$ \qquad Implica
    \item $\iff$ \qquad Coimplica, "se e solo se" 
\end{itemize}
\end{multicols}

\subsection{Proposizioni e utilizzo di simboli logici}

Le \textbf{proposizioni} sono "frasi" che hanno un valore di verità, cioè posso 
essere vere o false. Per legare due o più proposizioni si utilizzano svariati 
simboli logici. In generale la logica proposizionale si studia nel corso di 
logica quindi non ho intenzione di riportare più dello stretto necessario qui.

\begin{table}[H]
\centering
\begin{tabular}{|cc|c|c|c|}
    \hline
    p & q & \multicolumn{1}{l|}{p $\land$ q} & \multicolumn{1}{l|}{p $\lor$ q} 
        & \multicolumn{1}{l|}{p $\implies$ q} \\ \hline
    {\color[HTML]{000000} false} & {\color[HTML]{000000} false} & false & false 
        & true \\
    {\color[HTML]{000000} false} & {\color[HTML]{000000} true} & false & true 
        & true \\
    {\color[HTML]{000000} true} & {\color[HTML]{000000} false} & false & true 
        & false \\
    true & false & true & true & true \\ \hline
\end{tabular}

\caption{Tabella di verità per AND, OR e IMPLICAZIONE}
\end{table}

Nell'implicazione ($p \implies q$) $p$ è condizione sufficiente per $q$, mentre 
$q$ è condizione necessaria per $p$.

Se si vuole negare una proposizione si utilizza il simbolo di negazione ($^-$) 
oppure il classico $\neg$, quindi $p$ negato risulta $\overline{p}$ oppure 
$\neg p$. Di seguito alcune negazioni di simboli generali:
\begin{itemize}
    \item $\neg \forall = \exists$
    \item $\neg \exists = \forall$
\end{itemize}
Attenzione che i simboli $\not \exists$ e $\neq \exists$ hanno significati 
diversi. 

È interessante notare che se si ha un'implicazione, si negano entrambe le 
proposizioni e si scambiano i termini il risultato rimane invariato.

\begin{table}[H]
\centering
\begin{tabular}{|cc|c|c|}
\hline
    p & q & \multicolumn{1}{l|}{$p \implies q$} & \multicolumn{1}{l|}
    {$\overline{q} \implies \overline{p}$} \\ \hline
    {\color[HTML]{000000} false} & {\color[HTML]{000000} false} & false 
    & false \\
    {\color[HTML]{000000} false} & {\color[HTML]{000000} true} & false 
    & true \\
    {\color[HTML]{000000} true} & {\color[HTML]{000000} false} & false 
    & true \\
    true & false & true & true \\ \hline
\end{tabular}
\caption{Implicazione con termini negati e invertiti}
\end{table}

Avendo la stessa tabella di verità $p \implies q = \overline{q} \implies 
\overline{p}$. Da qui nasce la \textbf{dimostrazione per negazione} (o 
contronominale). Esempio: se vogliamo dimostrare che se $n^2$ \textit{è pari} 
allora $n$ \textit{è pari}, ci basta dimostrare che se $n$ \textit{è dispari} 
allora $n^2$ \textit{è dispari}:
\begin{equation*}
	n \; \text{dispari} \implies n^2 \; \text{dispari}
\end{equation*}
Assumiamo che $n$ sia dispari per dimostrare che $n^2$ è dispari. Per ipotesi:
\begin{equation*}
	\exists k \in \mathbb{N}: n = 2k + 1
\end{equation*}
Ne consegue che:
\begin{equation*}
	n^2 = (2k+1)^2 = 4k^2 + 4k + 1 = 2(2k^2 + 2k) + 1
\end{equation*}
Essendo $(2k^2 + 2k) \in \mathbb{N}$ ne consegue che:
\begin{equation*}
	2(2k^2 + 2k) + 1 \;\; \text{è dispari}
\end{equation*}
\hfill Qed.


\subsection{Insiemi numerici noti}

\subsubsection{Naturali, interi e razionali}

Il primo insieme che si introduce è l'insieme dei \textbf{numeri naturali}, 
indicato con $\mathbb{N}$. Non definiremo formalmente l'insieme, ci basti 
sapere che comprende tutti i numeri interi positivi. È generalmente scritto 
nella forma:
\begin{center}
    $\mathbb{N} = \{0, 1, 2, 3, ...\}$
\end{center}

Il secondo insieme che va introdotto è quello dei \textbf{numeri interi}. 
Questo insieme comprende tutti i numeri che non hanno la virgola (per l'appunto 
interi) e si distingue da $\mathbb{N}$ in quanto ha al suo interno anche i 
numeri negativi. Il suo simbolo è $\mathbb{Z}$ e viene generalmente scritto 
come segue:
\begin{center}
    $\mathbb{Z} = \{..., -2, -1, 0, 1, 2, ...\}$
\end{center}

Non sempre però è comodo descrivere gli insiemi elencando il loro termini (in 
realtà non viene quasi mai fatto in quanto si creerebbero delle ambiguità). È 
necessario quindi utilizzare una notazione diversa per descrivere un insieme, 
ed è quello che si usa anche per introdurre $\mathbb{Q}$, l'insieme dei 
\textbf{numeri razionali}:
\begin{center}
    \begin{equation*}
        \mathbb{Q} = \left\{\,\dfrac{p}{q}\, \middle| \, p, q \in \mathbb{Z},\, 
        q \neq 0 \right\}
    \end{equation*}
\end{center}
In questo caso l'insieme dei numeri razionali indica appunto l'insieme di tutte 
le frazioni\footnote{Matematici rigorosi abbiate pietà della mia anima}. E per 
esplicitarlo non cominciamo ad elencare tutte le frazioni, ma bensì utilizziamo 
una proposizione.

\subsubsection{Reali}
L'ultimo insieme che introduciamo è l'insieme dei \textbf{numeri reali}. Viene 
indicato con il simbolo $\mathbb{R}$ ed è l'insieme più importante di tutti i 
precedenti. Questo insieme viene introdotto perché all'insieme dei numeri 
razionali ($\mathbb{Q}$) mancano dei numeri.

Proviamo infatti a prendere una retta e a inserirci tutti i numeri che 
conosciamo per ora (cioè quelli contenuti in Q). Costruiamo ora un quadrato di 
lato 1, e proviamo a calcolare la lunghezza della sua diagonale con il teorema 
di Pitagora. Il numero che otterremo sarà $\sqrt{1^2+1^2} = \sqrt{2}$. Se 
proiettiamo questo numero sulla retta vediamo chiaramente che esiste un punto 
che corrisponde a $\sqrt{2}$.

\begin{figure}[h]
	\begin{center}

	\begin{tikzpicture}[scale=4]

		\draw[->] (0, 0) -- (2, 0);
		\draw[->] (0, 0) -- (0, 1.5);
		\draw[-] (1, 0) -- (1, 1);
		\draw[-] (0, 1) -- (1, 1);
		\draw[-] (0, 0) -- node[left, yshift=5pt] {$\sqrt{2}$} (1, 1);
			

		\draw (0,0) -- (1.414,0) arc(0:45:1.414);


		\draw[very thin, fill] (0,0) circle (0.4pt) node[left, yshift=10pt] 
            {$0$};
		\draw[very thin, fill] (1,0) circle (0.4pt) node[left, yshift=10pt] 
            {$1$};
		\draw[very thin, fill] (1.414,0) circle (0.4pt) node[right, 
            yshift=10pt] {$\sqrt{2}$};
	\end{tikzpicture}

	\end{center}
	\caption{$\sqrt{2}$ derivata dalla lunghezza della diagonale di un 
    quadrato}
\end{figure}


Questo punto è contenuto nell'insieme $\mathbb{Q}$? 

\pf{
	Dobbiamo provare che $\sqrt{2} \not \in \mathbb{Q}$. Assumiamo che sia 
    contenuto e riduciamoci a dimostrare l'assurdo. Se $\sqrt{2} \in 
    \mathbb{Q}$ significa che:
	\begin{equation*}
        \exists\, m, n, \in \mathbb{N} : \dfrac{m}{n} = \sqrt{2}
	\end{equation*}
	Cioè esistono due numeri appartenenti all'insieme dei numeri naturali tali 
    che il loro rapporto è esattamente $\sqrt{2}$. Ovviamente essendo una 
    frazione possiamo semplificare $m$ e $n$ fino ad eliminare tutti i loro 
    fattori comuni, cioè fino a farli diventare \textbf{coprimi} o meglio 
    $\mathrm{M.C.D}(m,n) = 1$.

	\begin{equation*}
        \dfrac{m}{n} = \sqrt{2} \implies \dfrac{m^2}{n^2} = \sqrt{2} 
        \implies m^2 = 2n^2
	\end{equation*}
	Possiamo quindi dedurre che $m^2$ è pari,  e quindi $m$ è pari\footnote{Lo 
    abbiamo visto qualche pagina sopra come esempio di dimostrazione per 
    negazione}.
	Quindi $\exists\, k \in \mathbb{N} : m = 2k$.
	\begin{equation*}
	m^2 = 2n^2 \implies (2k)^2 = 2n^2 \implies 4k^2 = 2n^2 \implies 2k^2 = n^2
	\end{equation*}
	Quindi se $n^2$ è pari, anche $n$ è pari.\\

	Abbiamo dimostrato che sia $m$ che $n$ sono pari, eppure avevamo imposto 
    che $m$ e $n$ non avessero fattori in comune. Assurdo! Quindi:
	\begin{equation*}
		\sqrt{2} \not \in \mathbb{Q}
	\end{equation*}

	\hfill Q.e.d.
}

Come abbiamo dimostrato $\sqrt{2}$ non appartiene all'insieme dei numeri 
razionali. Si può estendere quesa affermazione con il seguente teorema:
\thm{
    Sia $p \in \mathbb{N}$ un numero primo, allora $\sqrt{p} \not \in 
    \mathbb{Q}$
}
\pf{
    La dimostrazione è analoga al caso in cui $p = 2$. Si assume per assurdo 
    che:
    \begin{equation*}
        \exists m,n \in \mathbb{N} : \sqrt{p} = \dfrac{m}{n}
    \end{equation*}
    Assumiamo che $m$ e $n$ siano coprimi in quanto è sepre possibile 
    semplificare fino a farli diventare coprimi. In altre parole M.C.D$(m, n) 
    = 1$.
    Vale quindi che:
    \begin{equation*}
        p = \dfrac{m^2}{n^2} \implies n^2 \cdot p = m^2
    \end{equation*}
    Tutti i fattori di $m^2$ sono i fattori di $m$, quindi $p$ è un fattore di 
    $m$. $\exists k \in \mathbb{N}: m = kp$. Sostituende risulta:
    \begin{equation*}
        m^2 = p \cdot n^2 \implies (kp)^2 = p \cdot n^2 \implies k^2 \cdot 
        p^2 = p \cdot n^2 \implies k^2 \cdot p = n^2
    \end{equation*}
    Per la stessa osservazione fatta prima, questo significa che $p$ divide 
    $n$. Ma dividendo anche $m$ risulta che M.C.D$(m, n) = p$. Ed essendo $p$ 
    un numero primo $p \neq 1$. Assurdo! Quindi:
    \begin{equation*}
        \sqrt{p} \not \in \mathbb{Q}
    \end{equation*}

    \hfill Q.e.d.
}
Quindi tutti i numeri primi non appartengono all'insieme dei numeri razionali. 
Ma quanti numeri primi esistono?
\thm{
    I numeri primi sono infiniti
}
\pf{
    La dimostrazione è per assurdo quindi supponiamo che i numeri primi siano 
    finiti. Elenchiamoli in ordine crescente:
    \begin{equation*}
        p_1 < p_2 < p_3 < \cdots < p_n
    \end{equation*}
    Definiamo ora il numero: $m \vcentcolon = p_1 \cdot p_2 \cdots p_n + 1$. 
    Facciamo vedere che $m$ produce una contraddizione: $m > p_n$ quindi 
    essendo $p_n$ il massimo numero primo $m$ deve per forza essere non primo. 
    Essendo non primo $m$ deve per forza essere divisibile per almeno uno dei 
    numeri primi $p_1, p_2, \cdots, p_n$. Notiamo però che preso un qualsiasi 
    primo $p_i$, $m$ darà sempre resto $1$:
    \begin{equation*}
        m = p_i \cdot (p_1 \cdots p_{i - 1} \cdot p_{i + 1} \cdots p_n) + 1
    \end{equation*}
    Non essendo quindi divisibile per nessuno dei numeri primi $m$ deve per 
    forza essere primo, ma questo è assurdo in quanto $m$ non era primo per 
    l'osservazione fatta in precedenza.
    \hfill{Q.e.d.}
}

Ne consegue che esistono infiniti punti sulla retta che non appartengono a 
$\mathbb{Q}$, e anzi è molto più probabile che facendo la radice di un numero 
si ottenga un numero che non è incluso in $\mathbb{Q}$ piuttosto che il 
contrario. Quindi se si vuole formalizzare una teoria dei limiti è necessario 
lavorare su un insieme non "bucato", cioè un insieme \textbf{continuo}.
\bigbreak

La proprietà che infatti distingue e differenzia $\mathbb{R}$ da $\mathbb{Q}$ 
è proprio la continuità (non ci sono "buchi") e la completezza (tutti i punti 
sulla retta hanno associato un unico numero reale). Esiste infatti una 
\textbf{corrispondenza biunivoca} che associa tutti i punti della retta ai 
numeri reali.\footnote{Per approfondire il concetto di \textit{corrispondenza 
biunivoca} è necessario fare riferimento al capitolo legato alle funzioni.}
Per definire meglio la proprietà di completezza è necessario introdurre alcuni 
concetti:

\dfn{
    Dato $A \neq \emptyset$ e $A \subseteq R$
    \begin{enumerate}
        \item $M \in \mathbb{R}$ si dice \textbf{maggiorante} di $A$ se: 
            $\forall\, a \in A : a \leq M$.
        \item $m \in \mathbb{R}$ si dice \textbf{minorante} di $A$ se: 
            $\forall\, a \in A : m \leq a$.
        
        \item Se $A$ ammette un maggiorante è \textbf{superiormente limitato}.
        \item Se $A$ ammette un maggiorante è \textbf{inferiormente limitato}.
        \item Se $A$ ammettere un maggiorante e un minorante è 
            \textbf{limitato}.
        
        \item $b \in A$ si dice \textbf{massimo} di $A$ se $\forall\, a 
            \in A : a \leq b$.
        \item $c \in A$ si dice \textbf{minimo} di $A$ se $\forall\, a 
            \in A : c \leq a$.
        
        \item Il più piccolo dei maggioranti si chiama \textbf{estremo 
            superiore} di $A$. Si indicato con $\mathbf{sup}\,A$. Se il massimo 
            esiste coincide con l'estremo superiore. Se l'insieme è 
            superiormente illimitato si scrive $\mathbf{sup}\,A = +\infty$.
        \item Il più grande dei minoranti si chiama \textbf{estremo inferiore} 
            di $A$. Si indicato con $\mathbf{inf}\,A$. Se il minimo esiste 
            coincide con l'estremo inferiore. Se l'insieme è inferiormente 
            illimitato si scrive $\mathbf{inf}\,A = -\infty$.
        
        \item L'\textbf{insieme dei maggioranti} di $A$ si indica come 
            $\mathrm{Mg}(A) = \{n \in \mathbb{R}\,|\, n$ è un maggiorante di 
            $A\}$. Il minimo di questo insieme coincide con l'estremo superiore 
            di $A$.
        \item L'\textbf{insieme dei minoranti} di $A$ si indica come 
            $\mathrm{Mn}(A) = \{n \in \mathbb{R}\,|\, n$ è un minorante di 
            $A\}$. Il massimo di questo insieme coincide con l'estremo 
            inferiore di $A$.
    \end{enumerate}
}
La proprietà di \textbf{completezza} \label{sec_completezzaR} di $\mathbb{R}$ è 
quindi formalizzata nella seguente forma:
\imp{\begin{center}
    Dato un insieme limitato, esiste sempre un estremo inferiore e un estremo 
    superiore.
\end{center}}

Per esempio, dato l'insieme $A = \{q \in \mathbb{Q}\,|\,q^2 < 2\}$ non è 
possibile trovare un estremo superiore e un estremo inferiore in quanto 
l'intervallo (sezione \ref{sec_intervalli}) dell'insieme sarebbe $]-\sqrt{2}, 
\sqrt{2}[$ e, come dimostrato precedentemente, $\sqrt{2} \notin \mathbb{Q}$. 
Se invece si considera l'insieme $B = \{q \in \mathbb{R}\,|\,q^2 < 2\}$, è 
facile notare che $\mathrm{sup}\,A = \sqrt{2}$ e che $\mathrm{inf}\, A = 
-\sqrt{2}$.

\subsection{Intervalli} \label{sec_intervalli}
Per indicare gli intervalli generalmente si utilizza una notazione con le 
parentesi. Viene utilizzata la parentesi quadra ($[$ e $]$) per indicare 
rispettivamente se un estremo dell'intervallo è compreso o no. In particolare 
se l'apertura della parentesi è rivolta verso il numero, allora tale numero è 
compreso. Di seguito degli esempi per chiarire. Gli intervalli sono comunque 
un insieme di punti e quindi questo si può tradurre:
\begin{multicols}{2}
\begin{itemize}
    \item $[a, b] = \left\{ x \in \mathbb{R}\, \middle|\, a \leq x 	\leq 
        b\right\}$
    \item $]a, b] = \left\{ x \in \mathbb{R}\, \middle|\, a < x \leq b\right\}$
    \item $[a, b[ = \left\{ x \in \mathbb{R}\, \middle|\, a \leq x < b\right\}$
    \item $]a, b[ = \left\{ x \in \mathbb{R}\, \middle|\, a < x < b\right\}$
    
    \item $]-\infty, a] = \left\{ x \in \mathbb{R}\, \middle|\, a \leq 
        x \right\}$
    \item $[b, +\infty[ = \left\{ x \in \mathbb{R}\, \middle|\, x \geq 
        b \right\}$
    \item $]-\infty, +\infty[ = \mathbb{R}$
\end{itemize}
\end{multicols}
Con infinito si utilizza sempre la parentesi di esclusione. In alcuni libri 
viene utilizzata però la parentesi tonda: al posto di $[a,+\infty[$ viene 
scritto $[a. +\infty)$. La notazione è del tutto equivalente.

\subsection{Prodotto cartesiano}
\dfn{
    Dati due insiemi $A$ e $B$, il \textbf{prodotto cartesiano tra $A$ e $B$} 
    si definisce come:
    \begin{equation*}
        A\times B = \{(a, b)\,|\,a \in A \land b \in B\}
    \end{equation*}
}

È quindi l'insieme di coppie ordinate ($(1, 0) \neq (0, 1)$). Attenzione che 
non vale la proprietà commutativa: $A\times B \neq B\times A$. Un prodotto 
cartesiano con lo stesso insieme si può abbreviare come $A^2 = A\times A$.
\bigbreak 

$\mathbb{R}$ = Retta. $\mathbb{R}^2$ = Piano cartesiano. $\mathbb{R}^3$ = 
Sistema di coordinate a tre dimensioni. %FARE GRAFICO

\subsection{Cardinalità}
La cardinalità, per essere introdotta, ha bisogno di alcune nozioni riguardo le 
funzioni. È consigliato quindi andare a vedere la sezione \ref{sec_funzioni} e 
poi tornare qui.\bigbreak

\dfn{
    Dati due insiemi $A$ e $B$, se esiste una funzione biunivoca $f: A \to B$ 
    si dice che \textbf{$A$ e $B$ sono equipotenti}.
}

\dfn{
    Due insiemi hanno la stessa cardinalità se sono equipotenti. Per indicare 
    la cardinalità dell'insieme $A$ si scrive:
    \begin{equation*}
        |A|
    \end{equation*}
}

La cardinalità, in parole povere, è un modo per "misurare" e classificare gli 
insiemi in base a quanti elementi contengono. Se infatti abbiamo tre insiemi 
finiti $A=\{1, 3, 5\}$, $B=\{2, 4, 6\}$ e $C=\{1, 2\}$ possiamo dire che A e C 
NON hanno la stessa cardinalità ($|A|=3 \neq |C| = 2$), mentre A e B hanno la 
stessa cardinalità ($|A| = |B| = 3$). Quindi la cardinalità riguardo insiemi 
\textit{finiti} si limita a misurare gli elementi che contengono gli insiemi 
stessi, mentre la cardinalità con insiemi \textit{infiniti} ci permette di 
capire quali insiemi contengono più elementi di altri (nonostante siano 
infiniti).\bigbreak

\subsubsection{Cardinalità del numerabile}
La cardinalità infinita più piccola è la \textbf{cardinalità del numerabile}, 
cioè la stessa cardinalità dell'insieme dei numeri naturali ($|\mathbb{N}| = 
numerabile$).

Ora prendiamo in considerazione il secondo insieme che abbiamo introdotto 
subito dopo $\mathbb{N}$, cioè $\mathbb{Z}.$ L'intuito ci dice che la 
cardinalità dell'insieme dei numeri interi dovrebbe essere doppia rispetto a 
quella dei naturali, in quanto ci sono esattamente il doppio dei numeri (cioè 
anche i negativi oltre ai positivi). Proviamo però a definire una funzione $f: 
\mathbb{N} \to \mathbb{Z}$ tale che:

\begin{equation*}
	f(n) =
	\begin{cases}
		\dfrac{n}{2} \qquad \qquad \;\; \text{se $n$ è pari}\\
		-\dfrac{n+1}{2} \qquad \text{se $n$ è dispari}
	\end{cases}
\end{equation*}

I valori della funzione risultano quindi:
\begin{align*}
    f(0) = 0 \qquad \qquad f(1) = -1\\
    f(2) = 1 \qquad \qquad f(3) = -2\\
    f(4) = 2 \qquad \qquad f(5) = -3\\
    \vdots \qquad \qquad \qquad \qquad \vdots \qquad
\end{align*}

È facile notare che $f$ è biunivoca e, conseguentemente alle definizioni appena 
date, possiamo affermare che $\mathbb{Z}$ ha cardinalità del numerabile:
\begin{equation*}
    |\mathbb{Z}| = |\mathbb{N}|
\end{equation*}

Prendiamo ora in analisi l'insieme dei numeri razionali, ovvero $\mathbb{Q}$. 
Elenchiamo tutte le frazioni (e di conseguenza tutti i suoi elementi) nel 
seguente ordine:

\begin{figure}[H]
\centering
\begin{tikzpicture}
% the matrix entries
\matrix (mat) [table]
{
0 & 1 & 2 & 3 & 4 & 5 & 6 & \dots \\
-0 & -1 & -2 & -3 & -4 & -5 & -6 & \dots \\
$ +\frac{0}{2}$ & $ +\frac{1}{2}$ & $ +\frac{2}{2}$ & $ +\frac{3}{2}$ & 
    $ +\frac{4}{2}$ & $ +\frac{5}{2}$ & $ +\frac{6}{2}$ & \dots \\
$ -\frac{0}{2}$ & $ -\frac{1}{2}$ & $ -\frac{2}{2}$ & $ -\frac{3}{2}$ & 
    $ -\frac{4}{2}$ & $ -\frac{5}{2}$ & $ -\frac{6}{2}$ & \dots \\
$ +\frac{0}{3}$ & $ +\frac{1}{3}$ & $ +\frac{2}{3}$ & $ +\frac{3}{3}$ & 
    $ +\frac{4}{3}$ & $ +\frac{5}{3}$ & $ +\frac{6}{3}$ & \dots \\
$ -\frac{0}{3}$ & $ -\frac{1}{3}$ & $ -\frac{2}{3}$ & $ -\frac{3}{3}$ & 
    $ -\frac{4}{3}$ & $ -\frac{5}{3}$ & $ -\frac{6}{3}$ & \dots \\
\vdots &\vdots &\vdots &\vdots &\vdots &\vdots &\vdots &\\
};
% the matrix rules

% the arrows
\begin{scope}[shorten >=7pt,shorten <= 7pt]
\draw[->]  (mat-1-1.center) -- (mat-1-2.center);
\draw[->]  (mat-1-2.center) -- (mat-2-1.center);
\draw[->]  (mat-2-1.center) -- (mat-2-2.center);
\draw[->]  (mat-2-2.center) -- (mat-1-3.center);
\draw[->]  (mat-1-3.center) -- (mat-1-4.center);
\draw[->]  (mat-1-4.center) -- (mat-2-3.center);
\draw[->]  (mat-2-3.center) -- (mat-3-2.center);
\draw[->]  (mat-3-2.center) -- (mat-3-1.center);
\draw[->]  (mat-3-1.center) -- (mat-4-1.center);
\draw[->]  (mat-4-1.center) -- (mat-4-2.center);
\draw[->]  (mat-4-2.center) -- (mat-3-3.center);
\draw[->]  (mat-3-3.center) -- (mat-2-4.center);
\draw[->]  (mat-2-4.center) -- (mat-1-5.center);
\draw[->]  (mat-1-5.center) -- (mat-1-6.center);
\draw[->]  (mat-1-6.center) -- (mat-2-5.center);
\draw[->]  (mat-2-5.center) -- (mat-3-4.center);
\draw[->]  (mat-3-4.center) -- (mat-4-3.center);
\draw[->]  (mat-4-3.center) -- (mat-5-2.center);
\draw[->]  (mat-5-2.center) -- (mat-5-1.center);
\draw[->]  (mat-5-1.center) -- (mat-6-1.center);
\draw[->]  (mat-6-1.center) -- (mat-6-2.center);
\draw[->]  (mat-6-2.center) -- (mat-5-3.center);
\draw[->]  (mat-5-3.center) -- (mat-4-4.center);
\draw[->]  (mat-4-4.center) -- (mat-3-5.center);
\draw[->]  (mat-3-5.center) -- (mat-2-6.center);
\draw[->]  (mat-2-6.center) -- (mat-1-7.center);
\end{scope}
\end{tikzpicture}

\caption{Tabella numeri razionali}
\label{tab_NumeriRazionali}
\end{figure}

Non dimostreremo rigorosamente la cosa, però è facile notare che si può 
costruire una funzione biunivoca da $\mathbb{N}$ a $\mathbb{Q}$ che segua 
esattamente l'ordine delle frecce nella figura \ref{tab_NumeriRazionali}. 
Di conseguenza anche l'insieme dei numeri razionali ha \textbf{cardinalità del 
numerabile}:
\imp{
\begin{equation*}
    |\mathbb{N}| = |\mathbb{Z}| = |\mathbb{Q}|
\end{equation*}
}

\subsubsection{Cardinalità del continuo}
Concludiamo con l'analisi dei numeri reali, ovvero l'insieme $\mathbb{R}$. Se 
volessimo dimostrare che $\mathbb{R}$ non è numerabile, ci basterebbe 
dimostrare che un suo sottoinsieme non è numerabile.
\pf{
	Vogliamo dimostrare che $[0,1[$ non è numerabile perché questo ci 
    permetterebbe di concludere che $\mathbb{R}$ non è numerabile. Ci basta 
    quindi dimostrate che non esiste una funzione suriettiva da $\mathbb{N}$ a 
    $\mathbb{R}$:
	\begin{equation*}
		\not \exists f:N \xrightarrow[1-1]{SU} [0,1[
	\end{equation*}
	Supponiamo che esiste tale funzione e riduciamoci a dimostrare 
    l'assurdo\footnote{Il prof chiama questa dimostrazione \textit{una 
    dimostrazione per assurdo} quando chiaramente non lo è perché si suppone 
    l'esistenza della funzione da una semplice introduzione del not. La 
    differenza tra la dimostrazione per assurdo (RAA) e la not introduzione 
    sarà chiara soltanto quando farete deduzione naturale nel corso di logica. 
    Fino ad allora fregatevene di questa distinzione. La prova comunque non è 
    per assurdo, dando ragione al sommo CSC quando dice che "\textit{i 
    matematici si confondono sempre sulle dimostrazioni per assurdo}".}. 
    Essendo la funzione suriettiva, ogni numero nell'intervallo $[0,1[$ deve 
    avere una corrispondenza in $\mathbb{N}$. 
\begin{equation*}
	f(n) \in [0,1[
\end{equation*}
Quindi:
\begin{align*}
	f(0) &= 0,b_{00} \; b_{01} \; b_{02} \; b_{03} \; b_{04} \; b_{05} \cdots\\
	f(1) &= 0,b_{10} \; b_{11} \; b_{12} \; b_{13} \; b_{14} \; b_{15} \cdots\\
	f(2) &= 0,b_{20} \; b_{21} \; b_{22} \; b_{23} \; b_{24} \; b_{25} \cdots\\
	f(3) &= 0,b_{30} \; b_{31} \; b_{32} \; b_{33} \; b_{34} \; b_{35} \cdots\\
	f(4) &= 0,b_{40} \; b_{41} \; b_{42} \; b_{43} \; b_{44} \; b_{45} \cdots\\
	f(5) &= 0,b_{50} \; b_{51} \; b_{52} \; b_{53} \; b_{54} \; b_{55} \cdots\\
	\vdots\\
	f(n) &= 0,b_{n0} \; b_{n1} \; b_{n2} \; b_{n3} \; b_{n4} \; b_{n5} \cdots
\end{align*}
Ci basta ora trovare un numero $r \in[0,1[$ tale che: $f(n) \neq r \; \forall n 
\in \mathbb{N}$ per provare l'assurdo. Costruiamo il numero $r$ usando un 
procedimento diagonale\footnote{Questo procedimento esiste grazie al matematico 
Cantor.}:
\begin{align*}
	r := 0,r_0 \; r_1 \; r_2 \; r_3 \; r_4 \cdots\\
	r_{j} = 
	\begin{cases}
		5 \qquad \mathrm{se}\;\,b_{jj} \neq 5\\
		6 \qquad \mathrm{se}\;\,b_{jj} = 5\\
	\end{cases}
\end{align*}
La costruzione è in diagonale perché consideriamo come indici della $b$ solo la 
coppia $jj$, che quindi produce uno spostamento regolare $00, 11, 22, \cdots$. 
In pratica prendiamo una cifra da ogni numero che abbiamo già e la cambiamo. 
In questo modo tutti i numeri che abbiamo differiranno di almeno una cifra da 
quello che stiamo costruendo. Questa particolare costruzione implica che:
\begin{equation*}
	r_j \neq b_{jj} \quad \forall j \in \mathbb{N}
\end{equation*}
Nonostante ciò $r \in [0,1[$. Abbiamo quindi provato l'assurdo in quanto:
\begin{equation*}
	r \neq f(n) \quad \forall n \in \mathbb{N}
\end{equation*}
E quindi la funzione non è suriettiva.
\hfill Qed.

}
\imp{
	\begin{center}
		Quindi $|\mathbb{R}|$ non è numerabile, infatti: $\mathbb{R}$ ha 
        \textbf{cardinalità del continuo}.
	\end{center}
}
