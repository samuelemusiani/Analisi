\section{Successioni e serie} \label{sec_successioni}
\dfn{
    Una successione di numeri reali è una funzione:
    \begin{gather*}
        f: \mathbb{N} \to \mathbb{R}\\
        n \to f(n) = \vcentcolon a_n
    \end{gather*}
    E si indicato in tre possibili modi:
    \begin{equation*}
        (a_n)_{n \in \mathbb{N}},\;\, (a_n)_n\;\, \text{oppure}\;\, (a_n)
    \end{equation*}
}
L'idea alla base delle successioni matematiche è poter scrivere una lista di 
numeri che però abbiano un ordine. Avendole infatti definite come funzioni si 
può tenere l'ordine dei termini della successione semplicemente:
\begin{gather*}
    f(0) = a_0\\
    f(1) = a_1\\
    f(2) = a_2\\
    f(3) = a_3\\
    \vdots
\end{gather*}
Il poter scrivere numeri in successione e dare loro un ordine sembra 
apparentemente sembra inutile, ma in realtà si rivela estremamente utile in 
moltissime definizioni di funzioni che però vanno estese. Per esempio, come è 
spiegato nella sezione \ref{sec_esponenziale} riguardo all'esponenziale, per 
definire tale funzione per tutti i numeri reali è necessario utilizzare una 
successione.\\

\dfn{
    Data una successione $(a_n)_n$ e un insieme $A = \{ a_n | n \in 
    \mathbb{N}\}$. $(a_n)$ si dice:
    \begin{itemize}
        \item \textbf{Superiormente limitata} se $A$ è superiormente limitato.
        \item \textbf{Inferiormente limitata} se $A$ è inferiormente limitato.
        \item \textbf{Limitata} se $A$ è limitato.
    \end{itemize}
}

\subsection{Successioni monotone}
Per capire quanto segue è necessario prima guardare la sezioni sui limiti delle 
successioni (\ref{sec_lim_successioni}).
\dfn{
$(a_n)_n$ si dice \textbf{crescente} e si indica con $(a_n \nearrow)$ se:
\begin{equation*}
    \forall n \in \mathbb{N} : a_n \leq a_{n+1}
\end{equation*}
}
\dfn{
$(a_n)_n$ si dice \textbf{strettamente crescente} e si indica sempre con 
$(a_n \nearrow)$ se:
\begin{equation*}
    \forall n \in \mathbb{N} : a_n < a_{n+1}
\end{equation*}
}
\dfn{
$(a_n)_n$ si dice \textbf{decrescente} e si indica con $(a_n \searrow)$ se:
\begin{equation*}
    \forall n \in \mathbb{N} : a_n \geq a_{n+1}
\end{equation*}
}
\dfn{
$(a_n)_n$ si dice \textbf{strettamente decrescente} e si indica sempre con 
$(a_n \searrow)$ se:
\begin{equation*}
    \forall n \in \mathbb{N} : a_n > a_{n+1}
\end{equation*}
}

Una successione crescente o decrescente si dice \textbf{monotona}. Le 
successioni monotone \textbf{hanno sempre limite}.

\thm{
Se $(a_n)_n$ è \textbf{crescente}, allora:
\begin{equation*}
    \lim_{n\to +\infty} a_n = \sup \{a_n | n \in \mathbb{N}\}
\end{equation*}
Se $(a_n)_n$ è \textbf{decrescente}, allora:
\begin{equation*}
    \lim_{n\to +\infty} a_n = \inf \{a_n | n \in \mathbb{N}\}
\end{equation*}
}
Dimostriamo il teorema appena enunciato:
\pf{
Dimostreremo il teorema soltanto per $(a_n)_n \nearrow$. Si tratta quindi di 
provare:
\begin{equation*}
    \lim_{n\to +\infty} a_n = \sup \{a_n | n \in \mathbb{N}\}
\end{equation*}
Poniamo $L \coloneq \sup \{a_n | n \in \mathbb{N}\}$
Bisogna dimostrare due casi: $L = +\infty$ e $L \in \mathbb{R}$
\begin{enumerate}
    \item $\mathbf{L = + \infty}$
        Ci riduciamo a provare che 
        \begin{equation*}
            \lim_{n\to +\infty} a_n = +\infty
        \end{equation*}
        cioè
        \begin{equation*}
            \forall K > 0, \exists \bar{n} \in \mathbb{N} : \forall n \geq 
            \overline{n} : a_n > K
        \end{equation*}
        Dato $A = \{a_n | n \in \mathbb{N}\}$
        \begin{itemize}
            \item A non è superiormente limitato in quanto $L = +\infty$
            \item A di conseguenza non ammette maggioranti
            \item Ne consegue che K non è un maggiorante $\implies \exists 
              a_{\bar{n}} \in A : a_{\bar{n}} > K$
            \item Siccome per ipotesi $(a_n)_n \nearrow \implies \forall n \geq 
              \bar{n} : a_n \geq a_{\bar{n}} > K$ 
        \end{itemize}
    \item $\mathbf{L \in \mathbb{R}}$
    Ci riduciamo a provare che 
        \begin{equation*}
            \lim_{n\to +\infty} a_n = L
        \end{equation*}
        cioè
        \begin{align*}
            \forall \epsilon > 0, \exists \bar{n} \in \mathbb{N} : \forall n 
            \geq \overline{n} : |a_n& - L| < \epsilon\\
            &\Big\Updownarrow\\
            L - \epsilon < &a_n < L + \epsilon
        \end{align*}
        $a_n < L + \epsilon$ è ovvio per ipotesi in quanto $L$ è un maggiorante, 
        quindi $\forall n \; a_n \leq L < L + \epsilon$. Ci rimane quindi solo 
        da trovare $\bar{n}$ tale che:
        \begin{equation*}
            \forall n \geq \bar{n} : a_n > L - \epsilon
        \end{equation*}
        Essendo $L = \sup \{a_n | n \in \mathbb{N}\}$
        \begin{itemize}
            \item $L$ è il più piccolo dei maggioranti di $A$
            \item $L - \epsilon$ non è quindi un maggiorante di $A$ $\implies 
              \exists a_{\bar{n}} \in A : L -\epsilon < a_{\bar{n}}$
            \item Essendo $(a_n)_n \nearrow$ si ha che $\forall n \geq \bar{n} 
              \;\; L-\epsilon < a_{\bar{n}} \leq a_{n}$
        \end{itemize}
        quindi
        \begin{equation*}
            a_n > L - \epsilon
        \end{equation*}
        \hfill Q.e.d.
\end{enumerate}
}
\textbf{Corollari} \label{corol_successioni}
\begin{enumerate}
    \item Se $(a_n)_n \nearrow$ e $(a_n)_n$ è \textbf{superiormente limitata}, 
      allora $(a_n)_n$ è \textbf{convergente}, cioè
        \begin{equation*}
            \exists r \in \mathbb{R} : a_n \xrightarrow{\quad} r
        \end{equation*}
    \item Se $(a_n)_n \searrow$ e $(a_n)_n$ è \textbf{inferiormente limitata}, 
      allora $(a_n)_n$ è \textbf{convergente}, cioè
        \begin{equation*}
            \exists r \in \mathbb{R} : a_n \xrightarrow{\quad} r
        \end{equation*}
\end{enumerate}
