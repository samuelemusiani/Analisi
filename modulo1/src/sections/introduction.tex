\section{Introduzione agli appunti}

Non è mia intenzione dedicare molto spazio all'introduzione di questi appunti 
in quanto, più lunga diventa, più il rischio che nessuno la legga aumenta. 
Nonostante ciò ci sono alcune cose che vanno riportare. 

La prima è \textbf{la natura di questi appunti}. Essi infatti vogliono essere 
un aiuto allo studio, ma più in generale vogliono cercare di dare le idee che 
risiedono dietro a molti argomenti e che purtroppo vengono tralasciate nella 
fretta di insegnare la materia. Ci sono quindi alcune sezioni dedicate 
interamente alla spiegazione di un concetto. Se uno lo conosce già può saltarle 
completamente e passare direttamente alla definizione formale. Nonostante ciò 
io consiglio di leggerlo, perché se si ha una idea di cosa si sta facendo 
diventa tutto più semplice.\bigbreak

La seconda è \textbf{potrebbero esserci degli errori}. In generale ho cercato 
di minimizzare gli errori rileggendo svariate volte, ma purtroppo non credo di 
essere riuscito nell'intento miracoloso di correggerli tutti. \textbf{Non mi 
assumo nessuna diretta responsabilità riguardo eventuali informazioni errate 
presenti all'interno di questi appunti}.

\subsection{Le varie parti degli appunti}
Ci sono vari box colorati che introducono rispettivamente una definizione, una 
dimostrazione, un lemma e una cosa importante:

\dfn{
	Questa è una definizione
}

\pf{
	Questa è una dimostrazione
}

\mlem{
	Questo è un lemma
}

\imp{
	Questa è una cosa importante
}


\subsection{Segnalare errori e contribuire}
Se per caso si trovano errori \textbf{vi prego di segnalarli ad uno dei 
contatti che seguono}. Se li segnalerete eviterete che altri studenti, nella 
lettura di questi appunti, imparino una informazione errata. Vi prego quindi di 
segnalarlo, ci mettete veramente poco tempo. Per quanto mi riguarda farò in 
modo di fixare quello che mi verrà riportato il prima possibile.\bigbreak

Se volete contribuire agli appunti con aggiunte, modifiche o suggerimenti 
potete scrivermi o mandare direttamente una \textit{pull request} alla 
repository (\href{https://github.com/musianisamuele/Analisi}{link}) sul mio 
profilo GitHub.\bigbreak


\textbf{Email:} samuele.musiani@studio.unibo.it\bigbreak

\textbf{Telegram:} @Tastier\bigbreak

\textbf{Github:} \href{https://github.com/musianisamuele}{musianisamuele}
\hfill Ultima modifica: \textbf{\today}
