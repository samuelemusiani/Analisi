\section{Derivate}

% P. 64 Analisi_pt2. La parte prima sarebbe tutta una analisi sulle rette e i suoi coefficienti. Io la stringerei un po' e la farei meglio.


\dfn{
	Dato un intervallo $I \in \mathbb{R}$, $x_0 \in I$ si dice \textbf{punto interno} a $I$ se \textbf{esiste un intorno sferico:}
	\begin{equation*}
		B_r(x_0) = \{x \in \mathbb{R} \;| \; |x-x_0| < 0\}
	\end{equation*}
	tale che:
	\begin{equation*}
		B_r(x_0) \subseteq I
	\end{equation*}
}
L'insieme dei punti interni si indica con un piccolo \textit{tondino} in alto:
\begin{equation*}
	\mathring{I} := \{x \in I \; | \;\; \text{x è un punto interno a}\;I\} 
\end{equation*}
Questi punti sono diversi rispetto ai punti di accumulazione dei limiti per un paio di motivi. Il primo è che devono appartenere innanzitutto all'insieme. Se infatti si considera un insieme del tipo $\mathbb{R} \setminus \{0\}$, lo $0$ è un punto di accumulazione pur non appartenendo all'insieme. Non è però un punto interno. La seconda differenza principale con i punto di accumulazione è che, mentre questi ultimi possono essere "al bordo" di un insieme, i punti interni no, e devono avere l'intorno sferico completamente interno all'insieme. Cioè preso l'intervallo $[2, 4]$, mentre sia $2$ sia $4$ sono punti di accumulazione, nessuno dei due è interno all'insieme. In questo caso i punti interni a questo insieme sono $]2,4[$.

\subsection{Introduzione informale}
Pirma di dare la definizione formale di derviata vorrei spendere qualche riga per cercare di spiegare l'idea che si trova dietro a questo potentissimo oggetto matematico. Il problema è che spesso ci si perde nelle definzioni formali e negli esercizi, senza apprezzare veramente quello che si sta facendo. Perché alla fine si spera che ognuno, dopo aver seguito un corso di analisi, sappia che la derivata di $x^2$ è $2x$, oppure che la derivata del seno è il coseno, però veramente in pochi hanno capito cosa significa e perché è così.\\

\textbf{Derivata come tangente una curva:} Vi siete mai chiesti (probabilmente no ma la domanda retorica andava fatta) quale sia la retta tangete a una curva? Cioè, se io vi disegno una curva su un foglio e vi traccio un punto sopra e vi chiedo di disegnare la tangente a quella curva, probabilmente la richiesta non risulta così difficile. Magari la retta non verrà perfetta, però il disegno si avvicinerebbe parecchio ad una vera e propria tangente. Se però io invece di darvi un disegno di una curva vi do una funzione e un punto, trovare la tangente ora è molto più complicato perché dovreste prima disegnare la funzione e poi finalmente tracciare la tangente. Però poi nasce il problema di come si disegna una funzione in modo abbastanza preciso e quindi la situazione diventa piuttosto complicata. 

Facciamo un piccolo salto indietro e chiediamoci prima come si disegna una retta. Una retta, nella sua forma più classica, è data dalla seguente formula:
\begin{equation*}
	y = mx + q
\end{equation*}
Abbiamo quindi bisogno di due elementi per definirla: un coefficiente angolare $m$ e un termine noto $q$. Sappiamo però, dati due punti, trovare l'unica retta che passa per entrambi. Mettiamo di avere $A(x_a, y_a)$ e $B(x_b, y_b)$. Innanzitutto troviamo il coefficiente angolare della retta:
\begin{equation*}
	m = \dfrac{y_a - y_b}{x_a - x_b}
\end{equation*}
Fatto questo trovare $q$ non è molto difficile: basta inserire nella retta $m$ e uno qualsiasi dei due punti e risolvere l'equazione.

In realtà questo metodo di calcolare la retta presi due punti ci potrebbe venire molto comodo per calcolare la tangente ad una curva. Se infatti prendiamo due punti su di essa e poi li \textit{avviciniamo abbastanza} tra loro possiamo ottenere una buona approssimazione di una tangente (figura \ref{ApproxTangenteCurva}).



\begin{figure}[h]
\centering
\begin{tikzpicture}
\begin{axis}[xmax = 3, xmin = -0.5, ymax = 2, ymin = -0.5,
             axis lines=middle, xlabel=$x$, ylabel=$y$,
             xtick={10},
             ytick={10},
             xlabel style={anchor=north east},
             xticklabel style={anchor=north},
             yticklabel style={anchor=east},
             width=\linewidth,
            ]
\addplot[domain=-1:3, samples=200]{-0.2*x*x + 0.8 * x + 0.4};
\addplot[domain=-1:3, samples=200, dashed]{0.2 * x + 0.65};
\addplot[domain=-1:3, samples=200, dashed]{0.3 * x + 0.6};
\addplot[domain=-1:3, samples=200, dashed]{0.5 * x + 0.5};
\addplot[domain=-1:3, samples=200]{0.6 * x + 0.45};
\addplot [only marks, mark options={scale=1}] table {
0.5 0.75
1 1
2 1.2
2.5 1.15
};
\end{axis}
\end{tikzpicture}
  \caption{Approssimazione di una tangente con rette passanti per due punti} 
	\label{ApproxTangenteCurva}
\end{figure}

Il calcolo difficile non è tanto però il termine noto $q$, ma piuttosto il coefficiente angolare $m$. Questo perché dobbiamo prendere punti sempre più vicini e non è facile farlo venire preciso. Questo avvicinarci indefinitivamente ad un punto dovrebbe però far accendere una piccola lapadina in testa: abbiamo già uno strumento matematico che ci permette di avvicinarci indefinitivamente ad un punto: \textbf{il limite}. 

Prendiamo ora una funzione $f(x)$ per il momento generica\footnote{Per correttezza questa funzione non può essere generica perché non tutte le funzioni sono effettivamente derivabili, però rimaniamo così per il momento. In seguito viene definita rigorosamente.} e decidiamo di vole cacolare la tangente nel punto $x_0$. Come abbiamo appena visto ci servono due punti per calcolare la tangente: uno che resta fermo (in questo caso $x_0$) e uno che si avvicina a quest'ultimo. Per correttezza $x_0$ non è un punto, ma bensì $P(x_0, f(x_0))$. Il secondo punto che ci serve deve essere anchesso sulla funzione e quindi imponiamo questa condizione per la sua ordinata: $Q(x, f(x))$. Il coefficiente angolare tra i due risulta quindi:
\begin{equation*}
	m = \dfrac{Q_y - P_y}{Q_x - P_x} = \dfrac{f(x) - f(x_0)}{x-x_0}
\end{equation*}
Facendo in modo che il punto $Q$ si avvicini a $P$ indefinitamente:
\begin{equation*}
	\lim_{x \to x_0} \dfrac{f(x) - f(x_0)}{x-x_0}
\end{equation*}
E tadà, siamo arrivati alla definizione (informale) di derivata. La \textit{derivata della funzione $f$ nel punto $x_0$} è proprio il valore di quel limite, cioè il \textbf{coefficiente angolare della retta tangente alla funzione f nel punto $x_0$}.

%RIFLETTERE SUL LIMITE E FARE UN ESEMPIO (?)

%\textbf{Derivata come misura del cambiamento:} DA FAREEE











\subsection{Teoremi}
\subsubsection{Teorema di Fermat}
Il seguente teorema enuncia che la derivata nei minimi e nei massimi si annulla.
\thm {
Data una funzione $f: [a,b] \to \mathbb{R}$, un punto di massimo o di minimo $x_0 \in ]a,b[$ e inoltre $f$ è derivabile in $x_0$, allora:
\begin{equation*}
    f'(x_0) = 0
\end{equation*}
}
È essenziale che il punto $x_0$ si all'interno dell'intervallo! %DARE VEDERE FOTO DI UNA RETTA CON DOMIONIO RISTRETTO
Inoltre l'annullarsi della derivata prima in un punto $x_0$ è condizione \textbf{necessaria} affinché $x_0$ sia un punto di massimo o di minimo relativo, ma \textbf{non è sufficiente} in generale (es. $x^3$ in $x=0$). %FAR VEDERE GRAFICO

\subsubsection{Teorema di Rolle}
\thm {
$f:[a,b] \to \mathbb{R}$
\begin{enumerate}
    \item $f$ è continua su $[a,b]$
    \item $f$ è derivabile su $]a,b[$
    \item $f(a) = f(b)$
\end{enumerate}
Allora:
\begin{equation*}
    \exists \,c \in ]a,b[ \;: f'(c) = 0
\end{equation*}
}

\subsubsection{Teorema di Lagrange}
\thm {
$f:[a,b] \to \mathbb{R}$
\begin{enumerate}
    \item $f$ è continua su $[a,b]$
    \item $f$ è derivabile su $]a,b[$
\end{enumerate}
Allora:
\begin{equation*}
    \exists \,c \in ]a,b[ \;: \dfrac{f(b)-f(a)}{b-a} = f'(c)
\end{equation*}
}
Corollario: se una funzione ha derivata sempre zero è costante.

\subsubsection{Teorema di Cauchy}
\thm {
$f,g:[a,b] \to \mathbb{R}$
\begin{enumerate}
    \item $f,g$ è continua su $[a,b]$
    \item $f,g$ è derivabile su $]a,b[$
    \item $g'(x) \neq 0 \forall x\in ]a,b[$
\end{enumerate}
Allora:
\begin{equation*}
    \exists \,c \in ]a,b[ \;: \dfrac{f(b)-f(a)}{g(b)-g(a)} = \dfrac{f'(c)}{g'(c)}
\end{equation*}
}

\subsubsection{I teoremi di De L'Hôpital}
