\section{Derivate}

\dfn{
	Dato un intervallo $I \in \mathbb{R}$, $x_0 \in I$ si dice \textbf{punto interno} a $I$ se \textbf{esiste un intorno sferico:}
	\begin{equation*}
		B_r(x_0) = \{x \in \mathbb{R} \;| \; |x-x_0| < 0\}
	\end{equation*}
	tale che:
	\begin{equation*}
		B_r(x_0) \subseteq I
	\end{equation*}
}
L'insieme dei punti interni si indica con un piccolo \textit{tondino} in alto:
\begin{equation*}
	\mathring{I} := \{x \in I \; | \;\; \text{x è un punto interno a}\;I\} 
\end{equation*}
Questi punti sono diversi rispetto ai punti di accumulazione dei limiti per un paio di motivi. Il primo è che devono appartenere innanzitutto all'insieme. Se infatti si considera un insieme del tipo $\mathbb{R} \setminus \{0\}$, lo $0$ è un punto di accumulazione pur non appartenendo all'insieme. Non è però un punto interno. La seconda differenza principale con i punto di accumulazione è che, mentre questi ultimi possono essere "al bordo" di un insieme, i punti interni no, e devono avere l'intorno sferico completamente interno all'insieme. Cioè preso l'intervallo $[2, 4]$, mentre sia $2$ sia $4$ sono punti di accumulazione, nessuno dei due è interno all'insieme. In questo caso i punti interni a questo insieme sono $]2,4[$.

\subsection{Introduzione informale}
Pirma di dare la definizione formale di derivata vorrei spendere qualche riga per cercare di spiegare l'idea che si trova dietro a questo potentissimo oggetto matematico. Il problema è che spesso ci si perde nelle definzioni formali e negli esercizi, senza apprezzare veramente quello che si sta facendo. Perché alla fine si spera che ognuno, dopo aver seguito un corso di analisi, sappia che la derivata di $x^2$ è $2x$, oppure che la derivata del seno è il coseno, però veramente in pochi hanno capito cosa significa e perché è così.\\

\textbf{Derivata come tangente una curva:} Vi siete mai chiesti (probabilmente no ma la domanda retorica andava fatta) quale sia la retta tangete a una curva? Cioè, se io vi disegno una curva su un foglio e vi traccio un punto sopra e vi chiedo di disegnare la tangente a quella curva, probabilmente la richiesta non risulta così difficile. Magari la retta non verrà perfetta, però il disegno si avvicinerebbe parecchio ad una vera e propria tangente. Se però io invece di darvi un disegno di una curva vi do una funzione e un punto, trovare la tangente ora è molto più complicato perché dovreste prima disegnare la funzione e poi finalmente tracciare la tangente. Però poi nasce il problema di come si disegna una funzione in modo abbastanza preciso e quindi la situazione diventa piuttosto complicata. 

Facciamo un piccolo salto indietro e chiediamoci prima come si disegna una retta. Una retta, nella sua forma più classica, è data dalla seguente formula:
\begin{equation*}
	y = mx + q
\end{equation*}
Abbiamo quindi bisogno di due elementi per definirla: un coefficiente angolare $m$ e un termine noto $q$. Sappiamo però, dati due punti, trovare l'unica retta che passa per entrambi. Mettiamo di avere $A(x_a, y_a)$ e $B(x_b, y_b)$. Innanzitutto troviamo il coefficiente angolare della retta:
\begin{equation*}
	m = \dfrac{y_a - y_b}{x_a - x_b}
\end{equation*}
Fatto questo trovare $q$ non è molto difficile: basta inserire nella retta $m$ e uno qualsiasi dei due punti e risolvere l'equazione.

In realtà questo metodo di calcolare la retta presi due punti ci potrebbe venire molto comodo per calcolare la tangente ad una curva. Se infatti prendiamo due punti su di essa e poi li \textit{avviciniamo abbastanza} tra loro possiamo ottenere una buona approssimazione di una tangente (figura \ref{ApproxTangenteCurva}).



\begin{figure}[h]
\centering
\begin{tikzpicture}
\begin{axis}[xmax = 3, xmin = -0.5, ymax = 2, ymin = -0.5,
             axis lines=middle, xlabel=$x$, ylabel=$y$,
             xtick={10},
             ytick={10},
             xlabel style={anchor=north east},
             xticklabel style={anchor=north},
             yticklabel style={anchor=east},
             width=\linewidth,
            ]
\addplot[domain=-1:3, samples=200]{-0.2*x*x + 0.8 * x + 0.4};
\addplot[domain=-1:3, samples=200, dashed]{0.2 * x + 0.65};
\addplot[domain=-1:3, samples=200, dashed]{0.3 * x + 0.6};
\addplot[domain=-1:2.7, samples=200, dashed]{0.5 * x + 0.5};
\addplot[domain=-1:3, samples=200]{0.6 * x + 0.45};
\addplot [only marks, mark options={scale=1}] table {
0.5 0.75
1 1
2 1.2
2.5 1.15
};
\end{axis}
\end{tikzpicture}
  \caption{Approssimazione di una tangente con rette passanti per due punti} 
	\label{ApproxTangenteCurva}
\end{figure}

Il calcolo difficile non è tanto però il termine noto $q$, ma piuttosto il coefficiente angolare $m$. Questo perché dobbiamo prendere punti sempre più vicini e non è facile farlo venire preciso. Questo avvicinarci indefinitivamente ad un punto dovrebbe però far accendere una piccola lapadina in testa: abbiamo già uno strumento matematico che ci permette di avvicinarci indefinitivamente ad un punto: \textbf{il limite}. 

Prendiamo ora una funzione $f(x)$ per il momento generica\footnote{Per correttezza questa funzione non può essere generica perché non tutte le funzioni sono effettivamente derivabili, però rimaniamo così per il momento. In seguito viene definita rigorosamente.} e decidiamo di vole cacolare la tangente nel punto $x_0$. Come abbiamo appena visto ci servono due punti per calcolare la tangente: uno che resta fermo (in questo caso $x_0$) e uno che si avvicina a quest'ultimo. Per correttezza $x_0$ non è un punto, ma bensì $P(x_0, f(x_0))$. Il secondo punto che ci serve deve essere anchesso sulla funzione e quindi imponiamo questa condizione per la sua ordinata: $Q(x, f(x))$. Il coefficiente angolare tra i due risulta quindi:
\begin{equation*}
	m = \dfrac{Q_y - P_y}{Q_x - P_x} = \dfrac{f(x) - f(x_0)}{x-x_0}
\end{equation*}
Facendo in modo che il punto $Q$ si avvicini a $P$ indefinitamente:
\begin{equation*}
	\lim_{x \to x_0} \dfrac{f(x) - f(x_0)}{x-x_0}
\end{equation*}
E tadà, siamo arrivati alla definizione (informale) di derivata. La \textit{derivata della funzione $f$ nel punto $x_0$} è proprio il valore di quel limite, cioè il \textbf{coefficiente angolare della retta tangente alla funzione f nel punto $x_0$}. 

Se si vuole trovare più comodamente la retta tangente una funzione $f(x)$ in un punto $x_0$ basta usare la formula:
\begin{equation*}
	y = f'(x_0) (x - x_0) + f(x_0)
\end{equation*}
In generale è inutile comunque impararla perché si può sempre trovare la tangente in due passaggi: si calcola la derivata e quindi il coefficiente angolare e poi si impone il passaggio per il punto $P(x_0, f(x_0))$ e automaticamente si ha la stessa retta.\\

Facciamo un esempio: proviamo a trovare la retta tangente al grafico di $3x^3$ nel punto $x = 1$. Iniziamo calcolando la derivata nel punto:
\begin{equation*}
	\lim_{x \to 1} \dfrac{f(x) - f(1)}{x - 1} = \lim_{x \to 1} \dfrac{3x^3 - 3}{x - 1}
\end{equation*}
Raccogliamo il $3$ e usiamo qualche piccolo trucchetto per trovare il limite:
\begin{equation*}
	\lim_{x \to 1} 3 \cdot \dfrac{x^3 - 1}{x - 1} = \lim_{x \to 1} 3 \cdot \dfrac{(x - 1)(x^2 + x + 1)}{x - 1} = \lim_{x \to 1} 3(x^2 + x + 1) = 3 \cdot 3 = 9
\end{equation*}
Ne consegue che il coefficiente angolare della retta tangente nel punto $x = 1$ è $m = 9$. Imponiamo ora il passaggio per il punto $P(1, f(1)) = P(1, 3)$:
\begin{align*}
	y &= mx + q\\
	3 &= 9 \cdot 1 + q\\
	q &= 3 - 9 = -6
\end{align*}
Ne consegue che la nostra retta ha equazione:
\begin{equation*}
	y = 9x - 6
\end{equation*}

%RIFLETTERE SUL LIMITE E FARE UN ESEMPIO (?)

\textbf{Derivata come misura del cambiamento:} Un altro modo di vedere la derivata è come \textit{misura del cambiamento}. Successivamente introdurremo un teorema che esplicita meglio questo concetto, ma per ora limitiamoci ad introdurlo informalmente. Come abbiamo appena visto la derivata è \textit{il coefficiente angolare della retta tangente una curva in un punto}. Se osserviamo le figure \ref{Tangente12} e \ref{Tangente3} si può vedere come il coefficiente sia proporzionalmente più grande in base a quanto sta crescendo (o diminuendo per coefficienti negativi) la funzione. In pratica la derivata in questo caso sta risponendo alla domanda \textbf{quanto sta cambiando la funzione in quel punto?} 

\begin{figure}
\centering
\begin{subfigure}{0.49\textwidth}
\centering
	\begin{tikzpicture}
	\begin{axis}[xmax = 5, xmin = -2.5, ymax = 4, ymin = -0.5,
		     axis lines=middle, xlabel=$x$, ylabel=$y$,
		     xtick={10},
		     ytick={10},
		     xlabel style={anchor=north east},
		     xticklabel style={anchor=north},
		     yticklabel style={anchor=east},
		    ]
		\addplot[domain=-3:12, samples=200]{ln(x+5)};
		\addplot[domain=-3:3, samples=200, dashed]{0.1923* x + 1.6101};
		\addplot [only marks, mark options={scale=1}] table {
		0.2 1.648
		};
	\end{axis}
	\end{tikzpicture}
\end{subfigure}
\begin{subfigure}{0.49\textwidth}
\centering
	\begin{tikzpicture}
	\begin{axis}[xmax = 6, xmin = -0.5, ymax = 4, ymin = -0.5,
		     axis lines=middle, xlabel=$x$, ylabel=$y$,
		     xtick={10},
		     ytick={10},
		     xlabel style={anchor=north east},
		     xticklabel style={anchor=north},
		     yticklabel style={anchor=east},
		    ]
		\addplot[domain=-1:6, samples=200]{ln(x)+2};
		\addplot[domain=-1:5, samples=200, dashed]{2/3* x -0.594 + 2};
		\addplot [only marks, mark options={scale=1}] table {
		1.5 2.405
		};
	\end{axis}
	\end{tikzpicture}
\end{subfigure}
\caption{Tangenti una curva, coefficienti angolari rispettivamente (da sinistra a destra) $11^\circ$ e $33^\circ$} 
\label{Tangente12}
\end{figure}

\begin{figure}[h]
\end{figure}

\begin{figure}[h]
\centering
\begin{tikzpicture}
\begin{axis}[xmax = 4, xmin = -3, ymax = 10, ymin = -0.5,
             axis lines=middle, xlabel=$x$, ylabel=$y$,
             xtick={10},
             ytick={10},
             xlabel style={anchor=north east},
             xticklabel style={anchor=north},
             yticklabel style={anchor=east},
             width=0.6\linewidth,
            ]
	\addplot[domain=-3:12, samples=200]{x*x};
\addplot[domain=-2:4, samples=200, dashed]{4*x-4};
\addplot[domain=-3:2, samples=200, dashed]{-2*x-1};
\addplot [only marks, mark options={scale=1}] table {
2 4
-1 1
};
\end{axis}
\end{tikzpicture}
\caption{Tangente una curva, coefficiente angolare (da sinistra a destra) $-63^\circ$ e $75^\circ$} 
	\label{Tangente3}
\end{figure}

Per quello può essere vista come la misura del cambiamento, perché se derivi una funzione puoi scoprire quanto effettivamente quella stia cambiando in relazione al suo argomento.

\subsection{Introduzione formale}
\dfn{
	Data una funzione $f: I \to \mathbb{R}$ e un punto $x_0 \in \mathring{I}$, $f$ si dice \textbf{derivabile in $x_0$} se:

	\begin{equation*}
		\exists \lim _{x \to x_0} \dfrac{f(x) - f(x_0)}{x - x_0} \in \mathbb{R}
	\end{equation*}
	Ci sono due principali notazioni per indicare la derivata della funzione $f$ nel punto $x_0$:
	\begin{equation*}
		f'(x_0) \quad \text{oppure} \quad \dv{f}{x} \left(x_0\right)
	\end{equation*}
	In generale i matematici preferiscono la prima mentre i fisici la seconda.
}
Essendo che $c$ si sta avvicinando a $x_0$, è possibile riscrivere la derivata in maniera differente: In pratica $x$ lo scriviamo in funzione di $x_0$ e ci aggiungiamo una piccola variabile che diventa sempre più piccola. Il risultato è lo stesso ma a per certe dimostrazioni conviene scriverlo così:
\begin{equation*}
	\lim_{h \to 0} \dfrac{f(x + h) - f(x)}{h}
\end{equation*}

Infati $x_0 = x + h$ e il denominatore si semplifica in quanto $x - x_0 = x - x - h = -h$. Cambiando segno anche al numerato il risultato è lo stesso.

\dfn{
	Derivata destra e sinitra: $f: I \to \mathbb{R}$, $x_0 \in \mathring{I}$. 

	\begin{itemize}
		\item $f$ si dice \textbf{derivabile a sinistra in $x_0$}, e si indica $f'_-(x_0)$, se:
			\begin{equation*}
				\exists \lim _{x \to x_0^-} \dfrac{f(x) - f(x_0)}{x - x_0} \in \mathbb{R} 
			\end{equation*}

		\item $f$ si dice \textbf{derivabile a destra in $x_0$}, e si indica $f'_+(x_0)$, se:
			\begin{equation*}
				\exists \lim _{x \to x_0^+} \dfrac{f(x) - f(x_0)}{x - x_0} \in \mathbb{R} 
			\end{equation*}
	\end{itemize}
}

L'idea è la stessa dei limiti:
\begin{equation*}
	f \; \text{è derivabile in } x_0 \iff
	\begin{cases*}
		f \; \text{è derivabile sia a destra che a sinistra di } x_0\\
		f'_+(x_0) = f'_-(x_0) = f(x_0)
	\end{cases*}
\end{equation*}

\dfn{
	$f$ si dice \textbf{derivabile su} $I$ se $f$ è derivabile $\forall x \in I$
}
In questo ultimo caso, ad $f$ si può associare una nuova funzione: la sua derivata. In pratica la nuova funzione assume i valori della derivata di $f$ per ogni punto in cui è definita. Si indica generalmente con la notazione classica $f'(x)$.\\

\textbf{Alcuni esempi:} Di seguito riporto qualche esempio molto semplice di calcolo della derivata tramita la sua definzione. In realtà raramente si usa questa definzione nei calcoli effettivi, ma piuttosto si ricavano prima le derivate delle funzioni elementari e applicando qualche regola si riesce a derivare senza calcolare limiti. È bello però far vedere comunque la definzione e la sua potenza in quanto tutto quello che verrà dopo sarà comunque frutto di questo. %% NON mi piace la formulazione

Proviamo a derivare $f(x) = x$:
\begin{equation*}
	\lim_{h \to 0} \dfrac{f(x_0 + h) - f(x_0)}{h} = \lim_{h \to 0} \dfrac{x_0 + h -x_0}{h} = \lim_{h \to 0} 1 = 1
\end{equation*}

Quindi $f'(x) = 1$. È facile verificare che questo è vero perché la fuznione $x$ è semplcemente una retta, che ha lo stesso coefficiente angolare $\forall x \in \mathbb{R}$. Inoltre alla domanda "\textit{quanto sta cambiando questa funzione?}" è facile vedere che la funzione cambia sempre nello stesso modo, quindi il valore della sua derivata è costante.

La funzione costante $g(x) = k$ invece che derivata ci aspettiamo che abbia? In teoria, essendo costante, la funzione non cambia mai. Quindi possiamo aspettarci che la sua derivata sia nulla. Verifichiamolo:
\begin{equation*}
	\lim_{h \to 0} \dfrac{f(x_0 + h) - f(x_0)}{h} = \lim_{h \to 0} \dfrac{5 - 5}{h} = 0
\end{equation*}
In questo caso il limite non è una forma indeterminata perché il numeratore è \textbf{esattamente} $0$, mentre il denominatore si avvicinerà sempre di più a quel valore, ma non lo raggiungerà mai (come da definzione di limite).

Proviamo a derivare una funzione leggermente più complessa: $h(x) = x^2$ :
\begin{align*}
	&\lim_{h \to 0} \dfrac{f(x_0 + h) - f(x_0)}{h} = \lim_{h \to 0} \dfrac{(x+h)^2 - x^2}{h} =  \lim_{h \to 0} \dfrac{x^2 +2 x h + h^2 - x^2}{h} =\\
	&\lim_{h \to 0} \dfrac{2xh + h^2}{h} = \lim_{h \to 0} 2x + h =  2x
\end{align*}

La derivata in questo caso risulta essere $2x$. Ed effettivamente ha senso visto che il grafico di $x^2$ cambia sempre di più all'aumentare di $x$. Se infatti si guarda la classica forma di una parabola di può facilmente notare che più aumenta $x$, più il grafico \textit{acquista pendenza}. % FORSE VA RIFORMULATO :)



\subsection{Regole di derivazione} \label{RegoleDerivazione}
È oggettivamente infattibile derivare ogni singola funzione con la definzione di derivata. Di conseguenza tramite questa ultima si ricavano delle piccole regoline che permettono poi di semplificare enormemente il calcolo:
\imp{
\begin{enumerate}
	\item Costante per una funzione:
		\begin{equation*}
			(a \cdot f)' = a \cdot f'(x)
		\end{equation*}
	\item Somma e sottrazione di funzioni:
		\begin{equation*}
			(f \pm g)' = f' \pm g'
		\end{equation*}

	\item Prodotto di funzioni:
		\begin{equation*}
			(f \cdot g)' = f' \cdot g + f \cdot g'
		\end{equation*}

	\item Quoziente di funzioni:
		\begin{equation*}
			\left( \dfrac{f}{g} \right)' = \dfrac{f' \cdot g - f \cdot g'}{g^2}
		\end{equation*}

	\item Composizione di funzioni
		\begin{equation*}
			(f \circ g)' = (f' \circ g) \cdot g'
		\end{equation*}
		Il simbolo $\circ$ indica la composizione di funzioni, cioè $f \circ g = f(g(x))$.
\end{enumerate}
}
Nonostante il prof non abbia portato nessuna dimostrazione riguardo alle regole sopra citate, ritengo che per completezza queste dimostrazioni vadano comunque viste per togliere quel senso di mistero intrinseco che delle regole comparse dal nulla hanno.\\


\textbf{Dimostrazioni NON FATTE DAL PROF} (ma che io ritengo importanti):
\begin{enumerate}
	\item Somma e sottrazione di funzioni $j(x) = f(x) \pm g(x)$:
		\begin{align*}
			j'(x) &= \lim_{h \to 0} \dfrac{j(x + h) - j(x)}{h} = \lim_{h \to 0} \dfrac{f(x + h) \pm g(x + h) - f(x) \mp g(x)}{h} = \\
			& = \lim_{h \to 0} \dfrac{f(x + h) - f(x)}{h} \pm \dfrac{g(x + h) - g(x)}{h} = \lim_{h \to 0} \dfrac{f(x + h) - f(x)}{h} \pm \lim_{h \to 0} \dfrac{g(x + h) - g(x)}{h} = \\
			& = f'(x) \pm g'(x)
		\end{align*}

	\item Prodotto di funzioni $j(x) = f(x) \cdot g(x)$:
		\begin{align*}
			j'(x) &= \lim_{h \to 0} \dfrac{j(x + h) - j(x)}{h} = \lim_{h \to 0} \dfrac{f(x + h) \cdot g(x + h) - f(x) \cdot g(x)}{h} = \\
			&= \lim_{h \to 0} \dfrac{f(x + h) \cdot g(x + h) +f(x + h) \cdot g(x) - f(x + h) \cdot g(x) - f(x) \cdot g(x)}{h} = \\
			&= \lim_{h \to 0} \dfrac{f(x + h) \cdot (g(x + h) - g(x)) + g(x) \cdot (f(x + h) - f(x))}{h} = \\
			&= \lim_{h \to 0} f(x+h) \left[\dfrac{g(x+h) - g(x)}{h} \right] + \lim_{h \to 0} g(x) \left[\dfrac{f(x+h) - f(x)}{h} \right] = \\
			&= f(x) \cdot g'(x) + g(x) \cdot f'(x)
		\end{align*}


	\item Quoziente di funzioni $j(x) = \left( \dfrac{f}{g} \right)$:
		\begin{align*}
			j'(x) &= \lim_{h \to 0} \dfrac{j(x + h) - j(x)}{h} = \lim_{h \to 0} \dfrac{\dfrac{f(x + h)}{g(x + h)} - \dfrac{f(x)}{g(x)}}{h} =\\[10pt]
			&= \lim_{h \to 0} \dfrac{\dfrac{f(x + h) \cdot g(x) - f(x) \cdot g(x + h)}{g(x + h) \cdot g(x)}}{h} = \lim_{h \to 0} \dfrac{f(x + h) \cdot g(x) - f(x) \cdot g(x + h)}{g(x + h) \cdot g(x) \cdot h} =\\[10pt]
			&= \lim_{h \to 0} \dfrac{f(x + h) \cdot g(x) + f(x) \cdot g(x) - f(x) \cdot g(x) - f(x) \cdot g(x + h)}{g(x + h) \cdot g(x) \cdot h} =\\[10pt]
			&= \lim_{h \to 0} \dfrac{g(x) \cdot (f(x + h) - f(x)) -f(x) \cdot (g(x + h) - g(x))}{h} \cdot \dfrac{1}{g(x + h) \cdot g(x)} =\\[10pt]
			&= \lim_{h \to 0} \left [\dfrac{g(x) \cdot (f(x + h) - f(x))}{h} - \dfrac{f(x) \cdot (g(x + h) - g(x))}{h} \right] \cdot \dfrac{1}{g(x + h) \cdot g(x)} =\\[10pt]
			&= \left [ g(x) \cdot f'(x) - f(x) \cdot g'(x) \right] \cdot \dfrac{1}{g(x) \cdot g(x)} = \dfrac{f'(x) \cdot g(x) - f(x) \cdot g'(x)}{g^2(x)}
		\end{align*}

	\item Composizione di funzioni $j(x) = f(g(x))$:
		\begin{align*}
			j'(x) &= \lim_{x \to x_0} \dfrac{j(x) - j(x_0)}{x - x_0} = \lim_{h \to 0} \dfrac{f(g(x)) - f(g(x_0))}{x - x_0} =\\[10pt]
			&= \lim_{x \to x_0} \dfrac{f(g(x)) - f(g(x_0))}{g(x) - g(x_0)} \cdot \dfrac{g(x) - g(x_0)}{x - x_0}= f'(g(x)) \cdot g'(x) % DA VERIFICARE :)
		\end{align*}
\end{enumerate}

\subsection{Algebra delle derivate} 
Questa sottosezioine sembrerà un duplicato delle regole di derivazine (Sezione: \ref{RegoleDerivazione}). In parte lo è, ma in realtà vuole rendere più formali quelle regole che hanno qualche piccola premessa volutamente mancante ma che in teoria dovrebbe averle rese più chiari e semplici da leggere.\\

Date due funzioni $f, g: I \to \mathbb{R}$, e un punto $x_0 \in I$, se $f$ e $f$ sosno derivabili in $x_0$ allora:
\begin{enumerate}
	\item $f \pm g$ è derivabile in $x_0$ e: 
		\begin{equation*}
			(f \pm g)' (x_0) = f'(x_0) \pm g'(x_0)
		\end{equation*}

	\item $f \cdot g$ è derivabile in $x_0$ e: 
		\begin{equation*}
			(f \cdot g)' (x_0) = f'(x_0) \cdot g(x_0) + f(x_0) \cdot g'(x_0)
		\end{equation*}

	\item $\dfrac{f}{g}$ è derivabile in $x_0$ se $g(x_0) \neq 0$ e: 
		\begin{equation*}
			\left( \dfrac{f}{g} \right)' = \dfrac{f' \cdot g - f \cdot g'}{g^2}
		\end{equation*}
\end{enumerate}
\thm{
	Dati due intervalli $I, J \subseteq \mathbb{R}$, due funzioni $f:I \to \mathbb{R}$ e $g: J \to I$ e un punto $x_0 \in J$. Se $g$ è derivabile in $x_0$ e $f$ è derivabile in $g(x_0)$ allora $f \circ g$ è derivabile in $x_0$ e vale:

	\begin{equation*}
		(f \circ g)' (x_0) = f'(g(x_0)) \cdot g'(x_0)
	\end{equation*}
}

\subsection{Derivate funzioni elementari} 
\imp{
\begin{multicols}{2}
\begin{itemize}
	\item $(x^n)' = nx^{n-1}$ 

	\item $(e^x)' = e^x$
	\item $(a^x)' = a^x \cdot \ln(a)$ con $a > 0$

	\item $(\ln(x))' = \dfrac{1}{x}$
	\item $(\log_a (x))' = \dfrac{1}{x \cdot ln(a)}$

	\item $(\sin(x))' = \cos(x)$
	\item $(\cos(x))' = -\sin(x)$
	\item $(\tan(x))' = \dfrac{1}{\cos^2(x)} = 1 + \tan^2(x)$

	\item $(\arcsin(x))' = \dfrac{1}{\sqrt{1 - x^2}}$
	\item $(\arccos(x))' = -\dfrac{1}{\sqrt{1 - x^2}}$
	\item $(\arctan(x))' = \dfrac{1}{1 + x^2}$
\end{itemize}
\end{multicols}
}

Di seguito sono presenti le dimostrazioni di alcune di queste derivate:
\pf{
	Dobbiamo dimostrare:
	\begin{equation*}
		(x^n)' = n \cdot x^{n-1}
	\end{equation*}
	Prendimo una funzione funzione $f(x) = x^n$ con $n \in \mathbb{N}$ e $n > 2$. Ci riduciamo a dimostrare quindi:
	\begin{equation*}
		f'(x) = n \cdot x^{n-1}
	\end{equation*}
	Ora dalla definizione di derivata:
	\begin{equation*}
		f'(x) = \lim_{h \to 0} \dfrac{f(x + h) - f(x)}{h} = \lim_{h \to 0} \dfrac{(x + h)^n - x^n}{h}
	\end{equation*}
	Dal binomio di newton (Sezione \ref{BinomioNewton}):
	\begin{equation*}
		(x + h)^n = \sum \limits_{j = 0}^n \binom{n}{j} \cdot x^{n-j} \cdot h^j = x^n + \binom{n}{1} \cdot x^{n-1} \cdot h + \sum_{j = 2}^{n} \binom{n}{j} x^{n -j} \cdot h^j
	\end{equation*}
	Essendo $\binom{n}{1} = n$, possiamo sostituire nel limite:
	\begin{align*}
		&\lim_{h \to 0} \dfrac{x^n + n \cdot x^{n-1} \cdot h + \sum_{j = 2}^{n} \binom{n}{j} x^{n -j} \cdot h^j - x^n}{h} =\\[10pt]
		=& \lim_{h \to 0} \left( n \cdot x^{n-1} + \sum_{j = 2}^n \binom{n}{j} x^{n-j} \cdot h^{j-1} \right)
	\end{align*}
	L'esponente di $h$, essendo che la sommatoria parte da $j = 2$, sarà sempre maggiore o uguale a uno ($j-1 \geq 1$). Di conseguenza essendo $h \to 0$ per il limite, tutta la sommatoria va a $0$:
	\begin{equation*}
		n \cdot x^{n-1}
	\end{equation*}

	\hfill Qed.
	
}
\pf{
	Dobbiamo provare che:
	\begin{equation*}
		(\sin (x))' = \cos(x)
	\end{equation*}
	Chiamiamo $f(x) = \sin (x)$ e applichiamo la definzione:
	\begin{equation*}
		f'(x) = \lim_{h \to 0} \dfrac{f(x + h) - f(x)}{h} = \lim_{h \to 0} \dfrac{\sin(x + h) - \sin(x)}{h}
	\end{equation*}
	Usando le formule di addizione (Sezione \ref{FormuleAddSott}):
	\begin{align*}
		&\lim_{h \to 0} \dfrac{\sin(x)\cos(h) + \cos(x)\sin(h) - \sin(x)}{h} = \lim_{h \to 0} \dfrac{\sin(x) \cdot (\cos(h) -1) + \cos(x)\sin(h)}{h} =\\[10pt]
		= &\lim_{h \to 0} \sin(x) \cdot \dfrac{\cos(h) - 1}{h} + \cos(x) \cdot \dfrac{\sin(h)}{h}
	\end{align*}
	Usando qualche limite notevole visto in precedenza (Sezione \ref{LimitiNotevoli}):
	\begin{equation*}
		\lim_{h \to 0} \sin(x) \cdot 0 + \cos(x) \cdot 1 = \cos(x)
	\end{equation*}
	\hfill Qed.
}

\pf{
	Dobbiamo provare che:
	\begin{equation*}
		(\cos (x))' = -\sin(x)
	\end{equation*}
	Chiamiamo $f(x) = \cos (x)$ e applichiamo la definzione:
	\begin{equation*}
		f'(x) = \lim_{h \to 0} \dfrac{f(x + h) - f(x)}{h} = \lim_{h \to 0} \dfrac{\cos(x + h) - \cos(x)}{h}
	\end{equation*}
	Usando le formule di addizione (Sezione \ref{FormuleAddSott}):
	\begin{align*}
		&\lim_{h \to 0} \dfrac{\cos(x)\cos(h) - \sin(x)\sin(h) - \cos(x)}{h} = \lim_{h \to 0} \dfrac{\cos(x) \cdot (\cos(h) -1) - \sin(x)\sin(h)}{h} =\\[10pt]
		= &\lim_{h \to 0} \cos(x) \cdot \dfrac{\cos(h) - 1}{h} - \cos(x) \cdot \dfrac{\sin(h)}{h}
	\end{align*}
	Usando qualche limite notevole visto in precedenza (Sezione \ref{LimitiNotevoli}):
	\begin{equation*}
		\lim_{h \to 0} \cos(x) \cdot 0 - \sin(x) \cdot 1 = -\sin(x)
	\end{equation*}
	\hfill Qed.

}
\pf{
	Dobbiamo provare che:
	\begin{equation*}
		(\tan(x))' = \dfrac{1}{\cos^2(x)} = 1 + \tan^2(x)
	\end{equation*}
	In questo caso essendo:
	\begin{equation*}
		\tan(x) = \dfrac{\sin(x)}{\cos(x)}
	\end{equation*}
	Ci basta calcolare la derivata di un quoziente:
	\begin{equation*}
		(\tan(x))' = \left( \dfrac{\sin(x)}{\cos(x)} \right)' = \dfrac{\cos(x)\cos(x) - \sin(x) \cdot -\sin(x)}{\cos^2(x)} =  \dfrac{\cos^2(x) + \sin^2(x)}{\cos^2(x)}
	\end{equation*}
	Che in base a come si vuole semplificare porta a due possibili risultati:
	\begin{enumerate}
		\item 
			\begin{equation*}
				\dfrac{\cos^2(x) + \sin^2(x)}{\cos^2(x)} = \dfrac{1}{\cos^2(x)}
			\end{equation*}

		\item 
			\begin{equation*}
				\dfrac{\cos^2(x) + \sin^2(x)}{\cos^2(x)} = \dfrac{\cos^2(x)}{\cos^2(x)} + \dfrac{\sin^2(x)}{\cos^2(x)} = 1 + \tan^2(x)
			\end{equation*}
	\end{enumerate}
	
}
% IN CODA VANNO LE DIMOSTRAZIONI 


\subsection{Derivare un valore assoluto} % DA FARE
Proviamo a derivare il valore assoluto di $x$:
\begin{equation*}
    f(x) = |x| =
    \begin{cases*}
        x \quad \;\;\, \text{se} \quad x \geq 0\\
        -x \quad \text{se} \quad x < 0
    \end{cases*}
\end{equation*}
È abbastanza facile vedere che è derivabile sicuramente per tutto l'intervallo $\mathbb{R} \setminus \{0\}$ perché se prese singolarmente, sia $x$ che $-x$ sono derivabili. Il problema è che non sappiamo se è derivabile in $x = 0$. Usiamo la definzione approcciando questa derivata prima da sinistra e poi da destra:
\begin{equation*}
	f_-' (0) \lim_{x \to 0^-} \dfrac{f(x) - f(0)}{x -0} = \lim_{x \to 0^-} \dfrac{|x|}{x}
\end{equation*}
Essendo che ci stiamo avvicinando a $0$ da sinistra ($x \to 0^-$), il valore assoluto è semplicemente $-x$:
\begin{equation*}
	\lim_{x \to 0^-} \dfrac {-x}{x} = -1
\end{equation*} 
Usando lo stesso procedimento per la derivata destra:
\begin{equation*}
	f_+' (0) \lim_{x \to 0^+} \dfrac{f(x) - f(0)}{x -0} = \lim_{x \to 0^+} \dfrac{|x|}{x} = \lim_{x \to 0^+} \dfrac {x}{x} = 1
\end{equation*}
Concludiamo quindi che $|x|$ \textbf{non è derivabile} in $x = 0$ perché il valore della sua derivata destra non coincide con il valore della derivata sinistra:
\begin{equation*}
	f_-' (0) = -1 \; \neq \; 1 = f_+'(0)
\end{equation*}

In generale però \textbf{il valore assoluto in una funzione non rende per forza questa non derivabile}. Un classico esempio è:
\begin{equation*}
	g(x) = x|x| = 
	\begin{cases*}
		x^2 \quad \;\;\, \text{se} \quad x \geq 0\\
		-x^2 \quad \text{se} \quad x < 0
	\end{cases*}
\end{equation*}
Se si considera sempre l'intervallo $\mathbb{R} \setminus \{0\}$ è facile notare che presi distintamente, i due tratti della funzione sono derivabili. Il problema resta sempre però il punto $x = 0$. Approcciamolo da sinistra:
\begin{equation*}
	\lim_{x \to 0^-} \dfrac{f(x) - f(0)}{x - 0} = \lim_{x \to 0^-} \dfrac{-x^2}{x} = \lim_{x \to 0^-} -x = 0
\end{equation*}
Approcciamolo ora da destra:
\begin{equation*}
	\lim_{x \to 0^+} \dfrac{f(x) - f(0)}{x - 0} = \lim_{x \to 0^+} \dfrac{x^2}{x} = \lim_{x \to 0^+} x = 0
\end{equation*}
Ne consegue che \textbf{la funzione è derivabile} proprio peché la sua derivata destra e sinistra conicidono.

\subsection{Derivate di ordine superiore}
%Derivate di ordine superiore

%Li faccio come subsubsection i teoremi? Oppure gli dedico delle subsection per ognuno?

\subsection{Teoremi}
\subsubsection{Teorema di Fermat}
Il seguente teorema enuncia che la derivata nei minimi e nei massimi si annulla.
\thm {
Data una funzione $f: [a,b] \to \mathbb{R}$, un punto di massimo o di minimo $x_0 \in ]a,b[$ e inoltre $f$ è derivabile in $x_0$, allora:
\begin{equation*}
    f'(x_0) = 0
\end{equation*}
}
È essenziale che il punto $x_0$ si all'interno dell'intervallo! %DARE VEDERE FOTO DI UNA RETTA CON DOMIONIO RISTRETTO
Inoltre l'annullarsi della derivata prima in un punto $x_0$ è condizione \textbf{necessaria} affinché $x_0$ sia un punto di massimo o di minimo relativo, ma \textbf{non è sufficiente} in generale (es. $x^3$ in $x=0$). %FAR VEDERE GRAFICO

\subsubsection{Teorema di Rolle}
\thm {
$f:[a,b] \to \mathbb{R}$
\begin{enumerate}
    \item $f$ è continua su $[a,b]$
    \item $f$ è derivabile su $]a,b[$
    \item $f(a) = f(b)$
\end{enumerate}
Allora:
\begin{equation*}
    \exists \,c \in ]a,b[ \;: f'(c) = 0
\end{equation*}
}

\subsubsection{Teorema di Lagrange}
\thm {
$f:[a,b] \to \mathbb{R}$
\begin{enumerate}
    \item $f$ è continua su $[a,b]$
    \item $f$ è derivabile su $]a,b[$
\end{enumerate}
Allora:
\begin{equation*}
    \exists \,c \in ]a,b[ \;: \dfrac{f(b)-f(a)}{b-a} = f'(c)
\end{equation*}
}
Corollario: se una funzione ha derivata sempre zero è costante.

\subsubsection{Teorema di Cauchy}
\thm {
$f,g:[a,b] \to \mathbb{R}$
\begin{enumerate}
    \item $f,g$ è continua su $[a,b]$
    \item $f,g$ è derivabile su $]a,b[$
    \item $g'(x) \neq 0 \forall x\in ]a,b[$
\end{enumerate}
Allora:
\begin{equation*}
    \exists \,c \in ]a,b[ \;: \dfrac{f(b)-f(a)}{g(b)-g(a)} = \dfrac{f'(c)}{g'(c)}
\end{equation*}
}

\subsubsection{I teoremi di De L'Hôpital}
