\section{Limiti} \label{limiti}
Prendiamo una successione abbastanza semplice da definire, tipo:
\begin{equation*}
    a_n = \frac{n-1}{n}
\end{equation*}
Cominciamo ora ad elencare i suoi termi e calcolariamo i suoi valori:
\begin{align*}
    a_1 &= \dfrac{0}{1} = 0\\
    a_2 &= \dfrac{1}{2} = 0.5\\
    a_3 &= \dfrac{2}{3} = 0.\overline{6}\\
    a_4 &= \dfrac{3}{4} = 0.75\\
    \vdots\\
    a_{1000} &= \dfrac{999}{1000} = 0.999\\
    \vdots\\
    a_{100000} &= \dfrac{99999}{100000} = 0.99999\\
\end{align*}
È facile notare che più $n$ diventa grande più il valore della successione \textit{tende} a $1$. Come si formalizza questa cosa? Con il concetto di limite.

\subsection{Limite di una successione} \label{lim_successioni}
\dfn{
    Data una successione $(a)_n$, e un numero $L \in \mathbb{R}$, si dice che $(a)_n$ è \textbf{convergente} e si indica:
    \begin{equation*}
        \lim_{x \to +\infty} a_n = L \qquad \text{oppure} \qquad (a)_n \xrightarrow[n \to +\infty]{} L
    \end{equation*}
    \qquad se
    \begin{equation*}
        \forall \epsilon > 0, \exists \bar{n} = \bar{n}(\epsilon) \in \mathbb{N}: \forall n \geq \bar{n}: |a_n - L| < \epsilon
    \end{equation*}
}

\dfn{
    Data una successione $(a)_n$, si dice che $(a)_n$ è \textbf{divergente} e si indica:
    \begin{itemize}
        \item \begin{equation*}
                \lim_{x \to +\infty} a_n = +\infty \qquad \text{oppure} \qquad (a)_n \xrightarrow[n \to +\infty]{} +\infty
            \end{equation*}
            \qquad se
            \begin{equation*}
                \forall \epsilon > 0, \exists \bar{n} = \bar{n}(\epsilon) \in \mathbb{N}: \forall n \geq \bar{n}: a_n \geq \epsilon
            \end{equation*}
        \item \begin{equation*}
                \lim_{x \to +\infty} a_n = -\infty \qquad \text{oppure} \qquad (a)_n \xrightarrow[n \to +\infty]{} -\infty
            \end{equation*}
            \qquad se
            \begin{equation*}
                \forall \epsilon > 0, \exists \bar{n} = \bar{n}(\epsilon) \in \mathbb{N}: \forall n \geq \bar{n}: a_n \leq -\epsilon
            \end{equation*}
    \end{itemize}
}

\imp{
\begin{center}
    Se esiste il limite di una successione esso è \textbf{unico}
\end{center}
}

Esistono successioni che non hanno limite (non sono ne convergenti ne divergenti).
\begin{gather*}
    a_n = (-1)^2\\
    a_0 = 1,\; a_1 = -1,\; a_2 = 1 \;\cdots
\end{gather*}
La successione è limitata in quanto i suoi valori sono $-1 e 1$, eppure non ha limite in quanto oscilla.
\begin{gather*}
    a_n = (-1)^n \cdot n\\
    a_0 = 0,\; a_1 = -1,\; a_2 = 2,\; a_3 = -3,\; a_4 = 4 \;\cdots
\end{gather*}
La successione in quanto caso non è limitata e non ha un limite in quanto "salta" tra valori positivi e valori negativi.

\subsection{Numero di Eulero (o Nepero)} \label{NumeroDiEulero}
%DA FARE MOLTO MEGLIO
Prendiamo in considerazione la serie
\begin{equation*}
    a_n = (1+\dfrac{1}{n})^n
\end{equation*}
Dimostriamo che è una serie \textbf{convergente}, e chiamiamo il suo limite \textbf{\textit{e}}. Quindi:
\begin{equation*}
    \lim_{x\to +\infty} (1+\dfrac{1}{n})^n = e \in \mathbb{R}
\end{equation*}

\pf{
Per dimostrare che una successione è convergente ci basta dimostrare che è (strettamente) crescente e limitata. Questo dal corollario della dimostrazione nella sezione \ref{corollario_successioni}.
\begin{enumerate}
    \item Dimostriamo che $(a_n)$ è (strettamente) crescente:
    
    \mlem{
        È necessario, per questa dimostrazione, introdurre la \textit{disuguaglianza di Bernulli}. Tuttavia non la dimostreremo.
        \begin{equation*}
            \forall x \in R : x \geq -1, \forall x \in \mathbb{N}: (1+x)^n \geq 1 + nx
        \end{equation*}
    }
    Per dimostrare che $(a_n)_n \nearrow$ ci basta dimostrare che:
    \begin{equation*}
        \dfrac{a_{n+1}}{a_n} > 1
    \end{equation*}
    \begin{align*}
        \dfrac{a_{n+1}}{a_n} &= \dfrac{\left(1+\dfrac{1}{n+1}\right)^{n+1}}{\left(1+\dfrac{1}{n}\right)^n} = \dfrac{\left(\dfrac{n+2}{n+1}\right)^{n+1}}{\left(\dfrac{n+1}{n}\right)^n}
        = \dfrac{\dfrac{n+2}{n+1} \cdot \left( \dfrac{n+2}{n+1} \right)^n}{\left(\dfrac{n+1}{n}\right)^n} =\\[5pt]
        &= \dfrac{n+2}{n+1} \cdot \left( \dfrac{n+2}{n+1} \cdot \dfrac{n}{n+1} \right)^n = \dfrac{n+2}{n+1} \cdot \left( \dfrac{n^2+2n+1-1}{n^2+2n+1} \right)^n =\\[5pt]
        &= \dfrac{n+2}{n+1} \cdot \left( 1 -\dfrac{1}{n^2+2n+1} \right)^n = \dfrac{n+2}{n+1} \cdot \left( 1 + \left( -\dfrac{1}{n^2+2n+1}\right) \right)^n
    \end{align*}
    Usiamo la disuguaglianza di Bernulli. Però assicuriamoci prima di poterla usare:
    \begin{gather*}
        x = - \dfrac{1}{n^2+2n+1} \geq -1\\
        \dfrac{1}{n^2+2n+1} \leq 1\\
        \leq n^2+2n+1\\
        n^2+2n >= 0
    \end{gather*}
    Verificato! Ora usiamola:
    \begin{align*}
        &\dfrac{n+2}{n+1} \cdot \left( 1 + \left( -\dfrac{1}{n^2+2n+1}\right) \right)^n \geq \dfrac{n+2}{n+1} \cdot \left( 1 + n\left( -\dfrac{1}{n^2+2n+1}\right) \right)=\\[5pt]
        &= \dfrac{n+2}{n+1} \cdot \dfrac{n^2+2n+1-n}{n^2+2n+1} = \dfrac{n+2}{n+1} \cdot \dfrac{n^2+n+1}{n^2+2n+1} =\\[5pt]
        & = \dfrac{n^3+3n^2+3n+1+1}{n^3+3n^2+3n+1} = 1 + \dfrac{1}{n^3+3n^2+3n+1} > 1
    \end{align*}
    
    \item Dobbiamo provare ora che $(a_n)_n$ è limitata. Usiamo il binomio di newton:
    \begin{align*}
        (1+\dfrac{1}{n})^n &= \sum_{k = 0}^{n} \binom{n}{k} \cdot 1^{n-k} \cdot \left(\dfrac{1}{n}\right)^k = \sum_{k = 0}^{n} \binom{n}{k}\dfrac{1}{n^k} = \sum_{k = 0}^{n} \dfrac{n!}{(n-k)!k!}\cdot\dfrac{1}{k!}=\\
        &= \sum_{k = 0}^{n} \dfrac{n \cdot (n-1) \cdots (n-k+1)}{n^k} \cdot \dfrac{1}{k!} =\\
        &= \sum_{k = 0}^{n} \dfrac{n}{n} \cdot \dfrac{(n-1)}{n} \cdots \dfrac{(n-k+1)}{n} \cdot \dfrac{1}{k!} =
    \end{align*}
        È facile notare che:
    \begin{equation*}
        \dfrac{n}{n} \cdot \dfrac{(n-1)}{n} \cdots \dfrac{(n-k+1)}{n} \leq 1
    \end{equation*}
    In quanto il denominatore sarà sempre più grande del numeratore (tranne per il primo termine) e quindi ogni singolo termine sarà $\leq 1$ ed essendo tutti moltiplicati tra di loro il risultato sarà anch'esso $\leq 1$. Possiamo quindi concludere che:
    \begin{equation*}
        = \sum_{k = 0}^{n} \dfrac{n}{n} \cdot \dfrac{(n-1)}{n} \cdots \dfrac{(n-k+1)}{n} \cdot \dfrac{1}{k!} \leq \sum_{k = 0}^{n} \dfrac{1}{k!} = 2 + \sum_{k = 2}^{n} \dfrac{1}{k!}
    \end{equation*}
    Possiamo notare che:
    \begin{equation*}
        k! = k\cdot (k-1)\cdot (k-2) \cdots 1 \geq k (k-1) > 0 \qquad \text{per} k\geq 2
    \end{equation*}
    \begin{equation*}
        \implies k! > k(k-1) \implies \dfrac{1}{k!} < \dfrac{1}{k(k-1)} = \dfrac{1}{k-1} - \dfrac{1}{k}
    \end{equation*}
    Quindi sostituendo
    \begin{align*}
        &< 2 + \sum_{k = 0}^{n} \left(\dfrac{1}{k-1} - \dfrac{1}{k}\right) = 2 + \left[ \left(1-\dfrac{1}{2}\right) + \left(\dfrac{1}{2}-\dfrac{1}{3}\right) + \left(\dfrac{1}{3}-\dfrac{1}{4}\right) \cdots \left(\dfrac{1}{n-1}-\dfrac{1}{n}\right)\right]=
    \end{align*}
    Le coppie di termini dentro la parentesi quadre condividono il primo elemento con il secondo della coppia precedente. Se si espandono le somme si cancellano tutti tranne il primo e l'ultimo:
    \begin{equation*}
        = 2 + \left[1 - \dfrac{1}{n}\right] = 3 -\dfrac{1}{n} < 3
    \end{equation*}
    Quindi:
    \begin{equation*}
        a_n = \left(1+\dfrac{1}{n}\right)^{n} < 3 \qquad \forall n \in \mathbb{N}
    \end{equation*}
    Dunque $(a_n)_n \nearrow e superiormente limitata$. Ne consegue che, dal corollario presente nella sezione \ref{corollario_successioni}:
    \begin{equation*}
        \exists \lim_{x\to + \infty} \left(1+\dfrac{1}{n}\right)^n = e
    \end{equation*}
\end{enumerate}
}
Si può dimostrare che $e \notin \mathbb{Q}$, quindi è un numero irrazionale e il suo valore è approssimativamente:
\begin{equation*}
    e \approx 2,71828 18284 59045 23536 \cdots
\end{equation*}

\subsection{Limite di una funzione}

\dfn{
\textbf{Intorno sferico di un punto} $x_0 \in \mathbb{R}$ di raggio $r$:
    \begin{equation*}
        x_0 \in \mathbb{R}, r\in \mathbb{R} : r > 0
    \end{equation*}
    \begin{equation*}
        I_r (x_o) = \left\{x\in \mathbb{R} : \; |x-x_0| < r \right\}
    \end{equation*}
    Che in pratica risulta:
    \begin{equation*}
        I_r (x_0) = ]x_0-r, x_0+r[
    \end{equation*}
}

\dfn{
$x_0$ si dice \textbf{punto di accumulazione} di $\mathbb{A} \subseteq \mathbb{R}$ se:
\begin{equation*}
    \forall r > 0: A\cap \left( I_r (x_0) \setminus \{x_0\} \right) \neq \emptyset
\end{equation*}
}
Comunque prendo piccolo $r$ c'è sempre un elemento di $\mathbb{A}$ diverso da $x_0$. L'insieme dei punti di accumulazione di un insieme è indicato come segue (la D dovrebbe essere celtica??):
\begin{equation*}
    \mathcal{D} (\mathbb{A}) = \left\{ x \in \mathbb{R}\; |\; x\; \text{è un punto di accumulazione di}\; \mathbb{A} \right\}
\end{equation*}
Posso in pratica avvicinarmi indefinitivamente ad $x_0$ sempre rimanendo in $\mathbb{A}$

\textbf{Proposizione:}\\
$\mathbb{A} \subseteq \mathbb{R}, x_o \in \mathbb{R}$. $x_0$ è un punto di accumulazione per A se e solo se $\exists (a_n)_n \subseteq \mathbb{A}$ t.c.
\begin{enumerate}
    \item $a_n \neq x_0 \forall n$
    \item $a_n \xrightarrow{n\to +\infty} x_0$
\end{enumerate}
Esempio:
$\mathbb{A} = \{\dfrac{1}{n} | n \in \mathbb{N}*\}$
\begin{equation*}
    \lim_{x\to + \infty} \dfrac{1}{x} = 0 \implies \mathcal{D}(\mathbb{A}) = {0}
\end{equation*}

Un insieme con $\mathcal{D} = \emptyset $ è un \textbf{insieme discreto}

\subsubsection{SCHEMA RIASSUNTIVO LIMITI:} %Mettere a posto la formattazione della prima parte
$\lim$
\begin{enumerate}[label=(\roman*)]
    \item $x\to x_0$
    \item $x\to x_0^+$
    \item $x\to x_0^-$
    \item $x\to +\infty$
    \item $x\to -\infty$
\end{enumerate}
$f(x)$
\begin{enumerate}
    \item $l \in \mathbb{R}$
    \item $+ \infty$
    \item $- \infty$
\end{enumerate}

\textbf{Tradotto}:
\begin{equation*}
    \forall \epsilon >0, \exists \delta > 0: \forall x \in \mathcal{D}(f) :
    \begin{cases*}
        0 < |x-x_0| < \delta \qquad \text{(i)}\\
        x_0 < x < x_0 + \delta \qquad \text{(ii)}\\
        x_0 - \delta < x < x_0 \qquad \text{(iii)}\\
        x > \delta \qquad \qquad\qquad\;\; \text{(iv)}\\
        x < - \delta \qquad \qquad\qquad \text{(v)}
    \end{cases*}
    \qquad
    \implies 
    \begin{cases*}
        |f(x) - l| < \epsilon \qquad 1\\
        f(x) > \epsilon \qquad\qquad\, 2\\
        f(x) < \epsilon \qquad\qquad\, 3
    \end{cases*}
\end{equation*}

\subsection{Algebra dei limiti}
L'algebra dei limiti vale sie per le successioni che per le funzioni. Per comodità in seguito è riportata con il caso delle funzioni:\\
Siano $f(x)$ e $g(x)$ due funzioni tale che
\begin{equation*}
    f(x) \xrightarrow{\qquad} l_1 \qquad g(x) \xrightarrow{\qquad} l_2
\end{equation*}
Alora:
\begin{equation*}
    f(x)+g(x) \xrightarrow{\qquad}
    \begin{cases*}
        l_1 + l_2 \qquad \text{se}\;\;\, l_1,l_2 \in \mathbb{R}\\
        +\infty \quad\, \qquad \text{se}\;\;\, l_1 = +\infty \land (l_2 \in \mathbb{R} \lor l_2 = +\infty)\\
        -\infty \quad\, \qquad \text{se}\;\;\, l_1 = -\infty \land (l_2 \in \mathbb{R} \lor l_2 = -\infty)\\
        \textit{Stessa cosa se si scambia $l_1$ con $l_2$}
    \end{cases*}
\end{equation*}

\begin{equation*}
    f(x) \cdot g(x) \xrightarrow{\qquad}
    \begin{cases*}
        l_1 \cdot l_2 \qquad \text{se}\;\;\, l_1,l_2 \in \mathbb{R}\\
        +\infty \quad\, \qquad \text{se}\;\;\, l_1 = +\infty \land (l_2 \in \mathbb{R}_+^* \lor l_2 = +\infty)\\
        +\infty \quad\, \qquad \text{se}\;\;\, l_1 = -\infty \land (l_2 \in \mathbb{R}_-^* \lor l_2 = -\infty)\\
        -\infty \quad\, \qquad \text{se}\;\;\, l_1 = +\infty \land (l_2 \in \mathbb{R}_-^* \lor l_2 =-\infty)\\
        -\infty \quad\, \qquad \text{se}\;\;\, l_1 = -\infty \land (l_2 \in \mathbb{R}_+^* \lor l_2 = +\infty)\\
        \textit{Stessa cosa se si scambia $l_1$ con $l_2$}
    \end{cases*}
\end{equation*}
\textit{con $g(x) \neq 0$}
\begin{equation*}
    \dfrac{f(x)}{g(x)} \xrightarrow{\qquad}
    \begin{cases*}
        \dfrac{l_1}{l_2} \qquad \text{se}\;\;\, l_1,l_2 \in \mathbb{R}\\
        0 \quad\, \qquad \text{se}\;\;\, l_1 \in \mathbb{R} \land l_2 = \pm \infty\\
        \pm \infty \quad\, \qquad \text{se}\;\;\, l_1 = \pm \infty \land l_2 \in \mathbb{R}_+^*\\
        \mp \infty \quad\, \qquad \text{se}\;\;\, l_1 = \pm \infty \land l_2 \in \mathbb{R}_-^*\\
        \textit{Stessa cosa se si scambia $l_1$ con $l_2$}
    \end{cases*}
\end{equation*}

\subsubsection{Forme indeterminate}
Non essendo presenti tutti i casi nell'algebra dei limiti nascono quelle che vengono chiamate \textbf{forme indeterminate} in quanto corrispondo a situazione non univoche dove il risultato non si può stabilire a priori, ma è necessario trattare ogni caso singolarmente. Di seguito una lista di queste forme:
\begin{itemize}
    \item $+\infty - \infty$
    \item $-\infty + \infty$
    \item $0 \cdot \pm \infty $
    \item $\dfrac{\pm \infty}{\pm \infty}$
    \item $\dfrac{0}{0}$
\end{itemize}

\textbf{NOTA}: Le espressioni che contengono i simboli $+\infty$ e $-\infty$ sono solo \textbf{espressioni formali}, non hanno alcun valore matematico!

\subsection{Teoremi dei limiti}
\thm{ \label{TeoremaPermanenzaSegno}
Teorema di \textbf{permanenza del segno}. Data $f: A \to \mathbb{R}$, $x_0 \in \mathcal{D}(A)$ e
\begin{equation*}
    \lim_{x \to x_0} f(x) = l \in \mathbb{R} \;\; \land \;\; l > 0
\end{equation*}
Allora:
\begin{equation*}
    \exists \delta > 0: \forall x \in A, x_0 - \delta < x < x_0 + \delta, x \neq x_0 \implies f(x) > 0
\end{equation*}
Vale anche nel caso di $l < 0$.
}

\thm{\label{TeoremaConfronto}
Teorema del \textbf{confronto}. $f, g, h: A \to \mathbb{R}$, $x_0 \in \mathcal{D}(A)$ e 
\begin{equation*}
    \lim_{x \to x_0} g(x) = \lim_{x \to x_0} h(x) = l \in \mathbb{R}
\end{equation*}
\begin{equation*}
    \exists \epsilon > 0: g(x) \leq f(x) \leq h(x), \forall x \in \left[ A\cup I_r(x_0) \right] \setminus {x_0} \implies \lim_{x \to x_0} f(x) = l
\end{equation*}
}

Il limite per $x->x_{0}$ esiste soltanto se sia il limite dx e il limite sx esistono e sono uguali.\\ %DA FARE MEGLIO
Gerarchia degli infiniti e limite di un polinomio + dimostrazione. (27-10-2022) %FARE MOLTO MEGLIO

%Aggiungere teorema utile per i calcoli? p.7 Analisi_pt2 (Miei appunti)

\subsection{Limiti Notevoli} \label{LimitiNotevoli}
%Da aggiungere. Dovrebbero essere nella lezione del 20 Ottobre i primi. Ci sono anche delle dimostrazioni
I limiti notevoli sono particolari limiti che nonostante siano una forma indeterminata, il loro valore è conosciuto. In realtà molti limiti con forme indeterminate hanno un valore conosciuto perché attraverso tecniche e metodi di calcolo si riesce a ricavare. La differenza con i limiti notevoli però è che questi ultimi sono estremamente fondamentali per molte dimostrazioni e per molti esercizi. Inoltre comprendono solo funzioni elementari e vengono dimostrati una volta per tutte e poi vengono dati per buoni nel calcolo dei limiti.\\
Il limite notevole più famoso e sicuramente più importante è:
\imp{
\begin{equation*}
    \lim _{x\to 0} \dfrac{\sin{x}}{x} = 1
\end{equation*}
}
Per dimostrarlo è prima necessario dimostrare 2 lemmi:
\mlem{
Dobbiamo dimostrare:
\begin{equation*}
    \lim_{x \to 0} \sin{x} = 0
\end{equation*}
Costruiamo una circonferenza goniometrica come in figura:
\begin{center}
\includegraphics[width=300px]{../img/CfrSinDim.png}    %% DA FARE MOOOLTO MEGLIO
\end{center}
Per la definizione di seno e coseno: $\sin{x} = \overline{HP}$ e $\cos{x} = \overline{OH}$. Inoltre $\overline{PP'} < \arc{PP'}$ perché il "percorso" più breve tra due punti in geometria euclidea è il segmento che li congiunge. Facendo qualche semplificazione:
\begin{align*}
    \overline{PP'} &< \arc{PP'}\\
    2\overline{HP} &< 2\arc{PA}\\
    \overline{HP} &< \arc{PA}
\end{align*}
Essendo segmenti, ed essendo quindi sempre positivi, riscriviamo la nostra disuguaglianza con il valore assoluto.
\begin{equation*}
    \left | \overline{HP} \right | < \left | \arc{PA} \right |
\end{equation*}
Si noti che per la precedente definizione di seno, per la definizione di angolo in radianti e per il fatto che siamo sulla circonferenza goniometrica che ha raggio pari a 1:
\begin{equation*}
    \left | \sin{x} \right | < \left | x \right |
\end{equation*}
Ed essendo il valore assoluto sempre maggiore o uguale a zero:
\begin{equation*}
    0 \leq \left | \sin{x} \right | < \left | x \right |
\end{equation*}
Dal teorema del confronto (Sezione: \ref{TeoremaConfronto}) si ha che:
\begin{equation*}
    \lim _{x \to 0} \left | \sin{x} \right | = 0
\end{equation*}
e quindi:
\begin{equation*}
    \lim _{x \to 0} \sin{x} = 0
\end{equation*}
\hfill Qed.
}
\mlem{
Dobbiamo dimostrare:
\begin{equation*}
    \lim_{x \to 0} \cos{x} = 1
\end{equation*}
Facciamo qualche trasformazione algebrica e usiamo la duplicazione del coseno (Sezione: \ref{formuleDuplicazione}):
\begin{equation*}
    \cos{x} = \cos \left ( 2 \cdot \dfrac{x}{2} \right ) = 1 - 2\sin^2 \left(\dfrac{x}{2} \right)
\end{equation*}
Ne consegue che:
\begin{equation*}
    1 - \cos{x} = 2\sin^2 \left(\dfrac{x}{2} \right)
\end{equation*}
Dal lemma precedente abbiamo:
\begin{align*}
    0 \leq \left | \sin{x} \right | &< \left | x \right |\\
    0 \leq \left | \sin \left( \dfrac{x}{2}\right) \right | &< \left | \dfrac{x}{2} \right |\\
    0 \leq \sin^2 \left( \dfrac{x}{2}\right) &< \dfrac{x^2}{4}\\
    0 \leq 2\sin^2 \left( \dfrac{x}{2}\right) &< \dfrac{x^2}{2}
\end{align*}
E quindi facendo una sostituzione:
\begin{equation*}
    0 \leq 1 - \cos{x} = 2\sin^2 \left( \dfrac{x}{2} \right) < \dfrac{x^2}{2}
\end{equation*}
Sempre per il teorema del confronto (in quanto $x \to 0$ implica $\frac{x^2}{2} \to 0$):
\begin{equation*}
    \lim_{x \to 0} 2\sin^2 \left( \dfrac{x}{2} \right) = 0
\end{equation*}
Sostituendo:
\begin{equation*}
    \lim_{x \to 0} 1 - \cos{x} = 0
\end{equation*}
E quindi:
\begin{equation*}
    \lim_{x \to 0} \cos{x} = 1
\end{equation*}
\hfill Qed.
}
Finito di dimostrare i lemmi passiamo ora a dimostrare il limite notevole vero e proprio:
\pf{
Dobbiamo dimostrare:
\begin{equation*}
    \lim_{x \to 0} \dfrac{\sin{x}}{x} = 1
\end{equation*}
Disegniamo un arco di circonferenza goniometrica (quindi con centro nell'origine e raggio 1). Il segmento $\overline{AH}$ è perpendicolare a $\overline{OB}$. Il segmento $\overline{AD}$ è perpendicolare a $\overline{AC}$.

\begin{center}
    \includegraphics[width=300px]{../img/CfrSinDim-2.png}
\end{center}

Assumiamo per iniziare $0 < x < \dfrac{\pi}{2}$. Per definizione di seno, coseno e tangente abbiamo che:
\begin{itemize}
    \item $A(\cos{x}, \sin{x})$
    \item $B(1, 0)$
    \item $C(1, \tan{x})$
\end{itemize}

Notiamo che $\overline{AH} \leq \arc{AB}$ in quanto $\overline{AB} \leq \arc{AB}$, per il fatto che la distanza tra due punti in geometria euclidea è il segmento che li congiunge, e $\overline{AH} \leq \overline{AB}$ in quanto $ABH$ è un triangolo rettangolo dove $\overline{AB}$ è l'ipotenusa.\\

Dobbiamo inoltre notare che $\arc{AB} \leq \overline{BC}$, in quanto $\arc{AB} < \overline{BD} + \overline{AD}$ dalla geometria euclidea, inoltre $\overline{BD} + \overline{AD} < \overline{BD} + \overline{DC}$ in quanto $\overline{AD}$ è un cateto del triangolo $ACD$ dove $\overline{CD}$ è l'ipotenusa. Basta infine notare che $\overline{BD} + \overline{DC} = \overline{BC}$ e quindi:
\begin{equation*}
    \overline{AH} \leq \arc{AB} \leq \overline{BC}
\end{equation*}
Essendo
\begin{equation*}
    \sin{x} = \overline{AH} \qquad x = \arc{AB} \qquad \tan{x} = \overline{BC}
\end{equation*}
sostituendo diventa:
\begin{equation*}
    \sin{x} \leq x \leq \tan{x}
\end{equation*}
Visto che all'inizio abbiamo posto la condizione $0 < x < \dfrac{\pi}{2}$, per forza $\sin{x} > 0$. Possiamo quindi dividere tutto per $\sin{x}$:
\begin{equation*}
    \dfrac{\sin{x}}{\sin{x}} \leq \dfrac{x}{\sin{x}} \leq \dfrac{\tan{x}}{\sin{x}} = \dfrac{\sin{x}}{\cos{x}} \cdot \dfrac{1}{\sin{x}} = \dfrac{1}{\cos{x}}
\end{equation*}
Sia $\frac{x}{\sin{x}}$, sia $\frac{1}{\cos{x}}$ sono maggiori di zero per $0 < x < \frac{\pi}{2}$. Possiamo quindi passare ai reciproci:
\begin{equation*}
    1 \geq \dfrac{\sin{x}}{x} \geq \cos{x} \qquad \left( 0 < x < \dfrac{\pi}{2} \right)
\end{equation*}
In realtà, essendo $\sin{x}$ una funzione dispari e $\cos{x}$ una funzione pari:
\begin{equation*}
    \dfrac{\sin(-x)}{-x} = \dfrac{-\sin{x}}{-x} = \dfrac{\sin{x}}{x}
\end{equation*}
\begin{equation*}
    \cos(-x) = \cos x
\end{equation*}
Questo vuol dire che la relazione scritta sopra vale anche per l'intervallo negativo $-\dfrac{\pi}{2} < x < 0$. Quindi:
\begin{equation*}
    1 \geq \dfrac{\sin{x}}{x} \geq \cos{x} \qquad \left( 0 < |x| < \dfrac{\pi}{2} \right)
\end{equation*}
Per il lemma dimostrato precedentemente che prova che $\lim_{x \to 0} \cos{x} = 1$ e il teorema del confronto (Sezione: \ref{TeoremaConfronto}):
\begin{equation*}
    \lim_{x \to 0} = \dfrac{\sin{x}}{x} = 1
\end{equation*}
\hfill Qed.
}
% IN RELAZIONE AL TEOREMA CI SAREBBE LA PROVA DELL'AREA DELLA CIRCONFERENZA. La metto? Se si dove? Nel caso sarebbe in fondo al file Analisi-1 scritto da me.

Un secondo limite notevole:
\imp{
\begin{equation*}
    \lim_{x \to 0} \dfrac{1 - \cos{x}}{x^2} = \dfrac{1}{2}
\end{equation*}
}
\pf{
\begin{align*}
    &\lim_{x \to 0} \dfrac{1 - \cos{x}}{x^2} = \lim_{x \to 0} \dfrac{1-\cos^2{x}}{x^2} \cdot \dfrac{1}{1 + \cos{x}} =\\
    =&\lim_{x \to 0} \dfrac{\sin^2{x}}{x^2} \cdot \dfrac{1}{1 + \cos{x}} = \lim_{x \to 0} \left( \dfrac{\sin{x}}{x} \right)^2 \cdot \dfrac{1}{1 + \cos{x}} = \\
    =&\lim_{x \to 0} \left( \dfrac{\sin{x}}{x} \right)^2 \cdot \lim_{x \to 0} \dfrac{1}{1 + \cos{x}} = 1 + \dfrac{1}{2} = \dfrac{1}{2}
\end{align*}
}

Un terzo limite notevole che deriva dal secondo:
\imp{
\begin{equation*}
    \lim_{x \to 0} \dfrac{1 - \cos{x}}{x} = 0
\end{equation*}
}

\pf{
\begin{equation*}
    \lim_{x \to 0} \dfrac{1 - \cos{x}}{x} = \lim_{x \to 0} \dfrac{1 - \cos{x}}{x^2} \cdot x = \dfrac{1}{2} \cdot 0 = 0
\end{equation*}
}
Altri limiti notevoli:
\imp{
\begin{equation*}
    \lim_{x \to 0} \dfrac{a^x-1}{x} = \ln{a} \qquad (0 < a,\;\; a \neq 1)
\end{equation*}
}

Il caso particolare in qui $a = e$:
\imp{
\begin{equation*}
    \lim_{x \to 0} \dfrac{e^x-1}{x} = 1
\end{equation*}
}

\subsection{Asintoti}
\dfn{
$x = k$ è un \textbf{asintoto verticale} per $f(x)$ se
\begin{equation*}
    \lim_{x\to k^{+}} f(x) = \pm \infty \qquad \text{oppure} \qquad \lim_{x\to k^{-}} f(x) = \pm \infty
\end{equation*}
}

\dfn{
$y = l$ si dice \textbf{asintoto orizzontale} se: $f: A\to \mathbb{R}$, $\sup A = +\infty$ e 
\begin{equation*}
    \lim_{x\to + \infty} f(x) = l \in \mathbb{R}
\end{equation*}
Vale ovviamente anche il caso a $-\infty$. Nella stessa funzione ci possono essere al massimo 2 asintoti orizzontali.
}

\subsection{Limiti di funzioni elementari}
IL limite di un polinomio $p(x)$ è il polinomio calcolato nel punto:
\begin{equation*}
    \lim_{x \to x_0} p(x) = p(x_0)
\end{equation*}
\pf{
Dimostriamo che il limite di un polinomio $p(x)$ è il polinomio calcolato nel punto:
\begin{equation*}
    \lim_{x \to x_0} p(x) = p(x_0)
\end{equation*}
dove $x_0 \in \mathbb{R}$. Partiamo dal caso base:
\begin{equation*}
    \lim_{x \to x_0} x = x_0
\end{equation*}
Questo deriva direttamente dalla definizione di limite con $\delta = \epsilon$. Ora dall'algebra dei limiti:
\begin{align*}
    &\lim_{x \to x_0} x^2 = \lim_{x \to x_0} x \cdot \lim_{x \to x_0} x = x_0^2\\
    &\lim_{x \to x_0} x^3 = \lim_{x \to x_0} x^2 \cdot \lim_{x \to x_0} x = x_0^3\\
    & \qquad \qquad \vdots \\
    &\lim_{x \to x_0} x^j = x_0^j \qquad (j \in \mathbb{R} )\\
\end{align*}
Se aggiungiamo un coefficiente ($a \in \mathbb{R}$) davanti al polinomio non cambia nulla, infatti:
\begin{equation*}
    \lim_{x \to x_0} ax^j = \lim_{x \to x_0} a \cdot \lim_{x \to x_0} x_0^j = a x_0^j
\end{equation*}
Possiamo quindi provare una importante proprietà dei polinomi. Dato infatti un polinomio generico di grado $n$:
\begin{equation*}
    p(x) = \sum _{i = 0}^n a_i \cdot x^i
\end{equation*}
Vale che quello che vogliamo dimostrare, infatti:
\begin{equation*}
    \lim_{x \to x_0} \sum _{i = 0}^n a_i \cdot x^i = \sum _{i = 0}^n \lim_{x \to x_0} a_i \cdot x^i = \sum _{i = 0}^n a_i \cdot x_0^i = p(x_0)
\end{equation*}
\hfill Qed.
}



Di seguito un elenco di limiti delle funzioni elementari. Alcuni sono abbastanza ovvi quindi non ci sono dimostrazioni allegate:

\begin{multicols}{2}
    \begin{equation*}
        \lim _{x \to +\infty} \ln(x) = +\infty
    \end{equation*}
    
    \begin{equation*}
        \lim _{x \to 0^+} \ln(x) = -\infty
    \end{equation*}
    
    \begin{equation*}
        \lim _{x \to +\infty} e^x = +\infty
    \end{equation*}

    \begin{equation*}
        \lim _{x \to -\infty} e^x = 0^+
    \end{equation*}

    \begin{equation*}
        \lim _{x \to +\infty} \sqrt{x} = +\infty
    \end{equation*}

    \begin{equation*}
        \lim _{x \to 0^+} \sqrt{x} = 0^+
    \end{equation*}

    \begin{equation*}
        \not \exists \lim _{x \to \pm \infty} \sin{x}
    \end{equation*}
    
    \begin{equation*}
        \not \exists \lim _{x \to \pm \infty} \cos{x}
    \end{equation*}
\end{multicols}

%NE MANCA QUALCUNA?

\subsection{Confronto di infiniti}

Se si hanno due funzioni che vanno a $+\infty$ si possono confrontare. Il confronto ci permette di determinare qualche delle due funzioni "\textit{va a infinito più velocemente}". Si riesce quindi a stilare una sorta di "gerarchia" degli infiniti dove alcuni tendono a $+\infty$ più velocemente di altri. Questo è molto utile negli esercizi e soprattutto in analisi della complessità computazionale.\\

Date quindi due funzioni $f(x)$ e $g(x)$ che tendono entrambe a $+ \infty$ se:
\begin{equation*}
    \lim \dfrac{f(x)}{g(x)} =
    \begin{cases*}
        0 \qquad \qquad g \text{ cresce più velocemente di } f\\
        +\infty \qquad \;\;\; f \text{ cresce più velocemente di } g\\
        l \neq 0 \qquad \; \text{$f$ e $g$ sono infinitesimi dello stesso ordine}
    \end{cases*}
\end{equation*}
La gerarchia degli infiniti risulta quindi (dal più "lento" al più "veloce"):
\begin{enumerate}
    \item $\ln_a x$ con $a > 1$
    \item $\sqrt[n]{x}$
    \item $p(x)$
    \item $a^x$
    \item $x^x$
\end{enumerate}
Ne consegue quindi che, per esempio, $x^{100^{100^{100}}}$ è "più lento" di $(1,000001)^x$, cioè che:
\begin{equation*}
    \lim_{x \to + \infty} \dfrac{x^{100^{100^{100}}}}{(1,000001)^x} = 0
\end{equation*}

\subsection{Continuità di una funzione}
\dfn{
$x_0 \in A$ si dice \textbf{punto isolato di} $A\subseteq \mathbb{R}$ se $x_o \not \in \mathcal{D}(A)$. In pratica se non è un punto di accumulazione
}
\dfn{
Una funzione $f: A \to \mathbb{R}$ si dice \textbf{continua} in $x_0 \in A$ se:
\begin{enumerate}
    \item $x_0 \not \in \mathcal{D}(A)$ (cioè $x_0$ è un punto isolato di $A$)
    \item $x_0 \in \mathcal{D}(A) \implies \lim \limits_{x\to x_0} f(x) = f(x_0)$
\end{enumerate}
(Si noti che le due opzioni non possono valere contemporaneamente, sono quindi congiunte da un \textit{or} invece che un \textit{and}).
}
Se $f:A \to \mathbb{R}$ è continua $\forall x \in A$. allora è continua su $A$. Si scrive
\begin{equation*}
    f \in \mathcal{C} (A)
\end{equation*}
dove $\mathcal{C}(A)$ è l'insieme delle funzioni continue su $A$, ed è definito come:
\begin{equation*}
    \mathcal{C}(A) = \{ f:A\to \mathbb{R}\; |\; f\; \text{è continua in x}, \forall x \in A \}
\end{equation*}
La continuità è un grado molto importante per "classificare" la regolarità di una funzione. Inoltre moltissimi teoremi che riguardano una o più funzioni richiedono che queste siano continue. L'idea che ci sta dietro alla continuità e il voler classificare rigorosamente tutte quelle funzioni "belle" che puoi disegnare senza staccare la mano dal foglio\footnote{In realtà non è esattamente così, però è un buon modo per iniziare a capire l'argomento}.\\
Come vi diranno tutti i professori di analisi però, la definizione di continuità che si basa sul disegnare una funzione senza staccare la mano dal foglio è sbagliata. Questo perché vale solo per quelle funzioni che sono continue su tutto $\mathbb{R}$, mentre per quelle che sono continue nel loro dominio, ma proprio questo dominio è composto dall'unione di molteplici intervalli disgiunti, allora restano continue, ma per disegnarle è necessario comunque staccare la mano dal foglio. %% DA RIGUARDARE

\dfn{
Il \textbf{dominio naturale} è il più grande sottoinsieme in cui una funzione è definita
}

Dai teoremi di algebra dei limiti segue che date due funzioni $f: A \to \mathbb{R}$ e $g: B \to \mathbb{R}$ continue in $x_0 \in A \cap B$ allora:
\begin{multicols}{2}
    \begin{itemize}
        \item $f \pm g$ è continua in $x_0$
        \item $c \in \mathbb{R}$ : $c\cdot f$ è continua in $x_0$
        \item $f \cdot g$ è continua in $x_0$
        \item $\dfrac{f}{g}$ è continua in $x_0$ (se $g(x_0) \neq 0$)
        \item $|f|$ è continua in $x_0$
        \item $g(f(x_0))$ è continua in $x_0$ se $x_0 \in A \land f(x_0) \in B$ e $f$ è continua in $x_0$ e $g$ è continua in $f(x_0)$
    \end{itemize}
\end{multicols}

\pf{
Dimostriamo che $|x|$ è una funzione continua.
\begin{equation*}
    f(x) = |x| =
    \begin{cases*}
        x \quad \;\;\, \text{se} \quad x \geq 0\\
        -x \quad \text{se} \quad x < 0
    \end{cases*}
\end{equation*}
Prendiamo un punto $x_0 \in \mathbb{R}$, abbiamo due casi:
\begin{itemize}
    \item Caso $x_0 \neq 0$:
        \begin{equation*}
            \lim_{x \to x_0} |x| =
            \begin{cases*}
                \text{se} \quad x_0 > 0: \lim_{x \to x_0} |x| = \lim_{x \to x_0} x = x_0 = |x_0|\\
                \\
                \text{se} \quad x_0 < 0: \lim_{x \to x_0} |x| = \lim_{x \to x_0} -x = -x_0 = |x_0|
            \end{cases*}
        \end{equation*}

    \item Caso $x_0 = 0$:
        \begin{align*}
            &\lim_{x \to 0^-} |x| = \lim_{x \to x^-} -x = 0\\
            &\lim_{x \to 0^+} |x| = \lim_{x \to x^+} x = 0
        \end{align*}
        Questo implica che:
        \begin{equation*}
            \lim_{x \to 0} |x| = 0 = |0|
        \end{equation*}
\end{itemize}
E quindi dalla definizione di continuità, $|x|$ è una funzione continua su tutto $\mathbb{R}$.
\hfill Qed.
}

\pf{
Dimostriamo che $\sin(x)$ è una funzione continua. In pratica dobbiamo provare che
\begin{equation*}
    \lim_{x \to x_0} \sin(x) = \sin(x_0) \qquad (\forall x_0 \in \mathbb{R})
\end{equation*}
Possiamo riscrivere $\sin(x)$ come $\sin (x_0 + (x - x_0))$. Chiamiamo $h$ il fattore $h :=(x-x_0)$. Quando $x \to x_0$, $h \to 0$. Quindi possiamo riscrivere il limite da dimostrare nel seguente modo:
\begin{equation*}
    \lim_{x \to h} \sin(x_0 + h) = \sin(x_0) \qquad (\forall x_0 \in \mathbb{R})
\end{equation*}
Avendo riscritto l'argomento del seno attraverso una somma, possiamo applicare le formule di addizione del seno:
\begin{equation*}
    \sin(x_0 + h) = \sin(x_0)\cos(h) + \sin(h)\cos(x_0)
\end{equation*}
Sostituendo quindi nel limite (in quanto per $h \to 0$, $\cos(h) \to 1$ e $\sin(h) \to 0$):
\begin{equation*}
    \lim_{x \to h} \sin(x_0 + h) =  \lim_{x \to h} \sin(x_0)\cos(h) + \sin(h)\cos(x_0) = \sin(x_0)
\end{equation*}
\hfill Qed.
}
\pf{
Dimostriamo che $\cos(x)$ è una funzione continua. In pratica dobbiamo provare che
\begin{equation*}
    \lim_{x \to x_0} \cos(x) = \cos(x_0) \qquad (\forall x_0 \in \mathbb{R})
\end{equation*}
Con lo stesso trucco usato nella dimostrazione precedente ci riduciamo dimostrare:
\begin{equation*}
     \lim_{x \to h} \cos(x_0 + h) = \cos(x_0) \qquad (\forall x_0 \in \mathbb{R})
\end{equation*}
Usando le formule di addizione del coseno:
\begin{equation*}
    \cos(x_0 + h) = \cos(x_0)\cos(h) - \sin(x_0)\sin(h)
\end{equation*}
Sostituiamo nel limite e calcoliamo (in quanto per $h \to 0$, $\sin(h) \to 0$ e $\cos(h) \to 1$):
\begin{equation*}
    \lim_{x \to h} \cos(x_0 + h) = \lim_{x \to h} \cos(x_0)\cos(h) - \sin(x_0)\sin(h) = \cos(x_0)
\end{equation*}
\hfill Qed.
}

La \textbf{continuità della tangente} nel suo dominio naturale è dovuta dal fatto che la tangente può essere definita come rapporto tra \textit{seno} e \textit{coseno}, e il rapporto di funzioni continue è anch'esso continuo. L'\textbf{esponenziale} è continuo. Tutte le \textbf{funzioni inverse delle funzioni elementari} sono continue ($\ln{x}$, $\sqrt{x}$, $\arcsin{x}$, $\arccos{x}$ e $\arctan{x}$).\\

\subsubsection{Funzioni definite a tratti}
Per le funzioni definite a tratti il caso è un po' più particolare perché non si sa a priori se sono continue o no. In generale il metodo per scoprire sono sono continue è il seguente. Prendiamo una funzione $f$ definita a tratti in questo modo:
\begin{equation*}
    f(x) =
    \begin{cases*}
        g(x) \qquad (x \leq a)\\
        h(x) \qquad (x > a)
    \end{cases*}
\end{equation*}
Di solito sia $g$ che $h$ sono funzioni formate da composizioni di funzioni elementari, che rendono automaticamente continue le due funzioni. Il problema sorge quindi nel punto di separazione $a$. Per verificare che sia continua la funzionem $f$ bisogna fare in modo che sia $g$ sia $h$ si avvicinino ad $a$ con lo stesso valore, cioè che abbiano lo stesso limite per $x \to a$.
\begin{equation*}
    \lim_{x \to a^-} g(x) = \lim_{x \to a^+} h(x)
\end{equation*}
Inoltre è richiesto che il limite coincida con il valore della funzione $f(a)$. In questo caso non abbiamo fatto il test perché il valore della funzione era compreso nella funzione $g$ in quanto era definito per $\leq$ invece che un minore stretto. Se però questo non fosse il caso va inserito $f(a)$ nelle uguaglianze dei limiti scritti sopra.
