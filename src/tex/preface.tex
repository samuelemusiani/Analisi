\usepackage[utf8]{inputenc}

\usepackage{amsmath, amssymb, amsfonts, cmll, mathabx, mathtools, physics} % Pacchetti per matematica
\usepackage[margin=3cm]{geometry} %Margine di pagina
\usepackage{graphicx} %Immagini
\usepackage{enumitem} %Per gli elenchi con numeri romani
\usepackage{multicol} %Per avere più colonne in una parte del documento
\usepackage[most]{tcolorbox} %BOX
\tcbuselibrary{breakable}
\usepackage{tikz} \usetikzlibrary{matrix} %Tabella Numeri Razionali (Q)
\usepackage{float} %Posizione tabelle
%\usepackage{gensymb} %gradi sessagesimali

\usepackage{xcolor} %Colori personalizzati
\usepackage{enumitem} %Lettere nelle liste numerate

\usepackage{subcaption} %Per le doppie caption in immagini vicine



\usepackage{tikz, pgfplots} %GRAFICI
\pgfplotsset{compat=1.18}


%% ---------------- ARCO --------------------------
%% Tutto il comando seguente serve per inserire l'arco sopra due lettere, per indicare l'arco di circonferenza tra due punti
\makeatletter
\DeclareFontFamily{U}{tipa}{}
\DeclareFontShape{U}{tipa}{m}{n}{<->tipa10}{}
\newcommand{\arc@char}{{\usefont{U}{tipa}{m}{n}\symbol{62}}}%

\newcommand{\arc}[1]{\mathpalette\arc@arc{#1}}

\newcommand{\arc@arc}[2]{%
  \sbox0{$\m@th#1#2$}%
  \vbox{
    \hbox{\resizebox{\wd0}{\height}{\arc@char}}
    \nointerlineskip
    \box0
  }%
}
\makeatother
%% ---------------- FINE ARCO -------------------


\tikzset{ %Tabella numeri razionali
table/.style={
  matrix of nodes,
  row sep=-\pgflinewidth,
  column sep=-\pgflinewidth,
  nodes={rectangle,text width=3em,align=center},
  text depth=1.25ex,
  text height=2.5ex,
  nodes in empty cells
},
row 6/.style={nodes={text depth=0.4ex,text height=2ex}},
}

%DEFINIZIONE COLORI
\definecolor{rosso_0}{RGB}{150, 98, 113}
\definecolor{rosso_1}{RGB}{240, 225, 229}
\definecolor{blu_0}{RGB}{102, 133, 176}
\definecolor{blu_1}{RGB}{240, 245, 255}
\definecolor{arancione_0}{RGB}{255, 179, 117}
\definecolor{arancione_1}{RGB}{255, 244, 230}
\definecolor{green_0}{RGB}{210, 255, 189}

%AMBIENTI DEFINITI
\newenvironment{definition}
{\begin{tcolorbox}[title=Definizione, breakable, bottomrule at break = 0mm, toprule at break = 0mm, sharp corners, colframe=blu_0, colback=blu_1]
}{
\end{tcolorbox}}

\newenvironment{proof}
{\begin{tcolorbox}[title = Dimostrazione, breakable, bottomrule at break = 0mm, toprule at break = 0mm, colframe=arancione_0, coltitle=black, colback=arancione_1]
}{
\end{tcolorbox}}

\newenvironment{theorem}
{\begin{tcolorbox}[title = Teorema, bottomrule at break = 0mm, toprule at break = 0mm, colframe=rosso_0, colback=rosso_1]
}{
\end{tcolorbox}}

\newenvironment{mlemma}
{\begin{tcolorbox}[title = Lemma, breakable, bottomrule at break = 0mm, toprule at break = 0mm, colframe=brown, coltitle=black, colback=brown!40!white]
}{
\end{tcolorbox}}

\newenvironment{important}
{\begin{tcolorbox}[coltitle=black, sharp corners, colback=green_0, colframe=white]
}{
\end{tcolorbox}}

\newcommand{\dfn}[1]{\begin{definition}{#1}\end{definition}}
\newcommand{\pf}[1]{\begin{proof}{#1}\end{proof}}
\newcommand{\thm}[1]{\begin{theorem}{#1}\end{theorem}}
\newcommand{\mlem}[1]{\begin{mlemma}{#1}\end{mlemma}}
\newcommand{\imp}[1]{\begin{important}{#1}\end{important}}
