\section{Funzioni in $\mathbb{R}^n$}
Per definire le funzione abbiamo bisogni di definire gli intorni di un punto. In $\mathbb{R}^1$ abbiamo definito gli intorni sferici di un punto, mentre in $\mathbb{R}^n$ li definiamo come segue:
\dfn{
	Dato un \textbf{centro} $x \in \mathbb{R}^n$ e un \textbf{raggio} $r > 0$ poniamo 
	\begin{equation*}
		\mathcal{B}(x, r) = \{y \in \mathbb{R}^n \; | \; \lVert y - x \rVert < r\}
	\end{equation*}
	$\mathcal{B}$ si chiama \textbf{disco di centro $\mathbf{x}$ e raggio $\mathbf{r}$}
}
In pratica sarebbe l'insieme di punti che hanno distanza da $x$ minore o uguale a $r$. Se ci spostiamo un attimo in $\mathbb{R}^2$ e proviamo a prendere $x = (0, 0)$ e $r > 0$:
\begin{equation*}
	\mathcal{B}((0, 0), r) = \{(x, y) \in \mathbb{R}^2 \; | \; \lVert (x, y) - \underline{0} \rVert < r\}
\end{equation*}
Se si espande la definizione di norma si può notare che la condizione di appartenenza a questo insieme è:
\begin{equation*}
	\sqrt{x^2 + y^2} < r
\end{equation*}
Che corrisponde a tutti i punti contenuti in una circonferenza di raggio $r$ con centro nell'origine. Facendo invece lo stesso ragionamento ma con un punto $x = (x_0, y_0)$ generico si arriva alla condizione:
\begin{equation*}
	\sqrt{(x - x_0)^2 + (y - y_0)^2} < r
\end{equation*}
Dove anche questa volta l'equazione è una circonferenza di raggio $r$ ma con il centro spostato in $(x_0, y_0)$.

\dfn{
	Il \textbf{grafico} di $f: A \to \mathbb{R}^q$ con $A \subseteq \mathbb{R}^m$ è:
	\begin{equation*}
		\text{Graf}(f) = \{(x, f(x)) \in A\times \mathbb{R}^q\}
	\end{equation*}
}

\subsection{Funzioni scalari, affini, radiali e cilindriche}
Esistono vari tipi di funzioni in $\mathbb{R}^n$. In particolare quelle che hanno codomino $\mathbb{R}^1$ sono dette \textbf{funzioni scalari} in quanto appunto restituiscono uno scalare. Più precisamente:
\dfn{
	Una funzione scalare è del tipo $f: A \to \mathbb{R}$ dove $A \subseteq \mathbb{R}^n$.
}
Esempio: $f: \mathbb{R}^2 \to \mathbb{R}$ dove $f(x, y) = x^2 + y^2 = \lVert (x, y) \rVert ^2$. Se riprendiamo la definizione di grafico risulta che:
\begin{equation*}
	\text{Graf}(f) = \{(x, y, x^2 + y^2) \; | \; x, y \in \mathbb{R}^2\}
\end{equation*}
Si può notare che $(x, y, x^2 + y^2) \in \mathbb{R}^3$. Se lo intersechiamo con il piano $\pi = \{(0, y, z) \; | \; y, z \in \mathbb{R}\}$ il grafico diventa:
\begin{equation*}
	\text{Graf}(f) \medcap \pi = \{(0, y, y^2) \; | \; y \in \mathbb{R}^2\}
\end{equation*}
Che corrisponde all'insieme descritto dalle equazioni:
\begin{equation*}
	\begin{cases}
		x = 0\\
		z = y^2
	\end{cases}
\end{equation*}
Che non è altro che una parabola. %(DA FARE IL GRAFICO)
Se riscriviamo il nostro punto $(x, y)$ in coordinate polari e ricalcoliamo il valore della funzione:
\begin{equation*}
	f(x, y) = f(r\cos{x}, r\sin{x}) = r^2\cos^2(x) + r^2\sin^2(x) = r^2(\cos^2(x) + \sin^2(x)) = r^2
\end{equation*}
Prendendo ora $\pi_\theta = \{(r\cos{\theta}, r\sin\theta, z) | r \geq 0, z \in \mathbb{R}\}$ e lo interseco con il grafico di $f$:
\begin{equation*}
	\text{Graf}(f) \medcap \pi_\theta = \{(r\cos{\theta}, r\sin\theta, r^2) \; | \; r \geq 0\}
\end{equation*}
%(aggiungere foto GRAFICO)

Un ulteriore esempio sono le \textbf{funzionni radiali}:
\dfn{
	Una fuzione si dice \textbf{radiale} se è esprimibile nella forma:
	\begin{equation*}
		f: \mathbb{R}^2 \to \mathbb{R} \quad f(x, y) = g(\lVert (x, y) \rVert)
	\end{equation*}
	Con $g$ funzione \textit{opportuna}\footnote{In questo caso con funzione opportuna si intende una funzione sufficientemente regolare. Non siamo andati troppo nei dettagli in questo caso quindi riporto quello detto dal prof.} definita in $g:[0, +\infty[ \to \mathbb{R}$
}
Queste funzioni sono molto particolare perché il loro grafico è dato da rotazioni intorno all'asse $z$. Possiamo infatti considerare la funzione presa come esempio prima: 
\begin{equation*}
	f(x, y) = x^2 + y^2 = g(\lVert (x, y) \rVert) \quad \text{con} \;\; g(r) = r^2
\end{equation*}
Il grafico come abbiamo visto è una rotazione intorno all'asse $z$ di una parabola. L'equazione della parabola infatti risiede proprio in $g(r) = r^2$. È infatti questo il grafico che "ruota".

Se prendiamo un'altra funzione, tipo $g(r) = 1 -r$ e dove quindi:
\begin{equation*}
	f(x, y) = g(\lVert x, y \rVert) = 1 - \lVert x, y \rVert = 1 - \sqrt{x^2 + y^2}
\end{equation*}
Il grafico viso nella forma completa sembra complesso da disegnare, ma essendo questa una funzione radiale ci basta osservare la funzione $g$, disegnare il suo grafico e farlo ruotare intorno all'asse $z$. Il grafico risulta infatti un cono. %(FARE GRAFICO)

\dfn{
	Si definisce \textbf{funzione affine} una funzione nella forma:
	\begin{equation*}
		f: \mathbb{R}^2 \to \mathbb{R} \qquad f(x, y) = ax + by + c \;\; \text{con}\;\; a, b, c \in \mathbb{R}
	\end{equation*}
} 
Il grafico di queste funzioni risulta essere un piano in $\mathbb{R}^3$:
\begin{equation*}
	\text{Graf}(f) = \{(x, y, ax + by + c) | (x, y) \in \mathbb{R}^2\}
\end{equation*}
Prendiamo come esempio la funzione $f(x, y) = -y$: se la intersechiamo con il piano $\pi = \{(0, y, z) \; | \; y, z \in \mathbb{R}\}$ otteniamo un grafico del tipo:
\begin{equation*}
	\text{Graf}(f) \medcap \pi = \{(0, y, -y) | y \in \mathbb{R}\}
\end{equation*}
Che risulta un sempilce $z = -y$. Essendo che non dipende da $x$ possiamo intersecare tale grafico con quealsisi piano in cui la coordinata $x$ abbia un valore fissato. Ne cosegue che il grafico completo si ottiene traslando lungo l'asse $x$ la retta $z = -y$. %(FARE GRAFICO)

\dfn{
	Si definisce \textbf{funzione cilindrica} una funzione in due variabili che dipende da solo una delle due:
	\begin{equation*}
		f(x, y) = h(y) \qquad \text{oppure} \qquad f(x, y) = g(x)
	\end{equation*}
	Con $h$ e $g$ opportune.
}
Il grafico di queste funzioni si ottiene disegnando la funzione $z = h(y)$ (o nel caso $z = g(x)$) e successivamente traslando tale grafico sull'asse da cui non dipende la funzione. Per esempio la funzione presa prima in cosiderazione $f(x, y) = -y$ è una funzione cilindrica e si ottiene appunto tracciando la retta $z = -y$ e translado tale retta sull'asse $x$, cioè quello da cui non dipende la funzione. %(FARE GRAFICO)

Il fatto di traslare sull'asse da cui non dipende la funzione è semplicemente dato dal fatto che in $\mathbb{R}^3$ una grafico del tipo $\text{Graf}(f) = \{(x, y, h(y)) \;|\; x, y \in \mathbb{R}\}$ non impone nessun vincolo sul paramentro $x$, ne consegue che vale per tutti i suoi valori. Non siamo infatti constretti a disegnare tale funzione in $x = 0$, ma possiamo disegnarla per un qualsiasi valore di $x$ e poi traslare tutto il grafico lungo l'asse.

\subsection{Continuità funzioni scalari}
\dfn{
	Data una funzione $f: A \to \mathbb{R}$ con $A \subseteq \mathbb{R}^n$ e un punto $x_0 \in \mathbb{A}$, si dice che $f$ è \textbf{continua in $\mathbf{x_0}$} se:
	\begin{equation*}
		\forall (x_k)_{k \in \mathbb{N}} \subseteq A \quad (\text{cioè di punti in } A)
	\end{equation*}
	vale:
	\begin{equation*}
		x_k \xrightarrow[k \to +\infty]{} x_0 \implies f(x_k) \xrightarrow[k \to +\infty]{} f(x_0)
	\end{equation*}
	Oppure equivalentemente: $f$ è continua in $x_0 \in A$ se: 
	\begin{equation*}
		\forall \epsilon > 0, \exists \delta > 0 :
		\begin{cases}
			x \in A\\
			x \in \mathcal{B}(a, \delta)
		\end{cases}
		\implies | f(x) - f(x_0) | < \epsilon %(E' un valore assoluto o una norma??)
	\end{equation*}
	Dove $x \in B(x_0, \delta)$ è il disco di centro $x_0$ e raggio $\delta$.
}
\imp{
	Tutte le funzioni "elementari" sono \textbf{continue} nei loro domini.
}
Una funzione quindi del tipo
\begin{equation*}
	f(x, y) = \dfrac{\ln(y -x^2)}{\sqrt{1+\sin^2(xy)}}
\end{equation*}
è continua nel suo dominio, cioè:
\begin{equation*}
	\text{Dom}(f) = \{(x, y) \in \mathbb{R}^2 \;|\; y - x^2 > 0\}
\end{equation*}

\subsection{Derivabilità funzioni scalari}
In generale quando ci si trova in $\mathbb{R}^1$ si definisce la derivata di una funzione su un intervallo aperto del tipo $]a, b[$. Questo perché se si definisce la derivata su un intervallo chiuso si avrebbero dei problemi in quanto potrebbe esistere solo la derivata destra o sinistra in un punto. Definiamo quindi l'equivalente degli intervalli aperti in $\mathbb{R}^n$.
\dfn{
	Dato $A \subseteq \mathbb{R}^n$, dico che \textbf{$\mathbf{A}$ è aperto} se: 
	\begin{equation*}
		\forall x_0 \in A, \exists \epsilon > 0: \mathcal{B}(x_0, \epsilon) \subseteq A
	\end{equation*}
}
Se ci mettiamo in $\mathbb{R}^2$ un insieme di questo tipo è dato da un figura con i bordi tratteggiati, cioè dove tutti i punti sul bordo non appartengono all'insieme $A$. %(FARE FIGURA?)

\thm{
	\textbf{Individuazione degli intervalli aperti:} Data $f: \mathbb{R}^n \to \mathbb{R}$ continua, $\forall b \in \mathbb{R}$ l'insieme $\{x \in \mathbb{R}^n \;|\; f(x) > b\}$ è aperto. L'apertura è data dal fatto che l'insieme è definito tramite ugualianza stretta.
}

\subsubsection{Derivate parziali}
\dfn{
	Dato $A \subseteq \mathbb{R}^2$ aperto e sia $(x_0, y_0) \in A$. Si dice che $f$ \textbf{è derivabile rispetto a $\mathbf{x}$ in $\mathbf{(x_0, y_0)}$} se esiste \textbf{finito}:
	\begin{equation*}
		\pdv{f}{x}() (x_0, y_0) \vcentcolon = \lim_{t \to 0} = \dfrac{f(x_0 + t, y_0) - f(x_0, y_0)}{t} \in \mathbb{R}
	\end{equation*}
	Questo limite si indica con il simbolo di derivata parziale ($\partial$). Di seguito alcune notazioni equivalenti:
	\begin{equation*}
		\pdv{f}{x}() = \partial_x f = D_x f
	\end{equation*}
	Equivalentemente possiamo riscrivere il limite come:
	\begin{equation*}
		\pdv{f}{x}() (x_0, y_0) \vcentcolon = \lim_{x \to x_0} = \dfrac{f(x, y_0) - f(x_0, y_0)}{x - x_0} \in \mathbb{R}
	\end{equation*}
}
Vale ovviamente che la derivata rispetto a $y$ in $(x_0, y_0)$ risulta:
\begin{equation*}
		\pdv{f}{y}() (x_0, y_0) \vcentcolon = \lim_{t \to 0} = \dfrac{f(x_0, y_0 + t) - f(x_0, y_0)}{t} \in \mathbb{R}
\end{equation*}
Opppure equivalentemente:
\begin{equation*}
	\pdv{f}{y}() (x_0, y_0) \vcentcolon = \lim_{y \to y_0} = \dfrac{f(x_0, y) - f(x_0, y_0)}{y - y_0} \in \mathbb{R}
\end{equation*}

Esempio: prendiamo la funzione $f(x, y) = xy^2$. Le sue derivate parziali sono::
\begin{equation*}
	\pdv{f}{x}() (x, y) = y^2 \qquad \pdv{f}{y}() (x, y) = 2xy
\end{equation*}
L'idea per calcolare queste derivate è quella di considerarle come una funzioni normale (cioè come quelle viste in $\mathbb{R}^1$) e di considerare come variabile quella che bisogna derivare, mentre tutte le altre vengono considerate costanti. Nel primo caso infatti, dove abbiamo derivato rispetto alla $x$, ci è bastato considerare la $y$ costante. Quindi è come se avessimo fatto la derivata in $\mathbb{R}^1$ della semplice funzione:
\begin{equation*}
	f(x) = a^2 \cdot x \implies f'(x) = a^2
\end{equation*}
Dove però al posto della $a$ ci sarebbe la $y$. Nel secondo caso invece la derivata è rispetto alla $y$, quindi la $x$ viene considerata come semplice costante: $f(y) = a \cdot y^2 \implies f'(y) = 2ay$. Dove ovviamente al posto della $a$ ci sarebbe la $x$.\\

Se si vuole \textbf{generalizzare la definzione di derivata pariziale} è necessario riprendere la definzione di vettori coordinati (Sezione: \ref{sec_vettoriCoordinati}). In generale quindi data $f: \mathbb{R}^n \to \mathbb{R}$ e un punto $\overline{x} \in \mathbb{R}^n$ si definisce la \textit{i}-esima derivata parziale come:
\begin{equation*}
	\pdv{f}{x_i}()(\overline{x}) = \lim_{t \to 0} \dfrac{f(\overline{x} + t \cdot e_i) - f(\overline{x})}{t}
\end{equation*}

\subsubsection{Gradiente}
Le derivate parziali si possono aggregare in una funzione vettoriale (cioè che restituisce un vettore) chiamata gradiente:
\dfn{
	Data una funzione $f: A \in \mathbb{R}$ con $A \subseteq \mathbb{R}^n$, se esistono tutte le derivate pariziali di $f$ in ogni punto possiamo definire \textbf{il gradiente di $\mathbf{f}$} come:
	\begin{equation*}
		\nabla f(x_1, x_2, \cdots) = \left(\pdv{f}{x_1}()(x_1, x_2, \cdots), \pdv{f}{x_2}() (x_1, x_2, \cdots), \cdots  \right)
	\end{equation*}
	Nel caso particolare di $\mathbb{R}^2$:
	\begin{equation*}
		\nabla f(x, y) = \left(\pdv{f}{x}()(x, y), \pdv{f}{y}()(x, y)\right)
	\end{equation*}
	Il simbolo del gradiente ($\nabla$ si lege \textit{nabla}).
}
Esempio: data la funzione $f(x, y) = \sin(x + y^2)$ il suo gradiente risulta:
\begin{equation*}
	\nabla f (x, y) = (\cos(x + y^2), \cos(x + y^2) \cdot 2y)
\end{equation*}
