\section{Punti critici}
Per trovare i punti critici di una funzione non possiamo usare lo stesso 
approccio usato per le funzioni ad una varibiale in quanto in $\mathbb{R}^n$ 
non riusciamo a definire una vettore maggiore di zero e quindi non sappiamo 
distinguere funzioni crescenti e decrescenti. Esiste però un altro approccio 
per determinare se un punto è di minimo o di massimo che si basa sulle derivata 
seconda in un punto. Se infatti prendiamo una funzione $f: \mathbb{R} \to 
\mathbb{R}$ con derivata seconda continua possiamo dire che un punto $x_0$ è 
un punto di minimo se: 
\begin{equation*}
	\begin{cases}
		f'(x_0) = 0\\
		f''(x_0) > 0
	\end{cases}
\end{equation*}
Vale invece che se $f''(x_0) < 0$ il punto è di massimo mentre se $f''(x_0) 
= 0$ il punto è di flesso. Per definire quindi i punti di massimo, di minimo e 
di sella per le funzioni in più variabili useremo un approccio molto simile a 
questo.
\dfn{
	\textbf{Punto di massimo o minimo}: Dato $A \subseteq \mathbb{R}^n$, $f: A 
    \to \mathbb{R}$ e un punto $x_0 \in A$. Il punto $x_0$ si dice di minimo 
    per $f$ se:
	\begin{equation*}
		\exists \delta > 0: f(x) \geq f(x_0) \;\; \forall x \in A \cap 
        \mathcal{B}(x_0, \delta)
	\end{equation*}
	Per i massimi la condizione risulta essere: $f(x) \leq f(x_0)$. Inoltre 
    $x_0$ è un minimo assoluto se:
	\begin{equation*}
		f(x) \geq f(x_0) \quad \forall x \in A
	\end{equation*}
}

\thm{
	\textbf{Condizione necessaria del primo ordine per punti di massimo o di 
    minimo:} Dato $A \subseteq \mathbb{R}^n$ aperto e una funzione $f: A \to 
    \mathbb{R}$ differenziabile. Se $x_0 \in A$ è un punto di massimo (o di 
    minimo) locale allora:
	\begin{equation*}
		\nabla f(x_0) = \underline{0} \in \mathbb{R}^n
	\end{equation*}
	Se vale questa condizione il punto $x_0$ si dice \textbf{punto critico} (o 
    \textbf{stazionario}).
}
Da notare che questa condizione è estremamente simile a quella che si 
verificava nelle funzioni ad una varibiale in quanto il gradiente in 
$\mathbb{R}^1$ è semplicemente la derivata prima della funzione.
\pf{
	La prova la riduciamo solo al caso di $\mathbb{R}^2$. Prendiamo $A 
    \subseteq \mathbb{R}^2$ aperto  e una funzione $f: A \to \mathbb{R}$. 
    Assumiamo inoltre che il punto $(x_0, y_0)$ si un punto di minimo. 
    Dimostriamo ora che:
	\begin{equation*}
		\nabla f(x_0, y_0) = \underline{0} \in \mathbb{R}^n
	\end{equation*}
	In pratica ci riduciamo a dimostrare che:
	\begin{equation*}
		\pdv{f}{x}(x_0, y_0) = 0 \quad \land \quad \pdv{f}{y}(x_0, y_0) = 0
	\end{equation*}
	Riduciamoci a dimostrare solo la derivata rispetto ad $x$ in quanto l'altro 
    caso è analogo. Per farlo consideriamo la funzione ausiliaria:
	\begin{equation*}
		h(t) = f(x_0 + t, y_0)
	\end{equation*}
	Essa è definita per $t$ in un intorno dell'origine ed è derivabile. Visto 
    che $f$ ha minimo locale in $(x_0, y_0)$ per la definzione di punto di 
    minimo:
	\begin{equation*}
		h(t) \geq h(0) \quad \forall t \; \text{in un intorno di } 0
	\end{equation*}
	Notiamo infatti che $h(0) = f(x_0, y_0)$. Inoltre:
	\begin{equation*}
		h'(t) = \partial_x f(x_0 + t, y_0)
	\end{equation*}
	Per il teorema del primo sementre, in quanto $h$ è una funzione ad una 
    varibiale che ha un minimo in $0$:
	\begin{equation*}
		h'(0) = 0
	\end{equation*}
	Ne consegue che:
	\begin{equation*}
		\partial_x f(x_0 + t, y_0) = 0
	\end{equation*}
	\hfill Qed.
}

\subsection{Teorema classificazione punti critici}
\thm{
	Sia $f: A \to \mathbb{R}$ una funzione con derivate prime e seconde 
	continue sull'insieme aperto $A \subseteq \mathbb{R}^2$ e sia $x_0$ un 
	punto critico (quindi $\nabla f(x_0) = \underline{0}$). Allora:
	\begin{itemize}
		\item Se $\mathbf{H}f > 0$, allora $x_0$ è un punto di minimo
			locale.
		\item Se $\mathbf{H}f < 0$, allora $x_0$ è un punto di massimo
			locale.
		\item Se $\mathbf{H}f$ è \textit{indefinita}, allora $x_0$ è un punto 
            di sella.
	\end{itemize}
}
Prima di della dimostrazione è necesarrio dimostrare un lemma:
\mlem{
    Se $A = A^T \in \mathbb{R}^{n \times n}$ è definita positiva, allora:
    \begin{equation*}
        \exists m: \quad \langle Ah, h \rangle \geq m \lVert h \rVert^2 \quad \forall h \in \mathbb{R}^n
    \end{equation*}
    Dimostrazione:
    Fissiamoci in due dimensioni ed usiamo le coordinate polari. In particolare il vettore $h$ si può scrivere nella forma $h = (r\cos{\theta}, r\sin{\theta})$ con $r > 0$ e $\theta \in [0, 2\pi]$. Mettiamo che $A$ sia nella forma $A = \begin{bmatrix} a & b\\ b & c \end{bmatrix}$. Risulta quindi:
    \begin{equation*}
        \langle Ah, h \rangle = r^2(a\cos^2(\theta) + 2b\cos(\theta)\sin(\theta) + c\sin^2(\theta))
    \end{equation*}
    Notiamo che $r^2 = \lVert h \rVert^2$. Definiamo inoltre quanto appena ottenuto come una funzione $g$ rispetto al parametro $\theta$:
    \begin{equation*}
        g(\theta) \vcentcolon = (a\cos^2(\theta) + 2b\cos(\theta)\sin(\theta) + c\sin^2(\theta))
    \end{equation*}
    La funzione $g(\theta) > 0 \quad \forall \theta \in [0, 2\pi]$ ed inoltre è continua. Indicando quindi con $\theta_0 \in [0, 2\pi]$ un suo punto di minimo assoluto e con $m \vcentcolon = g(\theta_0) > 0$ il suo minimo assoluto si ottiene:
    \begin{equation*}
        \langle Ah, h \rangle = \lVert h \rVert^2 g(\theta) \geq m \lVert h \rVert^2
    \end{equation*}
    \hfill Qed.
}
Dimostriamo ora solo il primo caso, cioè quello del punti di minimo:
\pf{
    Sia $f: A \to \mathbb{R}$ con $A \subseteq \mathbb{R}$ aperto. Sia $x_0 \in A$ tale che $\nabla f(x_0) = 0$ e $\mathbf{H}f(x_0) > 0$. Dimostriamo che $x_0$ è un minimo locale, cioè:
    \begin{equation*}
        \exists \delta > 0 : f(x_0 + h) \geq f(x_0) \quad \forall h \in B(0, \delta)
    \end{equation*}
    Possiamo usare la fomula di Taylor di secondo ordine ricordandoci che il gradiente è nullo:
    \begin{equation*}
        f(x_0 + h) - f(x_0) = \dfrac{1}{2} \langle \mathbf{H}f(x_0) h, h \rangle + o(\lVert h \rVert^2)
    \end{equation*}
    Dal lemma:
    \begin{equation*}
    \exists m > 0 : \langle \mathbf{H}f(x_0) h, h \rangle \geq m \lVert h \rVert^2 \quad \forall h \in \mathbb{R}^2
    \end{equation*}
    Dunque:
    \begin{equation*}
        f(x_0 + h) - f(x_0) \geq \dfrac{m}{2} \lVert h \rVert^2 + o(\lVert h \rVert^2) \quad \text{per } h \to 0
    \end{equation*}
    Per definzione di o-piccolo (con $\epsilon = \frac{m}{4}$):
    \begin{equation*}
        \exists \delta > 0 : \forall \langle h \rangle < \delta \implies |o(\lVert h \rVert^2)| \leq \dfrac{m}{4} \lVert h \rVert^2
    \end{equation*}
    Notiamo che $\forall \langle h \rangle$ è come dire $\forall h \in B(0, \delta)$. Eliminando il valore assoluto prendendo solo la parte negativa:
    \begin{equation*}
        o(\lVert h \rVert^2) \geq -\dfrac{m}{4} \lVert h \rVert^2 \quad \forall h \in B(0, \delta)
    \end{equation*}
    Risostituendo nella formula precedente:
    \begin{equation*}
        f(x_0 + h) - f(x_0) \geq \dfrac{m}{2} \lVert h \rVert^2 -\dfrac{m}{4} \lVert h \rVert^2 \quad \forall h \in B(0, \delta)
    \end{equation*}
    Che semplificando diventa:
    \begin{equation*}
        f(x_0 + h) - f(x_0) \geq \dfrac{m}{4} \lVert h \rVert^2 \quad \forall h \in B(0, \delta)
    \end{equation*}
    Essendo tutta la parte di destra positiva, abbiamo dimostrato che $x_0$ è un minimo.
    \hfill Qed.
}
