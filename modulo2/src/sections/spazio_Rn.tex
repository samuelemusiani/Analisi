\section{Lo spazio $\mathbb{R}^n$}
\subsection{La struttura lineare}
\dfn{
	$\mathbb{R}^n$ è l'insieme delle \textit{n}-uple ordinate di numeri:
	\begin{equation*}
		\mathbb{R}^n = \{x = (x_1, x_2, \cdots , x_n) \; | \; x_1, x_2, \cdots, 
        x_n \in \mathbb{R}\}
	\end{equation*}
}
Gli elementi di $\mathbb{R}^n$ si dicono vettori. In generale si possono 
pensare i sui elementi come vettori applicati nell'origine. Lo spazio 
$\mathbb{R}^n$ ha una struttra lineare. In esso sono definite due operazioni 
di base: la somma tra vettori e il prodotto di un vettore per uno scalare.

\subsubsection{Somma tra vettori}
\dfn{
	Presi due vettori $x, y \in \mathbb{R}^n$ dove $x = (x_1, \cdots, x_n)$ e 
    $y = (y_1, \cdots, y_n)$ definiamo la somma tra vettori:
	\begin{equation*}
		x + y \vcentcolon = (x_1 + y_1, \cdots, x_n + y_n) \in \mathbb{R}^n
	\end{equation*}
}

Se ci si riduce ad $\mathbb{R}^2$ la somma tra vettori è la stessa della 
\textit{regola del parallelogramma} che spesso si studia in fisica.

\subsubsection{Prodotto di un vettore per uno scalare}
\dfn{
	Dato un vettore $x = (x_1, \cdots, x_n) \in \mathbb{R}^n$ e uno scalare 
    $\lambda \in \mathbb{R}$ definiamo il prodotto di un vettore per uno 
    scalare:
	\begin{equation*}
		\lambda x \vcentcolon = (\lambda \cdot x_1, \cdots, \lambda \cdot x_n)
	\end{equation*}
}
I numeri in $\mathbb{R}$ vengono chiamati scalari perché se moltiplicati per un 
vettore visivamente lo scalano proprio del loro valore. Cioè se $\lambda = 2$ 
allora il vettore che viene moltiplicato per $\lambda$ è il doppio del vettore 
originale.

\subsubsection{Vettori coordinati} \label{sec_vettoriCoordinati}
%(Non so se è una definzione) DA RIGUARDARE
\dfn{
	In $\mathbb{R}^n$ i \textbf{vettori coordinati} sono:
	\begin{align*}
		e_1 &= (1, 0, 0, \cdots, 0)\\
		e_2 &= (0, 1, 0, \cdots, 0)\\
		\vdots\\
		e_n&= (0, 0, 0, \cdots, 1)\\
	\end{align*}
}

I vettori coordinati sono ortogonali a coppie (per la definzione di 
ortogonalità fare riferimento alla sezione \ref{sec_ortogonalita}):
\begin{equation*}
	\forall j \neq k, \;\;j,k = 1\cdots n \quad \langle e_j, e_k \rangle = 0
\end{equation*}

\subsection{Prodotto scalare euclideo}
\dfn{
	Dati $x, y \in \mathbb{R}^n$ dove $x = (x_1, \cdots, x_n)$ e $y = (y_1, 
    \cdots, y_n)$ definiamo il prodotto scalare euclideo:
	\begin{equation*}
		\langle x, y \rangle = x_1 \cdot y_1 + \cdots + x_n \cdot y_n = 
        \sum_{k = 1}^n x_k \cdot y_k \in \mathbb{R}
	\end{equation*}
	È importante notare che questo prodotto restituisce \textbf{un numero in} 
    $\mathbb{R}$. Esiste anche un altra notazione per scrivere il prodotto 
    scalare: 
	\begin{equation*}
		\langle x, y \rangle = x \cdot y
	\end{equation*}
}
Esempio:
\begin{equation*}
	\langle (2, 3), (-1, 4) \rangle = 2 \cdot -1 + 3 \cdot 4 = 10
\end{equation*}
Altro esempio:
\begin{equation*}
	\langle (2, 1, 3), (-1, 4, 1) \rangle = 2 \cdot -1 + 1 \cdot 4 + 3 
    \cdot 1 = 5
\end{equation*}

\textbf{Proprietà:}
\begin{enumerate}
	\item $\langle x, y \rangle = \langle y, x \rangle \quad \forall x, y \in 
        \mathbb{R}^n$

	\item Dati $x, y, z \in \mathbb{R}^n$ e $\lambda, \mu \in \mathbb{R}$ 
		\begin{equation*}
			\langle \lambda x + \mu y, z \rangle = \lambda \langle x, y 
            \rangle + \mu \langle y, z \rangle
		\end{equation*}
		Ovviamente per la proprietà 1 vale:
		\begin{equation*}
			\langle z, \lambda x + \mu y \rangle = \lambda \langle z, x 
            \rangle + \mu \langle z, y \rangle
		\end{equation*}

	\item $\forall x \in \mathbb{R}^n \quad \langle x, x \rangle \geq 0$
	
	\item $\langle x, x \rangle = 0 \iff x = \underline{0} = (0, \cdots, 0)$
\end{enumerate}

Se ci riduciamo allo spazio $\mathbb{R}^2$ spesso viene data un'altra 
definzione di prodotto scalare euclideo. La sua definzione si basa sulla norma 
(Sezione: \ref{sec_norma}), mentre la sua dimostrazione di equivalenza si 
basa sulle coordinate polari (Sezione: \ref{sec_coordinatePolari}) che sono 
entrambi concetti che vengono introdotti in seguito. Nonostante ciò viene 
riportata qui la definzione per mantenere coerente la suddivisione per 
argomenti. 

Se abbiamo due vettori $x, y \in \mathbb{R}^2$ e chiamiamo \textit{l'angolo 
compreso tra i due} $\alpha$ allora:
\begin{equation*}
	\langle x, y \rangle = \lVert x \rVert \cdot \lVert y \rVert \cdot 
    \cos{\alpha}
\end{equation*}

Per la dimostrazione scriviamo $x$ e $y$ come:
\begin{align*}
	x &= (x_1, y_1) = (r_1\cos{\theta_1}, r_1\sin{\theta_1}) \\
	y &= (x_2, y_2) = (r_2\cos{\theta_2}, r_2\sin{\theta_2})
\end{align*}
%Fare la figura. Appunti 2023-03-13

Ne consegue che:
\begin{equation*}
	\langle x, y \rangle = \langle (x_1, y_1), (x_2, y_2) \rangle = r_1r_2
    \cos{\theta_1}\cos{\theta_2} + r_1r_2\sin{\theta_1}\sin{\theta_2} =
\end{equation*}
Ricordandoci le formule di addizione e sottrazione del coseno possiamo 
riscrivere quanto segue nella forma:
\begin{equation*}
	r_1r_2\cos{\theta_2 - \theta_1} 
\end{equation*}
Essendo che $\theta_2 - \theta_1$ non è altro che l'angolo tra i due vettori 
che noi abbiamo chiamato $\alpha$ possiamo riscrivere come:
\begin{equation*}
	r_1r_2 \cos{\alpha} 
\end{equation*}
Nelle coordiante polari abbiamo chiamato $ r_1 = \lVert (x_1, y_1) \rVert$, di 
conseguenza possiamo riscrivere come:
\begin{equation*}
	\lVert (x_1, y_1) \rVert \cdot \lVert (x_2, y_2) \rVert \cdot \cos{\alpha} 
    = \lVert x \rVert \cdot \lVert y \rVert \cdot \cos{\alpha}
\end{equation*}
Abbiamo quindi concluso la dimostrazione.


\subsection{Vettori ortogonali} \label{sec_ortogonalita}
\dfn{
	I vettori $x, y \in \mathbb{R}^n$ si dicono ortogonali (o perpendicolari) 
    se $\langle x, y \rangle = 0$
}
Esempi:
\begin{itemize}
	\item Se un vettore è il vettore nullo $x = \underline{0}$ allora tutti i 
        vettori sono ortogonali a lui:
		\begin{equation*}
			\langle \underline{0}, y \rangle = 0 \quad \forall y \in 
            \mathbb{R}^n
		\end{equation*}
	
	\item Se prendiamo $(x, y) = (\cos{\theta}, \sin{\theta})$ e 
        $(\overline{x}, \overline{y}) = \left(\cos(\theta + \dfrac{\pi}{2}), 
        \sin(\theta + \dfrac{\pi}{2}) \right) = (-\sin{\theta}, 
        \cos{\theta})$
		\begin{equation*}
			\langle (x, y), (\overline{x}, \overline{y}) \rangle = 
            \langle (\cos{\theta}, \sin{\theta}), (-\sin{\theta}, 
            \cos{\theta}) \rangle = -\cos{\theta} \cdot \sin{\theta} + 
            \sin{\theta} \cdot \cos{\theta} = 0
		\end{equation*}
	
	\item $(a, b) \in \mathbb{R}^2, (-b, a ) \in \mathbb{R}^2$
		\begin{equation*}
			\langle (a, b), (-b, a) \rangle = -ba + ba = 0
		\end{equation*}
\end{itemize}
Un esempio meno astratto e che in realtà conosciamo già e trovare la retta 
perpendicolare ad un'altra. Se infatti abbiamo due rette generiche 
$r_1: y = mx$ e $r_2: y = px$ possiamo chiederci per quali valori di $p$ la 
seconda sia perpendicolare (o ortogonale) alla prima. Assumiamo per semplicità 
$m \neq 0$ e prendiamo due punti generici, uno su $r_1$ e l'altro su $r_2$:
\begin{equation*}
	(1, m) \in r_1 \qquad (1, p) \in r_2
\end{equation*}
Ora imponiamo che siano perpendicolari, in particolare:
\begin{gather*}
	\langle (1, m), (1, p) \rangle = 0\\[10pt]
	1 + mp = 0 \\[10pt]
	p = -\dfrac{1}{m}
\end{gather*}
Abbiamo quindi ottenuto la condizione di perpendicolarità che ci è familiare 
nel piano.

\subsection{Norma di un vettore} \label{sec_norma}
\dfn{
	Dato $x \in \mathbb{R}^n$, definiamo\footnote{Il prof la indica senza le 
    doppie barre ma come un semplice valore assoluto. Secondo me fa molta 
    confusione quindi io userò sempre le doppie barrette.} la norma come:
	\begin{equation*}
		\lVert x \rVert \vcentcolon = \sqrt{\langle x, x \rangle} = 
        \sqrt{\sum_{k = 1}^n x^2_k}
	\end{equation*}
}
La norma viene spesso chiamata \textbf{lunghezza} perché in effetti rappresenta 
la lunghezza del vettore su cui è calcolata. In $\mathbb{R}$ è il modulo del 
vettore stesso, in $\mathbb{R}^2$ è il semplice teorema di pitagora, mentre in 
spazi di dimensione più grade è già difficile da visualizzare.\\

\textbf{Proprietà:} 
\begin{enumerate}
	\item $\forall \lambda \in \mathbb{R}, \;\forall x \in \mathbb{R}^n$ vale: 
        $\lVert \lambda x \rVert = |\lambda| \cdot \lVert x \rVert$. Questo 
        perché:
		\begin{equation*}
			\lVert \lambda x \rVert = \sqrt{(\lambda x_1)^2 + \cdots + 
            (\lambda x_n)^2} = \sqrt{\lambda^2 (x_1^2 + \cdots + x_n^2)} = 
            |\lambda| \sqrt{x_1^2 + \cdots x_n^2} = |\lambda| \cdot \lVert x 
            \rVert
		\end{equation*}
	
	\item $\forall x \in \mathbb{R}^n \quad \lVert x \rVert \leq 0$
	
	\item $\lVert x \rVert = 0 \iff x = \underline{0}$

	\item $\forall x, y \in \mathbb{R}^n$ vale: $\lVert x + y \rVert \leq 
        \lVert x \rVert + \lVert y \rVert$
\end{enumerate}

E possibile dimostrare che vale la \textbf{disuguaglianza di Cauchy-Schwarz}
\footnote{Per introdurla il prof ha usato il fatto che si può esprimere il 
prodotto scalare tra due vettori $x, y$ come $\lVert x \rVert \cdot \lVert y 
\rVert \cdot \cos{\alpha}$, e quindi essendo $\cos{\alpha} \in [-1, 1] \; 
\forall \alpha \in \mathbb{R}$ si riesce a raggiungere questa disuguaglianza. 
Il fatto però è che secondo questo ragionamento dovrebbe valere solatanto in 
$\mathbb{R}^2$ ma in realtà vale in $\mathbb{R}^n$}:
\begin{equation*}
	\lVert x, y \rVert \leq \lVert x \rVert \cdot \lVert y \rVert \qquad 
    \forall x, y \in \mathbb{R}^n
\end{equation*}

\subsubsection{Teorema di pitagora generalizzato}
\thm{
	Dati $x, y \in \mathbb{R}^n$, se $x \perp y$ allora
	\begin{equation*}
		\lVert x + y \rVert^2 = \lVert x \rVert^2 + \lVert y \rVert^2
	\end{equation*}
}

%Aggiungere figura appunti 2023-03-13

\pf{
	Prendo $x, y \in \mathbb{R}^n$ tali che $x \perp y$. Dalla definzione di 
    norma:
	\begin{align*}
		\lVert x + y \rVert^2 &= \langle x + y, x + y, \rangle = \langle x, 
        x + y, \rangle + \langle y, x + y, \rangle = \\[5pt]
		&= \langle x, x, \rangle + \langle x, y, \rangle + \langle y, x, 
        \rangle + \langle y, y, \rangle = \lVert x \rVert^2 + \lVert y 
        \rVert^2 + 2 \langle x, y \rangle
	\end{align*}

	Per ipotesi però $x \perp y \implies \langle x, y \rangle = 0$ dalla 
    definzione di vettori ortogonali, quindi:
	\begin{equation*}
		\lVert x + y \rVert^2 = \lVert x \rVert^2 + \lVert y \rVert^2
	\end{equation*}
	\hfill Qed.
}

\subsubsection{Vettore normalizzato}
Il vettore normalizzato nasce dall'idea che preso un vettore $x \in 
\mathbb{R}^n: x \neq \underline{0}$, si vuole trovare un numero $r > 0$ tale 
che:
\begin{equation*}
	\lVert rx \rVert = 1
\end{equation*}
Ricordandoci le proprietà della norma: 
\begin{equation*}
	\lVert rx \rVert = |r| \cdot \lVert x \rVert = 1
\end{equation*}
E quindi:
\begin{equation*}
	r = \dfrac{1}{\lVert x \rVert}
\end{equation*}
\dfn{
	Dato un vettore $x \in \mathbb{R}^n: x \neq \underline{0}$ si chiama 
    \textbf{normalizzato di} $\mathbf{x}$ il vettore:
	\begin{equation*}
		\dfrac{x}{\lVert x \rVert}
	\end{equation*}
	E vale:
	\begin{equation*}
		\left \lVert \dfrac{x}{\lVert x \rVert} \right \rVert = 1
	\end{equation*}
}
\textbf{Esempio:} Troviamo il normalizzato di $x = (2, 3)$. Calcoliamo prima 
la sua norma:
\begin{equation*}
	\lVert x \rVert = \lVert (2, 3) \rVert = \sqrt{4+9} = \sqrt{13}
\end{equation*}
Quindi il normalizzato:
\begin{equation*}
	\dfrac{x}{\lVert x \rVert} = \dfrac{(2, 3)}{\sqrt{13}} = \left(\dfrac{2}{
        \sqrt{13}}, \dfrac{3}{\sqrt{13}}\right)
\end{equation*}

\subsubsection{Coordinate polari} \label{sec_coordinatePolari}
Facciamo un'osservazione abbastanza banale che però si rivelerà estremamente 
utile nel caso particolare di $\mathbb{R}^2$: qualsiasi vettore (tranne 
$\underline{0}$) può essere scritto nella forma seguente:
\begin{equation*}
	x = \lVert x \rVert \cdot \dfrac{x}{\lVert x \rVert}
\end{equation*}
Notiamo che $\lVert x \rVert$ è un \textit{numero maggiore di 0} mentre 
$\frac{x}{\lVert x \rVert}$ è un \textit{vettore di lunghezza unitaria}
\footnote{La lunghezza è unitaria perché abbiamo definito la lunghezza di un 
vettore come la sua norma, e per definzione il vettore normalizzato ha norma 
uguale a 1}. Ne consegue che posso scrivere qualsiasi vettore con solo questi 
due elementi, cioè un vettore di lunghezza unitaria e uno scalare. Se riduciamo 
però il nostro studio allo spazio $\mathbb{R}^2$:
\begin{equation*}
	(x, y) = \lVert (x, y) \rVert \cdot \dfrac{(x, y)}{\lVert (x, y) \rVert}
\end{equation*}
Possiamo notare che il nostro vettore normalizzato $\frac{(x, y)}{\lVert (x, y) 
\rVert}$ è in realtà un vettore che sta sulla circonferenza goniometrica in 
quanto siamo sul piano e ha lunghezza 1. Questo implica che possiamo scriverlo 
come segue:
\begin{equation*}
	\dfrac{(x, y)}{\lVert (x, y) \rVert} = (\cos{\theta}, \sin{\theta}) \qquad 
    \theta \in [0, 2\pi]
\end{equation*}
\imp{
Ed essendo $\lVert (x, y) \rVert$ un semplice numero maggiore di 0 
\textbf{possiamo riscrivere qualsiasi vettore in $\mathbb{R}^2$ nella forma:}
\begin{equation*}
	(x, y) = r (\cos{\theta}, \sin{\theta})
\end{equation*}
La coppia $(r, \theta)$ sono le \textbf{coordinate polari} di $(x, y)$
}

%Fare qualche esempio e spiegare cosa le differenzia dalle coordiante normali

Con questo sistema di coordinate si possono inoltre descrivere porzioni di 
piano che prima erano estremamente difficili da scrivere. Un esempio banale è 
il semipiano nel \textit{primo quadrante} delimitato dalla retta $y = x$, esso 
indatti è descrivibile da:
\begin{equation*}
	A = \left\{(x, y) = (r\cos{\theta}, r\sin{\theta}) \; \middle|\; r > 0, 
    \theta \in \left[0, \dfrac{\pi}{2} \right] \right\}
\end{equation*}
Altri esempio più complessi sono i settori circolari. Se infatti volessimo 
descrivere un settore circolare che dista $r_1$ dall'origine ed è largo $d$ 
ci basterebbe scrivere:
\begin{equation*}
	B = \left\{(x, y) = (r\cos{\theta}, r\sin{\theta}) \; \middle|\; \theta 
    \in \left[0, \dfrac{\pi}{2}\right], r_1 < r < r_1 + d \;\; \text{con}\;\; 
    r_1 > 0, d > 0 \right\}
\end{equation*}
%Fare la figura del settore circolare

\subsubsection{Distanza in $\mathbb{R}^n$} 
\dfn{
	La distanza tra due punti $x, y \in \mathbb{R}^n$ è definita come il numero
	\begin{equation*}
		\lVert x - y \rVert
	\end{equation*}
}
