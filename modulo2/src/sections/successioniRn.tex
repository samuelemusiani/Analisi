\section{Successioni in $\mathbb{R}^n$}
\dfn{
	Una successione $(x_k)_{k \in \mathbb{N}} \in \mathbb{R}^n$ è una collezione di successioni in $\mathbb{R}$:
	\begin{equation*}
		x_k = (x^1_k,\; x^2_k,\; x^3_k,\; \cdots,\; x^n_k) \in \mathbb{R}^n
	\end{equation*}

	Da notare che in questo caso i numeri in apice non vogliono indicare un elevamento a potenza ma semplicemente una distinzione tra le varie serie.
}

%Questa la lascio qui o la sposto dai limiti?

\dfn{
	\textbf{Successione convergente}: Data $(x_k)_{k \in \mathbb{N}} \in \mathbb{R}^n$ e dato $a = (a_1, a_2, \cdots, a_n) \in \mathbb{R}^n$ si dice che:
	\begin{equation*}
		x_k \xrightarrow[k \to +\infty]{} a \iff
		\begin{cases}
			\lim_{k \to +\infty} x^1_k = a_1\\
			\vdots\\
			\lim_{k \to +\infty} x^n_k = a_n
		\end{cases}
	\end{equation*}
}
In pratica una successione in $\mathbb{R}^n$ per essere convergente deve essere composta da sole successioni convergenti. Per sempio $x_k$ è una successione convergente mentre $y_k$ no.
\begin{equation*}
	x_k = \left( \dfrac{1}{k}, \dfrac{k+2}{k-1} \right) \qquad y_k = \left( (-1)^k, \dfrac{1}{k} \right)
\end{equation*}
