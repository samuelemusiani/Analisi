\section{Integrali}

\subsection{Calcolo dell'area sottesa ad una curva}
\imp{
	Data una funzione $f:[a,b] \to \mathbb{R}$ con $f(x) \geq 0 \; \forall x \in [a,b]$. IL suo \textbf{sottografico} è:
	\begin{equation*}
		= \{(x,y) \in \mathbb{R}^2 \; | \; x \in [a,b],\, 0 \leq y \leq f(x)\}
	\end{equation*}
}
Il sottografico è quindi un insieme e corrisponde e tutti i punti che soddisfano per le ordinate le disuguaglianza $a \leq x \leq b$ e per le ascisse $0 \leq y \leq f(x)$.

\begin{figure}[h]
	\begin{center}

	\begin{tikzpicture}
		\begin{axis}[xmax = 15, xmin = -0.5, ymax = 8, ymin = -0.5,
			axis lines=middle, xlabel=$x$, ylabel=$y$,
			xtick={2, 10},
			xticklabels={$a$, $b$},
			ytick={10},
			xlabel style={anchor=north east},
			xticklabel style={anchor=north},
			yticklabel style={anchor=east},
		]
		\addplot[domain=-1:15, black, samples=200, name path=A]{ln(x+2) * sin(0.4 * deg(x) + 2) + 3};
		\addplot[domain=-1:15,draw=none,name path=B] {0};     % “fictional” curve
		\addplot[gray, draw opacity = 0, fill opacity = 0.2] fill between[of=A and B,soft clip={domain=2:10}]; % filling
		\end{axis}
	\end{tikzpicture}

	\end{center}
	\caption{Area del sottografico rappresentata in grigio}
	%\label{fig_sottografico}
\end{figure}


Come si calcola il sottografico? 

\subsection{Somme di Riemann}
\subsubsection{Scomposizione di un intervallo}
Sia dato un interallo $[a,b] \subseteq \mathbb{R}$ lo divido in $n \in \mathbb{N}$ intervalli uguali:
%Aggiungere immagine della retta con le tacchette a x_0, x_1, ...
\begin{align*}
	x_0 &= a\\
	x_1 &= x_0 + \frac{b-a}{n} = a + \dfrac{b-a}{n}\\
	x_2 &= x_1 + \frac{b-a}{n} = a + 2 \cdot \dfrac{b-a}{n}\\
	\vdots\\
	x_k &= a + k \cdot \dfrac{b-a}{n}
\end{align*}
Il primo punto corrisponde all'inizio dell'intervallo $x_0 = a$, l'ultimo punto alla fine dell'intervallo $x_n = b$ in quanto:
\begin{equation*}
	x_n = a + n \cdot \dfrac{b-a}{n} = a + b - a = b
\end{equation*}

Posso inoltre scegliere dei punti all'interno di questi intervalli: 
\begin{equation*}
	\forall k \in \mathbb{N} : 0 < k \leq n \quad \text{scelgo} \quad \xi_k \in [x_{k-1}, x_k]
\end{equation*}
È importante notare alcune cose:
\begin{enumerate}
	\item $\xi_k$ è un semplicissimo punto, lo si indica con la lettera greca $\xi$ (xi) per evitare di far confuzione successivamente.

	\item La scelta della posizione del punto $\xi_k$ è \textbf{totalmente arbitraria}. Può quindi essere il punto medio, coincidere con un estremo o essere completamente casuale purché rispetti la condizione imposta, cioè: $\xi_k \in [x_{k-1}, x_k]$

	\item Abbiamo una serie di punti, non solo 1, in quanto questo vale \textit{per ogni k}.
	\begin{align*}
		\xi_1 &\in [x_0, x_1] = \left[a, a + \dfrac{b-a}{n} \right]\\
		\xi_2 &\in [x_1, x_2] = \left[a + \dfrac{b-a}{n}, a + 2 \cdot \dfrac{b-a}{n} \right]\\
		\vdots\\
		\xi_n &\in [x_{n-1}, x_n] = \left[a + (n-1) \cdot \dfrac{b-a}{n}, b \right]
	\end{align*}

\end{enumerate}

\dfn{
	Sia $f:[a,b] \to \mathbb{R}$, continua su $[a, b]$. Sia inoltre $n \in \mathbb{N}$ e siano $x_0, x_1, \cdots x_n$ e $\xi_1, \xi_2, \cdots \xi_n$ i punti introdotti precedentemente. Si definisce \textbf{La somma di Riemann \textit{n-esima}} è il numero:
	\begin{equation*}
		S_n = \sum \limits_{k = 1}^n f(\xi_k) \cdot (x_k - x_{k-1})
	\end{equation*}
	Notando che il termine $(x_k - x_{k-1})$ è sempre uguale si può riscrivere la somma come segue:
	\begin{equation*}
		S_n = \dfrac{b-a}{n} \cdot \sum \limits_{k = 1}^n f(\xi_k) 
	\end{equation*}
}
La somma dipende dalla scelta dei punti $\xi_k$, non è quindi sempre la stessa. Rappresenta inoltre la somma delle aree dei rettangoli che approssimano il sottografico della funzione $f$ nell'intervallo $[a,b]$. %Fare figura

\begin{figure}[h]
	\begin{center}

	\begin{tikzpicture}
		\begin{axis}[xmax = 15, xmin = -0.5, ymax = 8, ymin = -0.5,
			axis lines=middle, xlabel=$x$, ylabel=$y$,
			xtick={2, 4, 6, 8, 10},
			xticklabels={$x_0 = a$, $x_1$, $x_2$, $x_3$, $x_4 = b$},
			ytick={10},
			xlabel style={anchor=north east},
			xticklabel style={anchor=north},
			yticklabel style={anchor=east},
			width=0.8\linewidth,
		]
		\addplot[domain=-1:15, black, samples=200, name path=A]{ln(x+2) * sin(0.4 * x/3.14 * 180 + 2) + 3};
		\addplot[domain=-1:15,draw=none,name path=B] {0};     % “fictional” curve
		
		\draw[-] (2,0) -- (2,4.0279);
		%\draw[-] (4,0) -- (4,4.78798);
		\draw[-] (4,0) -- (4,4.34829);
		\draw[-] (6,0) -- (6,4.34829);
		\draw[-] (8,0) -- (8,2.78172);
		\draw[-] (10,0) -- (10,1.06072);


		%\draw[-, dashed] (2.5, 0) -- (2.5, 4.2936);
		%\draw[-, dashed] (4.7, 0) -- (4.7, 4.78998);
		%\draw[-, dashed] (6.6, 0) -- (6.6, 3.9655);
		%\draw[-, dashed] (9, 0) -- (9, 1.8606);

		\draw[-, dashed] (2,04.0279) -- (4, 4.0279);
		\draw[-, dashed] (4,4.34829) -- (6,4.34829);
		\draw[-, dashed] (6,2.78172) -- (8,2.78172);
		\draw[-, dashed] (8,1.06072) -- (10,1.06072);


		\end{axis}
			
	\end{tikzpicture}

	\end{center}
	\caption{Somme di Riemann}
	%\label{fig_sottografico}
\end{figure}
\subsection{Integrale dalle somme di Riemann} %Non so bene ancora come gestire le sezioni e le sottosezioni

\thm{
	Sia $f:[a,b] \to \mathbb{R}$ continua su $[a,b]$ e $(S_n)_{n \in \mathbb{N}}$ una famiglia di somme di Riemann\footnote{Si noti che questa famiglia in realtà è una successione}. 
	\begin{equation*}
		\exists \lim_{n \to +\infty} S_n \in \mathbb{R}
	\end{equation*}
	Inoltre \textbf{il valore NON dipende dalla scelta dei punti} $\xi_k$. Tale limite si chiama \textbf{integrale di $f$}:
	\begin{equation*}
		\int_a^b f(x) \diff x \vcentcolon = \lim_{n \to +\infty} S_n \in \mathbb{R}
	\end{equation*}
}
Note importanti:
\begin{itemize}
	\item Il valore del limite, essendo in $\mathbb{R}$ è finito.
	
	\item $a$ e $b$ si chiamo \textbf{estremi di integrazione}.

	\item La varibile dentro la funzione è una \textit{variabile muta}. Non indica effettivamente nulla. Le seguenti notazioni sono equvalenti:
		\begin{equation*}
			\int_a^b f(x) \diff x = \int_a^b f(t) \diff t = \int_a^b f
		\end{equation*}
		Si adotta generalmente la variabile muta è il termine $\diff x$ che NON ha alcune definzione o qualsivoglia introduzione matematica per comodità.
\end{itemize}

Vediamo come è definita questa somma in alcuni esempi particolari:
\begin{itemize}
	\item $f(x) \leq 0 \;\; \forall x \in \mathbb[a, b]$ come in figura \ref{fig_sottografico_neg}:
		\begin{equation*}
			\int_a^b f(x) \diff x = - (\text{area del sottografico}) = -text{area}(A)
		\end{equation*}

		\begin{figure}[h]
			\begin{center}

			\begin{tikzpicture}
				\begin{axis}[xmax = 15, xmin = -0.5, ymax = 2, ymin = -8,
					axis lines=middle, xlabel=$x$, ylabel=$y$,
					xtick={2, 10},
					xticklabels={$a$, $b$},
					ytick={10},
					xlabel style={anchor=south east},
					xticklabel style={anchor=south east},
					yticklabel style={anchor=east},
				]
				\addplot[domain=-1:15, black, samples=200, name path=A]{ln(x+2) * sin(0.4 * deg(x) + 2) + -5};
				\addplot[domain=-1:15,draw=none,name path=B] {0};     % “fictional” curve
					\addplot[gray, draw opacity = 0, fill opacity = 0.2] fill between[of=A and B,soft clip={domain=2:10}]; % filling
				\end{axis}
				\draw (3,3.5) node[left] {$A$};
			\end{tikzpicture}

			\end{center}
			\caption{Visualizzazione di un integrale su una funzione negativa}
			\label{fig_sottografico_neg}

		\end{figure}




	\item Se la funzione assume sia valori positivi sia valori negativi (come in figura \ref{fig_sottografico_pos_neg}) allora l'intagrale risulta:
		\begin{equation*}
			\int_a^b f(x) \diff x = \text{area}(A_1) - \text{area}{A_2}
		\end{equation*}


		\begin{figure}[h]
			\begin{center}

			\begin{tikzpicture}
				\begin{axis}[xmax = 15, xmin = -0.5, ymax = 4, ymin = -4,
					axis lines=middle, xlabel=$x$, ylabel=$y$,
					xtick={2, 12},
					xticklabels={$a$, $b$},
					ytick={10},
					xlabel style={anchor=south east},
					xticklabel style={anchor=north},
					yticklabel style={anchor=east},
				]
				\addplot[domain=-1:15, black, samples=200, name path=A]{ln(x+2) * sin(0.4 * deg(x) + 2) };
				\addplot[domain=-1:15,draw=none,name path=B] {0};     % “fictional” curve
				\addplot[gray, draw opacity = 0, fill opacity = 0.2] fill between[of=A and B,soft clip={domain=2:12}]; % filling
				\end{axis}
				\draw (2.5,3.5) node[left] {$A_1$};
				\draw (5,2) node[left] {$A_2$};
			\end{tikzpicture}

			\end{center}
			\caption{Visualizzazione di un integrale su una funzione che cambia segno}
			\label{fig_sottografico_pos_neg}
		\end{figure}

	\item Se il sottografico è formato da due aree positive ma separate (come in figura \ref{fig_sottografico_pos_zero}):

		\begin{equation*}
			\int_a^b f(x) \diff x = \text{area}(A_1) + \text{area}{A_2} 
		\end{equation*}

		\begin{figure}[h]
			\begin{center}

			\begin{tikzpicture}
				\begin{axis}[xmax = 3, xmin = -0.5, ymax = 2, ymin = -0.5,
					axis lines=middle, xlabel=$x$, ylabel=$y$,
					xtick={0.1, 2},
					xticklabels={$a$, $b$},
					ytick={10},
					xlabel style={anchor=south east},
					xticklabel style={anchor=north},
					yticklabel style={anchor=east},
				]
					\addplot[domain=-1:3, black, samples=200, name path=A]{(x-1) * (x-1)};
				\addplot[domain=-1:3,draw=none,name path=B] {0};     % “fictional” curve
				\addplot[gray, draw opacity = 0, fill opacity = 0.2] fill between[of=A and B,soft clip={domain=0.1:2}]; % filling
				\end{axis}
				\draw (2,1.5) node[left] {$A_1$};
				\draw (4.7,1.5) node[left] {$A_2$};
			\end{tikzpicture}

			\end{center}
			\caption{Visualizzazione di un integrale su una funzione che tocca l'asse delle ascisse}
			\label{fig_sottografico_pos_zero}
		\end{figure}

	\item Se $f(x) = k \;\; \forall x \in [a,b]$, cioè in pratica la funzione è costante (come in figura \ref{fig_sottografico_rettangolo}). Consideriamo prima la somma di Riemann: 
		\begin{equation*}
			S_n = \sum \limits_{i = 0}^n f(\xi_i) \cdot \left(\dfrac{b-a}{n}\right) = \sum \limits_{i = 0}^n k \cdot \left(\dfrac{b-a}{n}\right) = n \cdot k \cdot \dfrac{b-a}{n} = k \cdot (b-a)
		\end{equation*}
		Ne consegue che:
		\begin{equation*}
			\int_a^b f(x) \diff x = \lim_{n \to +\infty} S_n = k \cdot (b-a)
		\end{equation*}
		Coindice con infatti l'area di un rettangolo di base $b-a$ e di altezza $k$.


		\begin{figure}[h]
			\begin{center}

			\begin{tikzpicture}
				\begin{axis}[xmax = 3, xmin = -0.5, ymax = 2, ymin = -0.5,
					axis lines=middle, xlabel=$x$, ylabel=$y$,
					xtick={0.3, 2},
					xticklabels={$a$, $b$},
					ytick={10},
					xlabel style={anchor=south east},
					xticklabel style={anchor=north},
					yticklabel style={anchor=east},
				]
				\addplot[domain=-1:3, black, samples=200, name path=A]{1.3};
				\addplot[domain=-1:3,draw=none,name path=B] {0};     % “fictional” curve
				\addplot[gray, draw opacity = 0, fill opacity = 0.2] fill between[of=A and B,soft clip={domain=0.3:2}]; % filling

				\draw (0, 1.3) node[xshift=25pt, yshift=8pt] {$f(x) = k$};

				\end{axis}
				\draw (3.5,2.7) node[left] {$A$};
			\end{tikzpicture}

			\end{center}
			\caption{Visualizzazione di un integrale su una funzione costante}
			\label{fig_sottografico_rettangolo}
		\end{figure}
	
	\item Se $a = b$ allora:
		\begin{equation*}
			\int_a^a f(x) \diff x = 0
		\end{equation*}
		In quanto:
		\begin{equation*}
			S_n = \sum \limits_{i = 0}^n f(\xi_i) \cdot \left(\dfrac{b-a}{n}\right) = \sum \limits_{i = 0}^n f(\xi_i) \cdot \left(\dfrac{a-a}{n}\right) = \sum \limits_{i = 0}^n f(\xi_i) \cdot 0 = 0
		\end{equation*}
\end{itemize}

\subsubsection{Proprietà dell'integrale}
\begin{enumerate}
	\item \textbf{Linearità}: Date due funzioni $f,g:[a,b] \to \mathbb{R}$ continue su $[a,b]$. Deti inoltre due punti $c_1, c_2 \in \mathbb{R}$
		\begin{equation*}
			\int_a^b (c_1 \cdot f(x) + c_2 \cdot g(x)) \diff x = \int_a^b c_1 \cdot f(x) \diff x \int_a^b c_2 \cdot g(x) \diff x = 
		\end{equation*}

	\item \textbf{Additività} Data una funzione $f:[a,b] \to \mathbb{R}$ e dato $c \in [a,b]$:
		\begin{equation*}
			\int_a^b f(x) \diff x = \int_a^c f(x) \diff x + \int_c^b f(x) \diff x
		\end{equation*}
		In generale si tende ad adottare la \textbf{convenzione} che se $b < a$:
		\begin{equation*}
			\int_a^b f(x) \diff x = -\int_b^a f(x) \diff x
		\end{equation*}
		Ne consegue quindi che si può \textbf{generalizzare la seconda proprietà} a: $f:\mathbb{R} \to \mathbb{R},\; \forall a,b,c \in \mathbb{R}$ implica:
		\begin{equation*}
			\int_a^b f(x) \diff x = \int_a^c f(x) \diff x + \int_c^b f(x) \diff x 
		\end{equation*}

	\item Con $f:[a,b] \to \mathbb{R}$ e $f(x) \geq 0 \;\; \forall x \in [a,b]$ allora:
		\begin{equation*}
			\int_a^b f(x) \diff x \geq 0
		\end{equation*}
		Si può \textbf{generalizzare}\footnote{Questo si dimostra con il caso non generalizzato e una funzione ausiliaria $h(x) = f(x) - g(x)$}: data un'ulteriore funzione $g:[a,b] \to \mathbb{R}$, se $f(x) \leq g(x) \quad \forall x \in [a,b]$ allora:
		\begin{equation*}
			\int_a^b f(x) \diff x \geq \int_a^b g(x) \diff x
		\end{equation*}
\end{enumerate}

\subsection{Media integrale}
\thm{
	Se $f:[a, b] \to \mathbb{R}$ continua su $[a,b]$, allora:
	\begin{equation*}
		\exists c \in [a,b] : \dfrac{1}{b-a} \int_a^b f(x) \diff x = f(c)
	\end{equation*}
	Il valore di $f(c)$ si definisce \textbf{media integrale di $f$ in $[a,b]$}.
}
Il nome media deriva dal datto che:
\begin{equation*}
	\dfrac{1}{b-a} S_n = \sum \limits_{k = 1}^n \dfrac{f(\xi_k)}{n}
\end{equation*}
Che non è altro che una media aritmetica.

\pf{
	Dimostraimo il teorema della media integrale: per il teorema di Weierstrass $\exists x_1, x_2$ punti di minimo e massimo assoluti:
	\begin{equation*}
		f(x_1) \leq f(x) \leq f(x_2) \quad \forall x \in [a,b]
	\end{equation*}
	Usando la proprietà della monotonia degli integrali:
	\begin{equation*}
		\int_a^b f(x_1) \diff x \leq \int_a^b f(x) \diff x \leq \int_a^b f(x_2) \diff x
	\end{equation*}
	Ed essendo $f(x_1)$ e $f(x_2)$ valori costanti:
	\begin{equation*}
		(b-a)f(x_1) \leq \int_a^b f(x) \diff x \leq (b-a) f(x_2)
	\end{equation*}
	Che quindi dividendo per $(b-a)$
	\begin{equation*}
		f(x_1) \leq \dfrac{1}{b-a} \int_a^b f(x) \diff x \leq f(x_2)
	\end{equation*}
	Dal teorema dei valori intermedi\footnote{Non mi ricordo di averlo fatto ne di averlo copiato. È sugli appunti del 2023-02-20. CONTROLLARE! :)}:
	\begin{equation*}
		\exists x \in [a,b] : f(x) = y
	\end{equation*}
	Cioè:
	\begin{equation*}
		f(c) = \dfrac{1}{b-a} \int_a^b f(x) \diff x
	\end{equation*}
	\hfill Qed.
}

\subsection{Primitiva}
\dfn{
	Sia $f:]a,b[ \to \mathbb{R}$. Una funzione $F:]a,b[ \to \mathbb{R}$ si dice \textbf{primitiva di $f$ su $]a,b[$} se vale:
	\begin{equation*}
		F'(x) = f(x) \qquad \forall x \in ]a,b[
	\end{equation*}
}
\textbf{Osservazione:} Se $F$ è primitiva di $f$ su $]a,b[$, allora:
\begin{equation*}
	\forall k \in \mathbb{R} \quad F(x) + k \;\; \text{è una primitiva di $f$ su } ]a,b[
\end{equation*}
\imp{
	\begin{center}
		\textbf{Ci sono infinite primitive di una funzione}
	\end{center}
}
\thm{ \label{theorem_caratterizzazionePrimitive}
	\textbf{Teorema di caratterizzazione delle primitive su un intervallo:} Se $F:]a,b[ \to \mathbb{R}$ e $G:]a,b[ \to \mathbb{R}$ sono primitive di $f:]a,b[ \to \mathbb{R}$, allora:
	\begin{equation*}
		\exists k \in \mathbb{R} : F(x) - G(x) = k \qquad \forall x \in ]a,b[
	\end{equation*}
}
\pf{
	Vogliamo dimostrare il teorema appena enunciato. Consideriamo la funzione ausiliaria:
	\begin{equation*}
		H(x) = F(x) - G(x) \qquad \forall x \in ]a,b[
	\end{equation*}
	Se faccio la derivata:
	\begin{equation*}
		H'(x) = F'(x) - G'(x) \qquad \forall x \in ]a,b[
	\end{equation*}
	Che per ipotesi:
	\begin{equation*}
		H'(x) = f(x) - f(x) = 0 \qquad \forall x \in ]a,b[
	\end{equation*}
	Da teorema di Lagrange (in particolare dal suo corollario), se una funzione continua su un intervallo ha derivata sempre nulla allora la funzione è costante. Ne consegue:
	\begin{equation*}
		\exists k \in \mathbb{R} : H(x) = k \qquad \forall x \in ]a,b[
	\end{equation*}
	E quindi:
	\begin{equation*}
		F(x) - G(x) = k \qquad \forall x \in ]a,b[
	\end{equation*}
	\hfill Qed.
}
È importante che la funzione sia definita su un intervallo perché è facile prendere un esempio di funzione non definita su un intervallo e fare vedere che il teorema non funziona. Per sempio presa la funzione $f:\mathbb{R} \setminus \{0\} \to \mathbb{R}, f(x) = \frac{1}{x^2}$ ha come primitive:
\begin{align*}
	F(x) &= -\dfrac{1}{x} \qquad \forall x \neq 0\\
	G(x) &=
	\begin{cases}
		-\dfrac{1}{x} \qquad x < 0\\[10pt]
		-\dfrac{1}{x} + 1 \qquad x > 0
	\end{cases}
\end{align*}
Nonostante ciò:
\begin{equation*}
	F(x) - G(x) \;\; \text{non è costante}
\end{equation*}
Proprio perché non stiamo considerando un intervallo.

\subsection{Funzioni integrali}
\dfn{
	Sia $f:]a,b[ \to \mathbb{R}$ continua e sia $c\in ]a,b[$. Definziamo la \textbf{funzione integrale di \textit{f} con punto base \textit{c}} come:
\begin{equation*}
	I_c:]a,b[ \to \mathbb{R}, \qquad I_c(x) = \int_c^x f(t) \diff t
\end{equation*}
}

%Aggiungere immagine

Osservazione 1:
\begin{equation*}
	I_c(c) = \int_c^c f(t) \diff t = 0
\end{equation*}

Osservazione 2: Consideriamo la funzione $f:]a,b[ \to \mathbb{R}$ continua e un punto $c_1, c_2 \in ]a,b[$. Le funzioni:
\begin{align*}
	I_{c_1} (x) = \int_{c_1}^x f \qquad \text{e} \qquad I_{c_2} (x) = \int_{c_2}^x f\\
\end{align*}
Hanno differenza costante:
\begin{equation*}
	I_{c_1} (x) - I_{c_2} (x) = \int_{c_1}^x f - \int_{c_2}^x f = \int_{c_1}^x f + \int_x^{c_2} f  = \int_{c_1}^{c_2} f
\end{equation*}

\subsubsection{Teoremi fondamentali del calcolo integrale}
Il seguente teorema è spesso indicato come secondo, ma il prof ha deciso di farlo per primo.
\thm{
	Sia $f:]a,b[ \to \mathbb{R}$ continua e sia $c\in ]a,b[$. Allora vale:
	\begin{equation*}
		I_c'(x) = f(x) \qquad \forall x \in ]a,b[
	\end{equation*}
	Cioè:
	\begin{equation*}
		\dv{}{x} \int_c^x f(t) \diff t = f(x) \qquad \forall x \in ]a,b[
	\end{equation*}
}
Esempio: Data $f:\mathbb{R} \to \mathbb{R}, f(x) = e^{x^2}$ se 
\begin{equation*}
	I(x) = \int_0^x e^{t^2} \diff t
\end{equation*}
Allora:
\begin{equation*}
	\dv{}{x} \int_0^x e^{t^2} \diff t = e^{x^2} \qquad \forall x \in \mathbb{R}
\end{equation*}
Oppure un altro esempio:
\begin{equation*}
	\dv{}{x} \int_2^x \sqrt{1+t^2} \cdot e^{-t} \sin(t) \diff t = \sqrt{1+x^2} \cdot e^{-x} \sin(x)
\end{equation*}
Non importa quindi sapere il risultato dell'integrale per sapere la sua derivata.\\

Perché l'estremo inferiore dell'integrale è ininfluente\footnote{Questa è una mia considerazione, il prof non l'ha detta}? Se sostiutiamo l'estremo inferiore $a$ con un qualsiasi $k \in \mathbb{R}$:
\begin{equation*}
	\dv{}{x} \int_a^x f(t) \diff t = \dv{}{x} \left(\int_a^k f(t) \diff t + \int_k^x f(t) \diff t\right) = \dv{}{x} \int_a^k f(t) \diff t + \dv{}{x} \int_k^x f(t) \diff t
\end{equation*}
L'integrale $\int_a^k f$ è una semplice costante, ne consegue che dopo la derivata si annulla, quindi rimane solo l'integrale:
\begin{equation*}
	\dv{}{x} \int_k^x f(t) \diff t = f(x)
\end{equation*}
Ricordiamoci che abbiamo scelto a caso la costante $k$, ne consegue che è ininfluente sotto l'effetto della derivata.

\pf{
	Vogliamo dimostrare il teorema appena enunciato. Supponiamo di avere una funzione $f:]a,b[ \to \mathbb{R}$ e un punto $c \in ]a,b[$. Vogliamo dimostrare che:
	\begin{equation*}
		\dv{}{x} \int_c^x f(t) \diff t = f(x) \qquad \forall x \in ]a,b[
	\end{equation*}
	Per la definzione di derivata dovremmo dimostrare che esiste sia il limite destro che il sinistro e che il loro valore coincide. Ci limitiamo a dimostrare il teorema per il limite destro in quanto è analogo nell'altro caso. Calcoliamo il limite:
	\begin{equation*}
		\lim_{h \to 0^+} \dfrac{\displaystyle\int_c^{x+h} f - \displaystyle\int_c^x f}{h} = \lim_{h \to 0^+} \dfrac{1}{h} \left(\displaystyle\int_c^{x+h} f + \displaystyle\int_x^c f\right)
	\end{equation*}
	Possiamo quindi usare la proprietà di additività degli integrali e ridurci a dimostrare:
	\begin{equation*}
		\lim_{x \to 0^+} \dfrac{1}{h} \int_x^{x+h} f(t) \diff t = f(x)
	\end{equation*}
	Usiamo ora il teorema della media integrale, dove in particolare gli estremi $a = x$ e $b = x + h$ e quindi $b - a = x + h - x = h$:
	\begin{equation*}
		\exists c(h) \in \;\; ]x, x+h[ \;\; : \dfrac{1}{h} \int_x^{x+h} f(t) \diff t = f(c(h))
	\end{equation*}
	In questo caso abbiamo scritto $c(h)$ e non semplicemente $c$ per sottolianere che il punto dipende da $h$. Essendo però $c(h) \in \;]x,x+h[$ diventa che:
	\begin{equation*}
		x < c(h) < x + h
	\end{equation*}
	Facendo il limite per $h \to 0^+$ e dal teorema del confronto:
	\begin{equation*}
		c(h) \xrightarrow[h \to 0^+]{} x
	\end{equation*}
	E quindi:
	\begin{equation*}
		f(c(h)) \xrightarrow[h \to 0^+]{} f(x)
	\end{equation*}
	\hfill Qed.
}

Il seguente viene spesso indicato come il primo teorema fondamentale del calcolo integrale.
\thm{
	Sia $f:]a_0,b_0[ \to \mathbb{R}$ continua. Sia inoltre $F:]a_0,b_0[ \to \mathbb{R}$ la primitiva di $f$ su $]a_0,b_0[$, allora:
	\begin{equation*}
		\forall [a,b] \subseteq ]a_0,b_0[ \quad \int_a^b f(x) \diff x = F(b) - F(a) = [F(x)]_a^b = F(x) |_{x=a}^{x=b} %Da fare meglio il simbolo
	\end{equation*}
}
Esempio: $f(x) = x^2$ e $F(x) = \dfrac{x^3}{3}$ è primitiva di $f$.
\begin{equation*}
	\int_2^3 x^2 \diff x = \left[\dfrac{x^3}{3}\right]_{x=2}^{x = 3} = \dfrac{3^3}{3} - \dfrac{2^3}{3} = \dfrac{19}{3}
\end{equation*}

\pf{
	Vogliamo dimostrare il teorema appena enunciato. Prendiamo una funzione $f:]a_0,b_0[ \to \mathbb{R}$ continua. Assumiamo inoltre che $F:]a_0,b_0[ \to \mathbb{R}$ sia la primitiva di $f$ su $]a_0,b_0[$. Prendiamo ora un punto $c \in ]a_0, b_0[$. La funzione integrale di $f$ con punto base $c$ è definita come segue:
	\begin{equation*}
		I_c(x) = \int_c^x f(t) \diff t \qquad \forall x \in ]a_0, b_0[
	\end{equation*}
	Per il "secondo"\footnote{Generalmente nei libri è indicato come secondo, ma il prof lo ha fatto per primo quindi è il primo teorema che si trova in questa sezione.} teorema del calcolo integrale vale che:
	\begin{equation*}
		I_c'(x) = f(x) \qquad \forall x \in ]a_0, b_0[
	\end{equation*}
	Abbiamo quindi due primitive di $f$ in $]a_0, b_0[$: la prima è $F$ per ipotesi e la seconda è $I_c$ per quanto appena dimostrato. Per il teorema di caratterizzazione delle derivate (Sezione: \ref{theorem_caratterizzazionePrimitive}):
\begin{equation*}
	\exists k \in \mathbb{R}: F(x) = I_c(x) + k \qquad \forall x \in ]a_0,b_0[
\end{equation*}
Ricordiamoci che vogliamo dimostrare che $\forall [a,b] \subseteq ]a_0,b_0[$ :
	\begin{align*}
		\int_a^b f(x) \diff x &= F(b) - F(a) = I_c(b) + k - I_c(a) - k = I_c(b) - I_c(a) = \\[5pt]
		&= \int_c^b f(t) \diff t - \int_c^a f(t) \diff t = \int_c^b f(t) \diff t + \int_a^c f(t) \diff t = \int_a^b f(t) \diff t
	\end{align*}
	Ed essendo la variabile interna all'integrale una varibile muta (cioè è indifferente come la chiamiamo). 
	\hfill Qed.
}

Generalizziamo ora il teorema fondamentale del calcolo integrarle:
\thm{
	Siano $A, I \subseteq \mathbb{R}$ intervalli aperti. Sia inoltre $f: A \to \mathbb{R}$ continua e $h: I \to A$ derivabile con derivata continua. Se $c \in A$ allora:
	\begin{equation*}
		\dv{}{x} \int_c^{h(x)} f(t) \diff t = f(h(x)) \cdot h'(x)
	\end{equation*}
}
Esempio:
\begin{equation*}
	\dv{}{x} \int_0^{\sin(x)} e^{t^2} \diff t = e^{\sin^2(x)} \cdot \cos(x)
\end{equation*}

\pf{
	Dimostraimo il teorema fondamentale del calcolo generalizzato. Chiamiamo in particolare:
	\begin{equation*}
		G(x) = \int_c^{h(x)} f(t) \diff t
	\end{equation*}
	Ora definiamo una funzione integrale $I_c(x)$:
	\begin{equation*}
		I_c (x) = \int_c^z f(t) \diff t
	\end{equation*}
	Per il teorema fondamentale del calcolo integrale vale che:
	\begin{equation*}
		I_c' (z) = f(z)
	\end{equation*}
	La nostra funzione $G$ possiamo riscriverla come:
	\begin{equation*}
		G(x) = I_c(h(x))
	\end{equation*}
	Facendo quindi la derivata:
	\begin{equation*}
		G'(x) = I_c'(h(x)) \cdot h'(x) = f(h(x)) \cdot h'(x)
	\end{equation*}

	\hfill Qed.
}

\subsection{Tabella primitive elementari}
Di seguito riporto la tabella delle derivate al contrario, che serve appunto per calcolare gli integrali immediati. Gli integrali sono riportati volutamente senza estremi. Non è inoltre riportata la costante $C$ che si mette per gli integrale indefiniti in quanto non mi interessa indicare l'infinità delle primitive, ma soltanto una di esse (in quanto caso con $C = 0$).
\imp{
\begin{multicols}{2}
	\begin{enumerate}
		\item $\displaystyle \int x^n \diff x = \dfrac{x^{n+1}}{n+1}$ con $n \in \mathbb{R}\setminus\{-1\}$. Se per caso $n < 0$ bisogna controllare che $x \neq 0$.

		\item $\displaystyle \int \dfrac{1}{x} \diff x = \ln{|x|}$
		\item $\displaystyle \int e^x \diff x = e^x$
		\item $\displaystyle \int \sin{x} \diff x = -\cos{x}$
		\item $\displaystyle \int \cos{x} \diff x = \sin{x}$
		\item $\displaystyle \int \dfrac{1}{\cos^2(x)} \diff x = \displaystyle \int 1 + \tan^2 (x) \diff x = \tan{x}$
		\item $\displaystyle \int \dfrac{1}{\sqrt{1-x^2}} \diff x = \arcsin{x}$
		\item $\displaystyle \int -\dfrac{1}{\sqrt{1-x^2}} \diff x = \arccos{x}$
		\item $\displaystyle \int \dfrac{1}{1+x^2} \diff x = \arctan{x}$
	\end{enumerate}
\end{multicols}
}

\subsection{Tecniche di integrazione}
\subsubsection{Integrazione per parti}
Consideriamo due funzioni $F, g:[a,b] \to \mathbb{R}$ dove $g$ è continua, $g'$ continua, $F$ è derivabile, $F'(x) = f(x) \;\forall x \in [a,b]$, $f$ continua. Dalla regola di derivazione del prodotto di due funzioni ci ricordiamo che:
\begin{equation*}
	\dv{}{x} \left[F(x) \cdot g(x)\right] = f(x)g(x) + F(x)g'(x)
\end{equation*}
Vogliamo ora calcolare il seguente integrale:
\begin{equation*}
	\int_a^b \dv{}{x} \left[F(x) \cdot g(x)\right] \diff x = \int_a^b [f(x)g(x) + F(x)g'(x)] \diff x
\end{equation*}
Usando il teorema fondamentale del calcolo integrale e spezzando l'integrale di destra:
\begin{equation*}
	\left[F(x) \cdot g(x)\right]_a^b = \int_a^b f(x)g(x) \diff x + \int_a^b F(x)g'(x) \diff x
\end{equation*}
Ne consegue quindi che:
\imp{
\begin{equation*}
	\int_a^b f(x)g(x) \diff x = \left[F(x) \cdot g(x)\right]_a^b - \int_a^b F(x)g'(x) \diff x
\end{equation*}
}

Questa formula ci permette quindi di semplificare alcuni tipi di integrali del prodotto di funzioni. Se infatti per una di esse riusciamo a trovare una primitiva, allora possiamo riscrivere il nostro integrale usando quella primitiva e la derivata dell'altra funzione. È importante notare che abbiamo due funzioni e di una di esse dobbiamo trovare una primitiva, mentre per l'altra la sua derivata. La scelta è generalmente arbitraria su quale derivare e quale invece riscrivere come primitiva, ma di solito solo uno modo porta alla soluzione. Non ci sono particolari consigli da dare in questo caso se non fare molti esercizi.\\

Un classico esempio che si porta di solito per far vedere l'utilità di questa formula (o in generale tecnica) è il calcolo del seguente integrale:
\begin{equation*}
	\int_a^b e^x \cdot x\diff x
\end{equation*}
In questo caso scegliamo di derivare $x$ e di trovare una primitiva di $e^x$. È facile notare che la primitiva di quest'ultima è la funzione stessa $e^x$. Possiamo quindi riscrivere il nostro integrale nel seguente modo:
\begin{equation*}
	\int_a^b e^x \cdot x \diff x = [e^x \cdot x]_a^b - \int_a^b e^x \cdot 1 \diff x = [e^x \cdot x]_a^b - [e^x]_a^b
\end{equation*}
Notiamo che se avessimo fatto la scelta inversa, cioè di derivare $e^x$ e di trovare una primitiva di $x$ l'integrale si sarebbe complicato e non saremmo riusciti a risolverlo:
\begin{equation*}
	\int_a^b e^x \cdot x\diff x = \left[e^x \cdot \dfrac{x^2}{2}\right]_a^b - \int_a^b e^x \cdot \dfrac{x^2}{2} \diff x
\end{equation*}
Alcuni integrali possono richiedere \textbf{un'applicazione ripetuta} della formula di integrazione per parti. Un esempio è il seguente\footnote{Non ripoterò gli estremi di integrazione ma farò solo l'integrale indefinito per semplicità. In pratica mi riduco solo a trovare le primitive della funzione. Nel caso in cui si voglia trovare l'integrale della funzione tra due estremi basta prendere una delle primitive finali trovate e fare la classica formula data nel teorema fondamentale del calcolo integrale.}:
\begin{align*}
	\int x^2 \cdot e^x \diff x &= e^x \cdot x^2 - \int e^x \cdot 2 x \diff x = e^x \cdot x^2 - 2 \int e^x \cdot x \diff x = \\
	&= e^x \cdot x^2 - 2 \left(e^x \cdot x - \int e^x \diff x \right) = e^x \cdot x^2 - 2 (e^x \cdot x - e^x) + C
\end{align*}

Come è facile notare questo tipo di integrali hanno in comune le così dette \textbf{funzioni circolari}, cioè quelle che hanno come derivata o se stesse oppure una funzione molto simile (tipo $\sin$ e $\cos$). Infatti il seguente integrale si calcola sempre con la formula appena introdotta:

\begin{align*}
	\int x \sin(x) \diff x &= -\cos(x) \cdot x - \int -\cos(x) \diff x = -\cos(x) \cdot x + \sin(x) + C
\end{align*}

Un altro tipo particolare di integrali che si fa con questa tecnica sono i seguenti:
\begin{equation*}
	\int \ln(x) \diff x
\end{equation*}
In particolare possiamo osservare che apparentemente non ci sono due funzioni, quindi sembrerebbe che la formula non si possa applicare. In realtà possiamo riscrivere l'integrale come segue:
\begin{equation*}
	\int 1 \cdot \ln(x) \diff x
\end{equation*}
In questo caso la funzione che dobbiamo integrare $f(x) = 1$, cioè la funzione costante. Una delle sue primitive è $x$, qindi possiamo applicare la formula di integrazione per parti:
\begin{equation*}
	\int 1 \cdot \ln(x) \diff x = x \cdot \ln(x) - \int x \cdot \dfrac{1}{x} \diff x = x \cdot \ln(x) - \int 1 \diff x = x \ln(x) - x + C
\end{equation*}
Un ulteriore intgrale che si fa con la stessa tecnica è il seguente:
\begin{align*}
	\int \arctan(x) \diff x &= x \cdot \arctan x - \int x \cdot \dfrac{1}{1 + x^2} \diff x = x \cdot \arctan(x) - \dfrac{1}{2} \int \dfrac{2x}{1+x^2} \diff x = \\
	&= x \arctan(x) - \dfrac{1}{2} \ln(1 + x^2) + C
\end{align*}

Esistono anche integrali che sono il prodotto di due funzioni circolari, ad esempio:
\begin{equation*}
	\int e^x \cdot sin(x) \diff x
\end{equation*}
In questo caso si nota che se si prova ad applicare la formula di intgrazione per parti vista sopra si entra in un "loop". Per risolvere questo tipo di integrali bisogna applicarla due volte e fare qualche manipolazione algebrica. In questi casi è indifferente la scelta della funzione da derivare e quella da integrare:
\begin{align*}
	\int e^x \cdot \sin(x) \diff x &= e^x \cdot \sin(x) - \int e^x \cos(x) \diff x = e^x \cdot \sin(x) - \left(e^x \cdot \cos(x) - \int e^x \cdot -\sin(x) \diff x\right) = \\
	&= e^x \cdot \sin(x) - e^x \cdot \cos(x) - \int e^x \cdot \sin(x) \diff x
\end{align*}
Notiamo ora che $\int e^x \sin(x) \diff x$ compare da entrambe le parti dell'uguaglianza, quindi per comodità chiamiamo il nostro integrale $I$ e riscriviamo:
\begin{equation*}
	I = e^x \cdot \sin(x) - e^x \cdot \cos(x) - I
\end{equation*}
Risolvendo per $I$:
\begin{equation*}
	I = \dfrac{1}{2} e^x \cdot (\sin(x) - \cos(x)) + C
\end{equation*}
In questo caso bisogna aggiungere $+ C$ in quanto stiamo sempre trattando con tutte le primitive della funzione. Però possiamo finalmente concludere che:
\begin{equation*}
	\int e^x \cdot \sin(x) \diff x = \dfrac{1}{2} e^x \cdot (\sin(x) - \cos(x)) + C
\end{equation*}

\subsubsection{Integrazione per sostituzione}
Gli integrali per sostituzione ci permettono di sotituire delle funzioni all'interno dell'integrale con delle funzioni "più semplici", in generale per semplificare notevolmente i calcoli. Per sempio se abbiamo il seguente integrale:
\begin{equation*}
	\int \dfrac{\sin(x)}{\cos(x)} \diff x
\end{equation*}
Possiamo fare una sostituzione tipo $t = \cos(x)$ e l'integrale diventa:
\begin{equation*}
	\int \dfrac{-1}{t} \diff t = -\ln (t) + C = -\ln(\cos(x)) + C
\end{equation*}
Non importa capire adesso come ha fatto la funzione $\sin(x)$ ha sparire completamente, ma piuttosto apprezzare la semplificazione dei calcoli data dalla sostituzione.

\thm{
	\textbf{Teorema del cambio di variabile}: Siano $I, A \subseteq \mathbb{R}$ intervalli aperti. Data $h: I \to A$ continua e con derivata continua e $f: A \to \mathbb{R}$ continua, vale che:
	\begin{equation*}
		\forall \alpha, \beta \in I \quad \int_{h(\alpha)}^{h(\beta)} f(x) \diff x = \int_\alpha^\beta f(h(t)) \cdot h'(t) \diff t
	\end{equation*}
}
\pf{
	Siano $h$ ed $f$ come da enunciato. Definisco $F: I \to \mathbb{R}$ dove:
	\begin{equation*}
		F(z) = \int_{h(\alpha)}^{h(z)} f(x) \diff x
	\end{equation*}
	Inoltre definisco $G: I \to \mathbb{R}$ dove:
	\begin{equation*}
		G(z) = \int_\alpha^z f(h(t)) \cdot h'(t) \diff t
	\end{equation*}
	Vogliamo quindi provare che sono la stessa funzione. Per il teorema di Lagrange abiamo che se due funzione hanno la stessa derivata allora differiscono per una costante. Se riusciamo a provare che hanno la stessa derivata e che la costante che le differenzia è 0 allora sono la stessa funzione\footnote{Sarebbe da chiedere meglio al prof perché non lo ha specificato nella prova ed è una mia supposizione che funzioni così.}. 

	Dimostriamo che hanno la stessa derivata, cioè:
	\begin{equation*}
		F'(x) = G'(x)
	\end{equation*}
	Usando il teorema del calcolo integrale e quello generalizzato:
	\begin{equation*}
		f(h(z)) \cdot h'(z) = f(h(z)) \cdot h'(z)
	\end{equation*}
	Dimostriamo ora che la costante per cui differiscono è 0. In particolare scelto un punto a caso la loro differenza deve essere 0. Scegliamo per comodità il punto $z = \alpha$:
	\begin{equation*}
		F(\alpha) - G(\alpha) = \int_{h(\alpha)}^{h(\alpha)} \cdots - \int_{\alpha}^{\alpha} \cdots = 0 - 0 = 0
	\end{equation*}
	Non ho riportato il contenuto degli integrali volutamente perché non era necessario per far vedere che erano entrambi 0. 

	\hfill Qed.
}
Facciamo degli esempi:
\begin{equation*}
	\int_3^5 e^{\sqrt{x}} \diff x 
\end{equation*}
Scegliamo $t = \sqrt{x}$, quindi $x = t^2$ e quindi $\diff x = 2t \diff t$: 
\begin{align*}
	\int_{\sqrt{3}}^{\sqrt{5}} e^t \cdot 2t \diff t &= 2\left( [e^t \cdot t]_{\sqrt{3}}^{\sqrt{5}} \;- \int_{\sqrt{3}}^{\sqrt{5}} e^t \diff t\right) = 2 \cdot [e^t (t - 1)]_{\sqrt{3}}^{\sqrt{5}}
\end{align*}
Un altro esempio potrebbe essere (lo faccio indefinito per evitare di portarmi dietro gli estremi di integrazione per tutto il calcolo):
\begin{align*}
	\int \sin(x^{\frac{1}{3}}) \diff x
\end{align*}
Facciamo la sostituzione $t = x^{\frac{1}{3}}$, quindi $x = t^3$ e quindi $\diff x = 3 t^2 \diff t$:
\begin{align*}
	\int \sin(x^{\frac{1}{3}}) \diff x &= \int \sin(t) \cdot 3t^2 \diff t = 3 \int t^2 \sin(t) \diff t = 3 \left(-\cos(t) \cdot t^2 + 2\int t \cdot \cos(t) \diff t \right) =\\
	&= 3 \left(-\cos(t) \cdot t^2 + 2\left(t \cdot \sin(t) - \int \sin(t) \diff t \right)\right) =\\
	&= 3 (-\cos(t) \cdot t^2 + 2(t \cdot \sin(t) + \cos (t) )) + C = 3 (-\cos(t) \cdot t^2 + 2t \cdot \sin(t) + 2\cos (t) ) + C=\\
	&= -3\cos(t) \cdot t^2 + 6t \cdot \sin(t) + 6\cos (t) + C =  -3\cos(x^{\frac{1}{3}}) \cdot x^{\frac{2}{3}} + 6x^{\frac{1}{3}} \cdot \sin(x^\frac{1}{3}) + 6\cos (x^\frac{1}{3}) + C
\end{align*}

Un ulteriore esempio che può essere considerato "utile" è il calcolo dell'area del cerchio. In particolare, essendo la circonferenza unitaria definita dall'equazione $x^2 + y^2 = 1$, possiamo considerare solo la porzione nel primo quadrante del piano, che ha equazione $\sqrt{1-x^2}$ e successivamente moltiplicare per $4$ l'area ottenuta. In particolare dobbiamo calcolare:
\begin{equation*}
	4 \int_0^1 \sqrt{1-x^2} \diff x
\end{equation*}
In questo caso la sostituzione da fare è $x = \cos(t)$, quindi $\diff x = -\sin(t) \diff t$:
\begin{align*}
	4 \int_0^1 \sqrt{1-x^2} \diff x &= 4 \int_{\arccos(0)}^{\arccos{1}} \sqrt{1-\cos^2(t)} \cdot - \sin(t) \diff t = \\[10pt]
	&= -4 \int_{\frac{\pi}{2}}^0 \sin(t) \cdot \sin(t) \diff t = 4 \int_0^{\frac{\pi}{2}} \sin^2(t)\diff t = \\[10pt]
	&= 4 \int_0^{\frac{\pi}{2}} \dfrac{1 - \cos(2t)}{2}\diff t = 2 \int_0^{\frac{\pi}{2}} 1 \diff t - 2 \int_0^{\frac{\pi}{2}} \cos(2t) \diff t = \\[10pt]
	&= 2[t]_0^{\frac{\pi}{2}} - 2\left[\dfrac{1}{2} \sin(2t) \right]_0^{\frac{\pi}{2}} = 2 \left[\dfrac{\pi}{2} - 0\right] - [\sin(\pi) - \sin(0)] = \\[10pt]
	&= \pi - 0 = \pi
\end{align*}


\subsubsection{Integrali di fratte}
Il prof non li ha fatti molto e ha detto al tutor di coprirli. Per completezza mi sento in dovere di scrivere comunque qualcosa al riguardo. DA FARE :) %%DA FARE 

\subsection{Integrali su intervalli non limitati (integrali generalizzati)}

\dfn{
	Data $f:[a, +\infty[ \to \mathbb{R}$ continua, si dice che $f$ è integrabile su $[a, +\infty[$ se esiste \textbf{finito}: 
	\begin{equation*}
		\lim_{z \to +\infty} \int_a^z f(x) \diff x = \vcentcolon \int_a^{+\infty} f(x) \diff x
	\end{equation*}
}
Il limite infatti può divergere o anche non esistere. È però richiesto che se la funzione è integrabile in quell'intervallo allora il limite è finito.

Esempio:
\begin{equation*}
	f(x) = \dfrac{1}{x^2} \qquad x \geq 1
\end{equation*}
Diventa quindi che:
\begin{equation*}
	\int_1^{+\infty} \dfrac{1}{x^2} \diff x = \lim_{z \to +\infty} \left[-\dfrac{1}{x}\right]_1^z = \lim_{z \to +\infty} \left[-\dfrac{1}{z} + \dfrac{1}{1}\right] = -0 + 1 = 1
\end{equation*}
Un altro esempio:
\begin{equation*}
	\int_1^{+\infty} \dfrac{1}{x} \diff x = \lim_{z \to +\infty} [\ln|x|]_1^z = \lim_{z \to +\infty} [\ln|z| - 0] = +\infty
\end{equation*}
In questo caso il limite non converge. In generale:
\begin{equation*}
	\int_1^{+\infty} \dfrac{1}{x^p} \diff x =
	\begin{cases}
		\text{converge} &\text{ se } p > 1\\[5pt]
		\text{diverge} &\text{ se } p \leq 1 \\
	\end{cases}
\end{equation*}
Infatti per $p = 1$ lo abbiamo già dimostrato che diverge, mentre per $p \neq 1$:
\begin{equation*}
	\int_1^{+\infty} \dfrac{1}{x^p} = \lim_{z \to +\infty}\left[\dfrac{x^{-p+1}}{-p+1}\right]_1^z = \lim_{z \to +\infty} \dfrac{z^{1-p}}{1-p} - \dfrac{1}{1-p} = 
	\begin{cases}
		-\dfrac{1}{1-p} &\text{ se } p > 1\\[5pt]
		+\infty &\text{ se } p < 1\\
	\end{cases}
\end{equation*}

\dfn{
	Data $f:]a,b] \to \mathbb{R}$ continua, si dice che $f$ è integrabile su $]a,b]$ se esiste \textbf{finito}:
	\begin{equation*}
		\int_a^b f(x) \diff x = \lim_{z \to a^+} \int_z^b f(x) \diff x
	\end{equation*}
}
Esempio:
\begin{equation*}
	\int_0^1 \dfrac{1}{\sqrt{x}} \diff x = \lim_{z \to 0} [2 \sqrt{x}]_z^1 = \lim_{z \to 0} [2 \sqrt{1} - 2 \sqrt{z}] = 2 
\end{equation*}
Altro esepio:
\begin{equation*}
	\int_0^1 \dfrac{1}{x} = \lim_{z \to 0} [2 \ln(x)]_z^1 = \lim_{z \to 0} [2 \ln(1) - 2 \ln(z)] = +\infty
\end{equation*}


\subsection{Altro}
Il prof ha fatto le seguenti proposizioni, io le riporto per completezza ma mi sembra veramente inutile.\\
\textbf{Proposizione:} Se $f$ è dispari e continua in $\mathbb{R}$:
\begin{equation*}
	\forall a, b \in \mathbb{R} \quad \int_a^b f(x) \diff x = - \int_{-b}^{-a} \diff x
\end{equation*}
\textbf{Dimostrazione:} 
\begin{equation*}
	\int_a^b f(x) \diff x
\end{equation*}
Sostituiamo $t = -x$, cioè $x = -t$ e quindi $\diff x = - \diff t$. 
\begin{align*}
	\int_a^b f(x) \diff x = \int_{-a}^{-b} f(-t) \cdot -1 \diff t = - \int_{-a}^{-b} - f(t) \diff t = -\int^{-a}_{-b} f(t) \diff t
\end{align*}
\textbf{Proposizione:} Se $f$ è pari:
\begin{equation*}
	\forall a, b \in \mathbb{R} \quad \int_a^b f(x) \diff x = \int_{-b}^{-a} \diff x
\end{equation*}
