\section{Dimostrazioni (non essenziali)}


\subsection{Quadrato di un numero dispari}

\pf{
    \textbf{Ipotesi}: $n$ dispari $\implies$ $n^2$ dispari.\\
    
    $n$ dispari $\implies \exists k \in \mathbb{N} : n = 2k + 1$
    
    \begin{equation*}
        n^2 = (2k+1)^2 = 4k^2 + 4k + 1 = 2(2k^2+2k) + 1
    \end{equation*}
    
    Se sostituiamo $2k^2 + 2k$ con un qualsivoglia numero $m \in \mathbb{N} : m = 2k^2 + 2k$ l'espressione potremmo riscriverla come $2m + 1$, che è ovviamente un numero dispari.\\
    
    \textbf{Tesi}: $n$ dispari $\implies$ $n^2$ dispari.
    \label{QuadratoDispari}
    \hfill Q.e.d.
}
Come si vede nella sezione \ref{dimostrazioneNegazione}, questa dimostrazione ci permette di dire che, in quanto $p \implies q = \overline{q} \implies \overline{p}$, $n^2$ \textit{pari} $\implies n$ \textit{pari}.