\section{Derivate}

\subsection{Teorema di Fermat}
Il seguente teorema enuncia che la derivata nei minimi e nei massimi si annulla.
\thm {
Data una funzione $f: [a,b] \to \mathbb{R}$, un punto di massimo o di minimo $x_0 \in ]a,b[$ e inoltre $f$ è derivabile in $x_0$, allora:
\begin{equation*}
    f'(x_0) = 0
\end{equation*}
}
È essenziale che il punto $x_0$ si all'interno dell'intervallo! %DARE VEDERE FOTO DI UNA RETTA CON DOMIONIO RISTRETTO
Inoltre l'annullarsi della derivata prima in un punto $x_0$ è condizione \textbf{necessaria} affinché $x_0$ sia un punto di massimo o di minimo relativo, ma \textbf{non è sufficiente} in generale (es. $x^3$ in $x=0$). %FAR VEDERE GRAFICO

\subsection{Teorema di Rolle}
\thm {
$f:[a,b] \to \mathbb{R}$
\begin{enumerate}
    \item $f$ è continua su $[a,b]$
    \item $f$ è derivabile su $]a,b[$
    \item $f(a) = f(b)$
\end{enumerate}
Allora:
\begin{equation*}
    \exists \,c \in ]a,b[ \;: f'(c) = 0
\end{equation*}
}

\subsection{Teorema di Lagrange}
\thm {
$f:[a,b] \to \mathbb{R}$
\begin{enumerate}
    \item $f$ è continua su $[a,b]$
    \item $f$ è derivabile su $]a,b[$
\end{enumerate}
Allora:
\begin{equation*}
    \exists \,c \in ]a,b[ \;: \dfrac{f(b)-f(a)}{b-a} = f'(c)
\end{equation*}
}
Corollario: se una funzione ha derivata sempre zero è costante.

\subsection{Teorema di Cauchy}
\thm {
$f,g:[a,b] \to \mathbb{R}$
\begin{enumerate}
    \item $f,g$ è continua su $[a,b]$
    \item $f,g$ è derivabile su $]a,b[$
    \item $g'(x) \neq 0 \forall x\in ]a,b[$
\end{enumerate}
Allora:
\begin{equation*}
    \exists \,c \in ]a,b[ \;: \dfrac{f(b)-f(a)}{g(b)-g(a)} = \dfrac{f'(c)}{g'(c)}
\end{equation*}
}

\subsection{I teoremi di De L'Hôpital}